% conto-intesa-guida-etf.tex -*- coding: utf-8-unix -*-

\documentclass[12pt,a4paper]{article}
\renewcommand{\rmdefault}{ptm} % Times Roman font, for PDF output
\usepackage[italian]{babel}
\usepackage[utf8]{inputenc}
\usepackage[T1]{fontenc}
\usepackage[]{textcomp}
\usepackage[]{eurosym}
\usepackage[]{amsmath}
\usepackage[]{amsfonts}
\usepackage[]{amsthm}
\usepackage[]{amssymb}
\usepackage[]{syntonly}
\usepackage[copy-decimal-marker,retain-explicit-plus]{siunitx}
\usepackage[hidelinks]{hyperref} % To generate PDFs with hyperlinks

\pagestyle{headings}

%page
%% ------------------------------------------------------------
%% Comandi.
%% ------------------------------------------------------------

\newcommand{\Undefine}[1]{\let#1\CustomUndefined}
\newcommand{\Define}[2]{%
\Undefine{#1}
\newcommand{#1}{#2}
}

\newcommand{\Perc}[1]{\SI{#1}{\percent}}
\newcommand{\Eur}[1]{\SI{#1}{\text{\euro{}}}}

\newcommand{\MediaPonderataDue}[4]{\frac{\num{#1} \times{} \num{#2} + \num{#3} \times{} \num{#4}}{\num{#1} + \num{#3}}}
\newcommand{\MediaPonderataTre}[6]{\frac{\num{#1} \times{} \num{#2} + \num{#3} \times{} \num{#4} + \num{#5} \times{} \num{#6}}{\num{#1} + \num{#3} + \num{#5}}}

\newcommand{\CalcoloCostoOperazione}[1]{\num{0,50} + \num{2,50} + \num{0,0024} \times{} \num{#1}}

\newcommand{\CalcoloRendimentoPercentuale}[2]{\frac{\num{#1} - \num{#2}}{#2} \times{} \num{100}}

\newcommand{\Parentesi}[1]{(#1)}
\newcommand{\Virgolette}[1]{``#1''}
\newcommand{\Etf}[1]{\textsc{etf}}

%% ------------------------------------------------------------------------

% Prezzo medio di effettivo, prezzo medio eseguito
\newcommand{\Pme}[1]{P_{\mathrm{me}#1}}

% Prezzo medio effettivo di acquisto
\newcommand{\Pmea}[1]{\Pme{\mathrm{a}#1}}

% Prezzo medio effettivo di vendita
\newcommand{\Pmev}[1]{\Pme{\mathrm{v}#1}}

% Prezzo medio effettivo nel saldo
\newcommand{\Pmes}[1]{\Pme{\mathrm{s}#1}}

% Prezzo medio di carico
\newcommand{\Pmc}[1]{P_{\mathrm{mc}#1}}

% Prezzo medio di carico di acquisto
\newcommand{\Pmca}[1]{\Pmc{\mathrm{a}#1}}

% Prezzo medio di carico nel saldo
\newcommand{\Pmcs}[1]{\Pmc{\mathrm{s}#1}}

% Prezzo medio netto di vendita
\newcommand{\Pmnv}[1]{P_{\mathrm{mnv}#1}}

%page
%% ------------------------------------------------------------
%% Intestazione
%% ------------------------------------------------------------

\author{Marco Maggi}
\title{Pericolosa e incompleta guida agli investimenti in \Etf{} con \emph{Banca Intesa Sanpaolo}}

\begin{document}

\maketitle

\begin{abstract}
  \noindent
  Questa guida è un  aiuto per chi volesse ricostruire i calcoli di  contabilità del Conto Titoli di
  \emph{Banca   Intesa   Sanpaolo}   per   gli  investimenti   in   \emph{Exchange   Traded   Funds}
  \Parentesi{\Etf{}}.   Questa guida  è incompleta:  non copre  tutto ciò  che c'è  da sapere  sugli
  \Etf{}, né l'operatività del sito di  \emph{home banking} di \emph{Banca Intesa Sanpaolo}.  Questa
  guida è pericolosa: ognuno la usa a suo rischio e pericolo.

  In questa guida si tenta di mantenere  il testo matematicamente ``facile''; si evita l'utilizzo di
  simboli matematici nelle  formule; chi ha frequentato  le Scuole Medie Inferiori, e  ne ricorda le
  basi del programma di matematica, dovrebbe farcela.
\end{abstract}

\tableofcontents

\newpage{}

\noindent
Copyright \copyright{} 2017, 2018, 2020 Marco Maggi \texttt{<mrc.mgg@gmail.com>}.

Permission is  granted to copy, distribute  and/or modify this document  under the terms of  the GNU
Free  Documentation License,  Version  1.3 or  any  later  version published  by  the Free  Software
Foundation; with no  Invariant Sections, no Front-Cover  Texts, and no Back-Cover Texts.   A copy of
the license is included in the section entitled ``GNU Free Documentation License''.

Il codice sorgente di questa guida è disponibile in Rete all'indirizzo:
\begin{center}
  \url{https://github.com/marcomaggi/conto-intesa-guida-etf}
\end{center}

\newpage{}

%page
\section{Introduzione}


Il calcolo  dei redditi da  capitale e dei  redditi diversi per  investimenti in \Etf{}  deve essere
eseguito come specificato nella ``Circolare dell'Agenzia dell'Entrate 21/E del 10 Luglio 2014''.  Il
calcolo di questi redditi permette sia la determinazione dei rendimenti che il calcolo delle tasse.

Per ogni operazione di compravendita eseguita sul  Conto Titoli: nella sezione documenti del sito di
\emph{home banking} è disponibile una \emph{nota di eseguito} con tutti i dati dell'operazione.  Nei
calcoli qui  illustrati: alcuni valori  riportati nelle  note di eseguito  sono i dati  di partenza;
altri valori sono i risultati ricalcolati per verifica.

Questa  guida utilizza  sia la  terminologia  della Banca  che quella  della circolare  dell'Agenzia
dell'Entrate.   Da notare  che i  prezzi di  acquisto  e vendita  sulle Borse  di negoziazione  sono
chiamati \emph{effettivi} dalla circolare.

Tutti i numeri in questa guida sono presentati arrotondati, il piú delle volte al centesimo di Euro;
l'esecuzione dei calcoli è invece eseguita con precisione maggiore.

%page
\section{Media aritmetica e media ponderata}


Nei calcoli di contabilità  Conto Titoli si usa spesso l'operazione di  \emph{media aritmetica} e in
particolare la sua riscrittura come \emph{media ponderata};  è utile richiamare queste idee e alcune
delle loro proprietà.

Si ricorda che la \textbf{media aritmetica} tra i numeri \num{11}, \num{22} e \num{33} si scrive:
\begin{equation*}
  \frac{\num{11} + \num{22} + \num{33}}{3} = \num{22}
\end{equation*}
il denominatore è \num{3} perché al numeratore  ci sono \num{3} numeri; la \emph{media aritmetica} è
una procedura di calcolo il cui risultato è il \emph{valore medio}.

L'operazione si costruisce  nello stesso modo se qualche  numero compare piú di una  volta; la media
aritmetica tra i numeri: \num{11}, \num{11}, \num{11}, \num{22}, \num{33} e \num{33} si scrive:
\begin{equation*}
  \frac{\num{11} + \num{11} + \num{11} + \num{22} + \num{33} + \num{33}}{6}
  = \frac{121}{6} \simeq \num{20,17}
\end{equation*}
il denominatore è \num{6} perché:
\begin{itemize}
\item \num{11} compare \num{3} volte;
\item \num{22} compare \num{1} volta;
\item \num{33} compare \num{2} volte;
\end{itemize}
quindi \(\num{3} + \num{1} + \num{2} = \num{6}\).

Il \Virgolette{numero di volte in cui un  numero compare} si chiama \textbf{molteplicità}; allora si
può dire che:
\begin{itemize}
\item \num{11} compare con \emph{molteplicità} \num{3};
\item \num{22} compare con \emph{molteplicità} \num{1};
\item \num{33} compare con \emph{molteplicità} \num{2};
\end{itemize}
evidenziando le molteplicità e considerando le semplici identità:
\begin{align*}
  \num{11} + \num{11} + \num{11} &= \num{3} \times{} \num{11} &&&
  \num{22} &= \num{1} \times{} \num{22} &&&
  \num{33} + \num{33} &= \num{2} \times{} \num{33}
\end{align*}
l'espressione della \emph{media aritmetica} si può riscrivere:
\begin{align*}
  &\frac{\num{11} + \num{11} + \num{11} + \num{22} + \num{33} + \num{33}}{6}
  = \frac{\num{3} \times{} \num{11}
    + \num{1} \times{} \num{22}
    + \num{2} \times{} \num{33}}
    {6} = \\
  &=
    \frac{\num{3} \times{} \num{11}}{6} +
    \frac{\num{1} \times{} \num{22}}{6} +
    \frac{\num{2} \times{} \num{33}}{6}
  =
    \frac{\num{3}}{6} \times{} \num{11} +
    \frac{\num{1}}{6} \times{} \num{22} +
    \frac{\num{2}}{6} \times{} \num{33}
\end{align*}
in cui si evidenzia come:
\begin{itemize}
\item il numero \num{11} compaia con ``peso'' \num{3} rispetto al totale \num{6};
\item il numero \num{22} compaia con ``peso'' \num{1} rispetto al totale \num{6};
\item il numero \num{33} compaia con ``peso'' \num{2} rispetto al totale \num{6}.
\end{itemize}

In questo  modo: la  \emph{media aritmetica}  tra i numeri  \num{11}, \num{11},  \num{11}, \num{22},
\num{33} e \num{33} è riscritta come media  \textbf{media ponderata} tra i numeri \num{11}, \num{22}
e \num{33} rispetto alle loro molteplicità \num{3}, \num{1} e \num{2}.

Il significato del \emph{valore medio} si può evidenziare osservando che se vale l'identità:
\begin{align*}
  \frac{\num{3} \times{} \num{11} + \num{1} \times{} \num{22}
  + \num{2} \times{} \num{33}}{\num{6}}
  \simeq \num{20,17}
\end{align*}
allora vale anche l'identità:
\begin{align*}
  \frac{\num{3} \times{} \num{11} + \num{1} \times{} \num{22}
  + \num{2} \times{} \num{33}}{\num{6}}
  \simeq \frac{\num{6} \times{} \num{20,17}}{\num{6}}
\end{align*}
perciò la \emph{media  ponderata} tra: \num{11} con molteplicità \num{3},  \num{22} con molteplicità
\num{1},  \num{33}  con  molteplicità  \num{2},   è  equivalente  alla  \emph{media  ponderata}  del
\emph{valore medio} \num{20,17} con molteplicità \num{6}.

%page
\subsection{Interpretazione della media ponderata come prezzo medio}


Nei calcoli di  contabilità del Conto Titoli: si  eseguono medie ponderate tra i  prezzi di acquisto
\Parentesi{\textbf{non}  di vendita}  di quote  di uno  stesso \Etf{}  rispetto al  numero di  quote
acquistate.

Negli esempi precedenti:  i numeri \num{11}, \num{22}, \num{33} si  possono interpretare come prezzi
di acquisto  di quote;  le molteplicità \num{3},  \num{1}, \num{2} si  possono interpretare  come il
numero di quote acquistate; cioè:
\begin{itemize}
\item \num{3} quote acquistate al prezzo di \Eur{11,00};
\item \num{1} quota acquistata al prezzo di \Eur{22,00};
\item \num{2} quote acquistate al prezzo di \Eur{33,00}.
\end{itemize}

Affermare che  la \emph{media ponderata}  dei prezzi è pari  a \Eur{20,17} significa  affermare che:
acquistare  \num{3} quote  di un  fondo al  prezzo di  \Eur{11,00}, poi  acquistare \num{1}  quota a
\Eur{22,00}, infine acquistare \num{2} quote a \Eur{33,00} è equivalente ad acquistare \num{6} quote
al \textbf{prezzo medio di acquisto} di \Eur{20,17}.

Il  numero  totale   di  quote  acquistate  \(\num{6}   =  \num{3}  +  \num{1}  +   \num{2}\)  è  la
\emph{molteplicità equivalente} del \emph{prezzo medio di acquisto} \Eur{20,17}, infatti si può scrivere:
\begin{align*}
  \frac{\num{3} \times{} \num{11} + \num{1} \times{} \num{22} + \num{2} \times{} \num{33}}{\num{6}}
  \simeq \frac{\num{6} \times{} \num{20,17}}{\num{6}}
  = \num{20,17}
\end{align*}

%page
\subsection{Calcolo incrementale della media ponderata}


L'attività di investimento in  un \Etf{} \Parentesi{cosí come in altri  titoli finanziari quotati} è
costituita da  una sequenza  di operazioni  di acquisto  e vendita  di quote  del fondo;  acquisti e
vendite possono  essere intercalati \Parentesi{prima un  acquisto, poi una vendita  parziale, poi un
   altro acquisto, eccetera}.  Dopo ogni acquisto: le nuove quote vengono messe insieme alle vecchie
già possedute.  Invece di conservare tutta la storia  della sequenza di acquisti e vendite: si tiene
la contabilità conservando solo il numero di quote attualmente possedute e il loro \emph{prezzo medio di acquisto}.

È utile richiamare  come il calcolo di  una \emph{media ponderata} possa essere  eseguito passo dopo
passo.

Si considerino  ancora i numeri  \num{11}, \num{11}, \num{11}, \num{22},  \num{33} e \num{33};  si è
calcolato che  la loro \emph{media  aritmetica} può essere  riscritta come \emph{media  ponderata} e
risulta:
\begin{equation*}
  \frac{\num{11} + \num{11} + \num{11} + \num{22} + \num{33} + \num{33}}{6}
  = \frac{\num{3} \times{} \num{11}
     + \num{1} \times{} \num{22}
     + \num{2} \times{} \num{33}}
  {6} \simeq \num{20,17}
\end{equation*}
Si  esegua lo  stesso calcolo  in piú  passi: si  considerino prima  i numeri  \num{11}, \num{22}  e
\num{33}; poi si aggiungano i numeri \num{11} e \num{11}; infine si aggiunga il numero \num{33}.
\begin{enumerate}
\item  La  \emph{media  ponderata}  tra  i  numeri  \num{11},  \num{22}  e  \num{33},  ciascuno  con
  molteplicità \num{1}, risulta:
  \begin{equation*}
    \frac{\num{1} \times{} \num{11}
       + \num{1} \times{} \num{22}
       + \num{1} \times{} \num{33}}{\num{3}}
    = \num{22}
  \end{equation*}
  il numero  totale di  numeri \(\num{3}  = \num{1} +  \num{1} +  \num{1}\) è  la \emph{molteplicità
     equivalente} del \emph{valore medio} \num{22}.

\item Ora si aggiungano  i numeri \num{11} e \num{11}; in totale:  \num{11} compare con molteplicità
  \num{3}; \num{22} compare con molteplicità \num{1}; \num{33} compare con molteplicità \num{1}.  La
  \emph{media ponderata aggiornata} risulta:
  \begin{equation*}
    \frac{\num{3} \times{} \num{11}
       + \num{1} \times{} \num{22}
       + \num{1} \times{} \num{33}}{\num{5}}
    = \frac{\num{88}}{\num{5}} \simeq \num{17,6}
  \end{equation*}
  il numero  totale di  numeri \(\num{5}  = \num{3} +  \num{1} +  \num{1}\) è  la \emph{molteplicità
     equivalente} del \emph{valore medio} \num{17,6}.

  È possibile calcolare  questo risultato partendo dalla media ponderata  del passo precedente, cioè
  \num{22}, combinandola con  in numeri \num{11} e  \num{11}?  La media ponderata  aggiornata si può
  scrivere:
  \begin{align*}
    \frac{\num{3} \times{} \num{11}
    + \num{1} \times{} \num{22}
    + \num{1} \times{} \num{33}}{\num{5}}
    = \frac{\num{2} \times{} \num{11}}{\num{5}}
    + \frac{\num{1} \times{} \num{11}
    + \num{1} \times{} \num{22}
    + \num{1} \times{} \num{33}}{\num{5}}
  \end{align*}
  la seconda frazione al membro di destra \Parentesi{il cui numeratore è uguale a quello della media
     ponderata al passo precedente} si può riscrivere:
  \begin{align*}
    \frac{\num{3}}{\num{5}} \times \frac{\num{5}}{\num{3}} \times
    \frac{\num{1} \times{} \num{11}
    + \num{1} \times{} \num{22}
    + \num{1} \times{} \num{33}}{\num{5}}
    &= \frac{\num{3}}{\num{5}} \times
      \frac{\num{1} \times{} \num{11}
      + \num{1} \times{} \num{22}
      + \num{1} \times{} \num{33}}{\num{3}}
  \end{align*}
  in cui il secondo termine della moltiplicazione al membro di destra è la media ponderata del passo
  precedente; quindi la \emph{media ponderata} aggiornata risulta:
  \begin{align*}
    \frac{\num{2} \times{} \num{11}}{\num{5}}
    + \frac{\num{3} \times{} \num{22}}{\num{5}}
    = \frac{\num{2} \times{} \num{11} + \num{3} \times{} \num{22}}{\num{5}}
    \simeq \num{17,6}
  \end{align*}
  cioè il \emph{valore medio} aggiornato \num{17,6} è pari alla \emph{media ponderata} tra:
  \begin{itemize}
  \item  il \emph{valore  medio} precedente  \num{22},  con la  sua \emph{molteplicità  equivalente}
    \num{3};
  \item il nuovo numero aggiunto \num{11}, con la sua molteplicità \num{2}.
  \end{itemize}

\item  Infine si  aggiunga il  numero  \num{33}; la  nuova  \emph{media ponderata}  aggiornata è  la
  \emph{media ponderata} totale e procedendo come prima risulta:
  \begin{equation*}
    \frac{\num{1} \times{} \num{33}
       + \num{5} \times{} \num{17,6}}{\num{6}}
    \simeq \num{20,17}
  \end{equation*}
\end{enumerate}

%page
\subsection{Intepretazione del calcolo incrementale della media ponderata}

Nei calcoli di contabilità del Conto Titoli: dopo ogni  acquisto di quote di un fondo, si calcola il
\emph{prezzo medio  di acquisto} aggiornato,  di ogni  quota posseduta, come  \emph{media ponderata}
tra:
\begin{itemize}
\item il  \emph{prezzo medio di  acquisto} precedente, con il  precedente numero di  quote possedute
  come molteplicità;
\item il prezzo delle nuove quote acquistate, con il numero di quote acquistate come molteplicità.
\end{itemize}

Come  si aggiorna  il  \emph{prezzo  medio di  acquisto}  se si  vendono  delle  quote?  Esso  resta
invariato!   Si considerino  ancora  i numeri  \num{11}, \num{11},  \num{11},  \num{22}, \num{33}  e
\num{33}; si è calcolato che la loro \emph{media ponderata} risulta:
\begin{equation*}
  \frac{\num{11} + \num{11} + \num{11} + \num{22} + \num{33} + \num{33}}{6}
  = \frac{\num{3} \times{} \num{11}
     + \num{1} \times{} \num{22}
     + \num{2} \times{} \num{33}}
  {6} \simeq \num{20,17}
\end{equation*}
in cui \num{20,17}  può essere interpretato come  \emph{prezzo medio di acquisto}  di \num{6} quote.
Si supponga  di togliere \num{2} quote  vendendole al prezzo  di \num{44}: il \emph{prezzo  medio di
   acquisto} delle restanti \num{4} quote è ancora \num{20,17}.

%page
\section{Descrizione di un'operazione di acquisto}

\input{conto-intesa--calcoli-esempio--operazione-acquisto.inc}


Quando  si  acquistano  quote di  un  titolo  \Parentesi{anche  non  \Etf{}} è  generalmente  meglio
specificare la strategia di acquisto \Virgolette{con  prezzo limite} e selezionare esplicitamente il
mercato su cui si  vogliono eseguire le operazioni, per esempio, nel  caso degli \Etf{}: \emph{Borsa
   Italiana, segmento ETFplus}.

Quando  l'operazione è  di  \textbf{acquisto}:  il \Virgolette{prezzo  limite}  è un  \textbf{prezzo
   massimo a cui si acquistano le quote}.  Perciò si inserisce l'ordine di acquisto per un numero di
quote del  fondo a un certo  prezzo massimo: per essere  conforme all'ordine inserito, la  banca che
agisce da intermediario può eseguire l'ordine  acquistando a qualsiasi prezzo \emph{minore o uguale}
al limite fissato.

L'ordine può essere  eseguito in piú fasi  in cui solo una  parte delle quote è  acquistata, in ogni
fase a  un prezzo diverso; è  possibile che non  tutte le quote  siano acquistate, nel qual  caso ci
interessa solo la  frazione \Virgolette{eseguita} dell'ordine.  Il \textbf{prezzo  medio eseguito di
   un acquisto} (nella  terminologia della Banca), anche detto \textbf{prezzo  medio effettivo di un
   acquisto} (nella  terminologia dell'Agenzia  dell'Entrate), è  la media  ponderata dei  prezzi di
acquisto rispetto al numero di quote acquistate.  Per  esempio, se un singolo ordine di acquisto per
\num{\unoEsAcqAcquistoNumeroQuote} quote è eseguito nelle tre fasi:
\begin{enumerate}
\item acquisto di \num{20} quote al prezzo di \Eur{52,00};
\item acquisto di \num{30} quote al prezzo di \Eur{53,00};
\item acquisto di \num{50} quote al prezzo di \Eur{55,00};
\end{enumerate}
allora il \emph{prezzo medio eseguito di acquisto} risulta:
\begin{equation*}
  \MediaPonderataTre{20}{52,00}{30}{53,00}{50}{55,00} = \Eur{\unoEsAcqAcquistoPrezzoMedioEffettivo{}}
\end{equation*}

Il \textbf{controvalore  dell'operazione di acquisto}  è il prodotto tra  il numero totale  di quote
acquistate e il  \emph{prezzo medio eseguito}.  Per esempio, se  si acquistano \num{\unoEsAcqAcquistoNumeroQuote}
quote   al   \emph{prezzo  medio   eseguito}   di   \Eur{\unoEsAcqAcquistoPrezzoMedioEffettivo}  il   controvalore
dell'operazione risulta:
\begin{equation*}
  \num{\unoEsAcqAcquistoNumeroQuote} \times{} \num{\unoEsAcqAcquistoPrezzoMedioEffettivo{}}
  = \Eur{\unoEsAcqAcquistoControvaloreOperazione}
\end{equation*}

Il \textbf{costo dell'operazione  di acquisto} è la  somma tra costi, spese  e commissioni associate
all'operazione; per ogni ordine di acquisto eseguito, occorre pagare:
\begin{itemize}
\item  le commissioni  per  l'intermediario, fissate  a  \Eur{2,50} quale  che  sia il  controvalore
  dell'operazione eseguita  \Parentesi{la Borsa  di negoziazione  su cui  si eseguono  le operazioni
     vuole essere pagata per ogni eseguito};
\item le commissioni fisse per la Banca pari a \Eur{0,50};
\item le  commissioni variabili per  la Banca  pari allo \SI{0,24}{\percent}  del \emph{controvalore
     dell'operazione}.
\end{itemize}
Se un ordine non è  eseguito: si paga nulla.  Se un ordine è eseguito in  piú fasi: nella prima fase
si pagano  i costi  fissi piú  la commissione  percentuale; nelle  fasi successive  si paga  solo la
commissione   percentuale.     Per   esempio,   se   il    \emph{controvalore   dell'operazione}   è
\Eur{\unoEsAcqAcquistoControvaloreOperazione} il \emph{costo dell'operazione} risulta:
\begin{equation*}
  \CalcoloCostoOperazione{\unoEsAcqAcquistoControvaloreOperazione}
  = \Eur{\unoEsAcqAcquistoCostoOperazione}
\end{equation*}

Il \textbf{controvalore totale di un acquisto}  è la somma tra \emph{controvalore dell'operazione} e
\emph{costo dell'operazione} ed è la quantità di  denaro addebitata sul Conto Corrente; per esempio,
se  il \emph{controvalore  dell'operazione} è  \Eur{\unoEsAcqAcquistoControvaloreOperazione{}} e  il
\emph{costo dell'operazione} è pari  a \Eur{\unoEsAcqAcquistoCostoOperazione}, il \emph{controvalore
   totale} risulta:
\begin{equation*}
  \num{\unoEsAcqAcquistoControvaloreOperazione} +
  \num{\unoEsAcqAcquistoCostoOperazione}
  = \Eur{\unoEsAcqAcquistoControvaloreTotale}
\end{equation*}

Il \textbf{prezzo medio di carico di un acquisto}  é il rapporto tra il \emph{controvalore totale} e
il numero di  quote acquistate ed è  usato per l'aggiornamento del  saldo: è il prezzo  medio di una
singola   quota    acquistata,   tutto    incluso.    Per   esempio,    se   si    sono   acquistate
\num{\unoEsAcqAcquistoNumeroQuote}      quote       al      \emph{controvalore       totale}      di
\Eur{\unoEsAcqAcquistoControvaloreTotale}, il \emph{prezzo medio di carico} risulta:
\begin{equation*}
  \num{\unoEsAcqAcquistoControvaloreTotale} / \num{\unoEsAcqAcquistoNumeroQuote}
  = \Eur{\unoEsAcqAcquistoPrezzoMedioCarico}
\end{equation*}

%page
\section{Aggiornamento del saldo dopo un acquisto}

\input{conto-intesa--calcoli-esempio--saldo-acquisto.inc}


Dopo l'esecuzione di ogni operazione di acquisto  occorre ricalcolare il saldo del Conto Titoli; per
un investimento in quote di \Etf{}, la parte da aggiornare è rappresentata dalle tre quantità:
\begin{itemize}
\item numero di quote di quell'\Etf{} in carico nel Conto Titoli;
\item  \textbf{prezzo  medio  \underline{effettivo}  nel  saldo}:  è  il  prezzo  medio,  per  quota
  acquistata, degli ordini eseguiti sulla Borsa di negoziazione; si calcola come media ponderata tra
  il \emph{prezzo  medio \underline{effettivo} di  un nuovo  acquisto} e il  precedente \emph{prezzo
     medio  \underline{effettivo} nel  saldo}; nella  documentazione  della Banca:  questo valore  è
  chiamato \emph{Net Asset Value}  (\textsc{nav}) del prezzo medio di carico  (da non confondere con
  il \textsc{nav} dell'attività sottostante il fondo);
\item \textbf{prezzo medio di \underline{carico} nel saldo}: è il prezzo medio per quota acquistata,
  tutto incluso; si calcola come media ponderata  tra il \emph{prezzo medio di \underline{carico} di
     un nuovo acquisto} e il precedente \emph{prezzo medio di \underline{carico} nel saldo}.
\end{itemize}

Dai valori nel saldo si può calcolare il  \textbf{costo medio di acquisto per ogni quota in carico}:
si  calcola  come differenza  tra  il  \emph{prezzo medio  di  \underline{carico}  nel saldo}  e  il
\emph{prezzo medio \underline{effettivo} nel saldo}.

Per esempio, si supponga di eseguire le operazioni:
\begin{enumerate}
\item acquisto di \num{\unoEsSaldoAcqAcquistoNumeroQuote} quote  al \emph{prezzo medio effettivo} di
  \Eur{\unoEsSaldoAcqAcquistoPrezzoMedioEffettivo};
\item acquisto di \num{\dueEsSaldoAcqAcquistoNumeroQuote} quote  al \emph{prezzo medio effettivo} di
  \Eur{\dueEsSaldoAcqAcquistoPrezzoMedioEffettivo};
\item acquisto di \num{\treEsSaldoAcqAcquistoNumeroQuote} quote  al \emph{prezzo medio effettivo} di
  \Eur{\treEsSaldoAcqAcquistoPrezzoMedioEffettivo};
\end{enumerate}
all'inizio  si consideri  un  Conto Titoli  vuoto,  con  saldo convenzionale  di:  \num{0} quote  di
quell'\Etf{}; \emph{prezzo  medio di carico} di  \Eur{0}; \emph{prezzo medio effettivo}  di \Eur{0}.
L'aggiornamento del saldo avviene come segue.
\begin{enumerate}
\item Per la prima operazione:
  \begin{itemize}
  \item numero di quote acquistate: \num{\unoEsSaldoAcqAcquistoNumeroQuote};
  \item \emph{prezzo medio effettivo}: \Eur{\unoEsSaldoAcqAcquistoPrezzoMedioEffettivo};
  \item \emph{controvalore dell'operazione}: \Eur{\unoEsSaldoAcqAcquistoControvaloreOperazione};
  \item \emph{costo dell'operazione}: \Eur{\unoEsSaldoAcqAcquistoCostoOperazione};
  \item \emph{controvalore totale}: \Eur{\unoEsSaldoAcqAcquistoControvaloreTotale};
  \item \emph{prezzo medio di carico}: \Eur{\unoEsSaldoAcqAcquistoPrezzoMedioCarico}.
  \end{itemize}

  Dopo la prima operazione:
  \begin{itemize}
  \item numero  quote nel  saldo: \num{\unoEsSaldoAcqSaldoNumeroQuote}; uguale  al numero  quote del
    primo acquisto;
  \item \emph{prezzo  medio di effettivo nel  saldo}: \Eur{\unoEsSaldoAcqSaldoPrezzoMedioEffettivo};
    uguale al \emph{prezzo medio effettivo} del primo acquisto;
  \item \emph{prezzo medio di  carico nel saldo}: \Eur{\unoEsSaldoAcqSaldoPrezzoMedioCarico}; uguale
    al \emph{prezzo medio di carico} del primo acquisto;
  \end{itemize}
  il \emph{costo medio per quota nel saldo} risulta:
  \begin{equation*}
    \num{\unoEsSaldoAcqSaldoPrezzoMedioCarico} - \num{\unoEsSaldoAcqSaldoPrezzoMedioEffettivo}
    = \Eur{\unoEsSaldoAcqSaldoCostoMedio}
  \end{equation*}

\item Per la seconda operazione:
  \begin{itemize}
  \item numero di quote acquistate: \num{\dueEsSaldoAcqAcquistoNumeroQuote};
  \item \emph{prezzo medio effettivo}: \Eur{\dueEsSaldoAcqAcquistoPrezzoMedioEffettivo};
  \item \emph{controvalore dell'operazione}: \Eur{\dueEsSaldoAcqAcquistoControvaloreOperazione};
  \item \emph{costo dell'operazione}: \Eur{\dueEsSaldoAcqAcquistoCostoOperazione};
  \item \emph{controvalore totale}: \Eur{\dueEsSaldoAcqAcquistoControvaloreTotale};
  \item \emph{prezzo medio di carico}: \Eur{\dueEsSaldoAcqAcquistoPrezzoMedioCarico}.
  \end{itemize}

  Dopo la seconda operazione:
  \begin{itemize}
  \item                  numero                   quote                  nel                  saldo:
    \(\num{\unoEsSaldoAcqSaldoNumeroQuote}      +     \num{\dueEsSaldoAcqAcquistoNumeroQuote}      =
    \num{\dueEsSaldoAcqSaldoNumeroQuote}\);
  \item  il \emph{prezzo  medio effettivo  nel saldo}  è la  media ponderata  dei \emph{prezzi  medi
       effettivi} del saldo precedente e del nuovo acquisto:
    \begin{equation*}
      \MediaPonderataDue
      {\unoEsSaldoAcqSaldoNumeroQuote}{\unoEsSaldoAcqSaldoPrezzoMedioEffettivo}
      {\dueEsSaldoAcqAcquistoNumeroQuote}{\dueEsSaldoAcqAcquistoPrezzoMedioEffettivo}
      = \Eur{\dueEsSaldoAcqSaldoPrezzoMedioEffettivo}
    \end{equation*}
  \item il  \emph{prezzo medio di carico  nel saldo} è la  media ponderata dei \emph{prezzi  medi di
       carico} del saldo precedente e del nuovo acquisto:
    \begin{equation*}
      \MediaPonderataDue
      {\unoEsSaldoAcqSaldoNumeroQuote}{\unoEsSaldoAcqSaldoPrezzoMedioCarico}
      {\dueEsSaldoAcqAcquistoNumeroQuote}{\dueEsSaldoAcqAcquistoPrezzoMedioCarico}
      = \Eur{\dueEsSaldoAcqSaldoPrezzoMedioCarico}
    \end{equation*}
  \end{itemize}
  il \emph{costo medio per quota nel saldo} risulta:
  \begin{equation*}
    \num{\dueEsSaldoAcqSaldoPrezzoMedioCarico} - \num{\dueEsSaldoAcqSaldoPrezzoMedioEffettivo}
    = \Eur{\dueEsSaldoAcqSaldoCostoMedio}
  \end{equation*}

\item Per la terza operazione:
  \begin{itemize}
  \item numero di quote acquistate: \num{\treEsSaldoAcqAcquistoNumeroQuote};
  \item \emph{prezzo medio effettivo}: \Eur{\treEsSaldoAcqAcquistoPrezzoMedioEffettivo};
  \item \emph{controvalore dell'operazione}: \Eur{\treEsSaldoAcqAcquistoControvaloreOperazione};
  \item \emph{costo dell'operazione}: \Eur{\treEsSaldoAcqAcquistoCostoOperazione};
  \item \emph{controvalore totale}: \Eur{\treEsSaldoAcqAcquistoControvaloreTotale};
  \item \emph{prezzo medio di carico}: \Eur{\treEsSaldoAcqAcquistoPrezzoMedioCarico}.
  \end{itemize}

  Dopo la terza operazione:
  \begin{itemize}
  \item                  numero                   quote                  nel                  saldo:
    \(\num{\dueEsSaldoAcqSaldoNumeroQuote}      +     \num{\treEsSaldoAcqAcquistoNumeroQuote}      =
    \num{\treEsSaldoAcqSaldoNumeroQuote}\);
  \item  il \emph{prezzo  medio effettivo  nel saldo}  è la  media ponderata  dei \emph{prezzi  medi
       effettivi} del saldo precedente e del nuovo acquisto:
    \begin{equation*}
      \MediaPonderataDue
      {\dueEsSaldoAcqSaldoNumeroQuote}{\dueEsSaldoAcqSaldoPrezzoMedioEffettivo}
      {\treEsSaldoAcqAcquistoNumeroQuote}{\treEsSaldoAcqAcquistoPrezzoMedioEffettivo}
      = \Eur{\treEsSaldoAcqSaldoPrezzoMedioEffettivo}
    \end{equation*}
  \item il  \emph{prezzo medio di carico  nel saldo} è la  media ponderata dei \emph{prezzi  medi di
       carico} del saldo precedente e del nuovo acquisto:
    \begin{equation*}
      \MediaPonderataDue
      {\dueEsSaldoAcqSaldoNumeroQuote}{\dueEsSaldoAcqSaldoPrezzoMedioCarico}
      {\treEsSaldoAcqAcquistoNumeroQuote}{\treEsSaldoAcqAcquistoPrezzoMedioCarico}
      = \Eur{\treEsSaldoAcqSaldoPrezzoMedioCarico}
    \end{equation*}
  \end{itemize}
  il \emph{costo medio per quota nel saldo} risulta:
  \begin{equation*}
    \num{\treEsSaldoAcqSaldoPrezzoMedioCarico} - \num{\treEsSaldoAcqSaldoPrezzoMedioEffettivo}
    = \Eur{\treEsSaldoAcqSaldoCostoMedio}
  \end{equation*}
\end{enumerate}

Evidenziando  i prezzi  medi: ogni  nuova quota  acquistata viene  \Virgolette{buttata nel  secchio}
insieme alle vecchie, indipendentemente  dal suo prezzo di acquisto; è come se  ogni quota in carico
fosse stata acquistata allo stesso prezzo, pari al prezzo medio.

Si osservi come, \textbf{solo quando si sono eseguiti degli acquisti ma nessuna vendita}:
\begin{itemize}
\item il \emph{prezzo medio effettivo nel saldo} si possa calcolare anche come media ponderata tra i
  \emph{prezzi medi effettivi degli acquisti}:
  \begin{equation*}
    \MediaPonderataTre
    {\unoEsSaldoAcqAcquistoNumeroQuote}{\unoEsSaldoAcqAcquistoPrezzoMedioEffettivo}
    {\dueEsSaldoAcqAcquistoNumeroQuote}{\dueEsSaldoAcqAcquistoPrezzoMedioEffettivo}
    {\treEsSaldoAcqAcquistoNumeroQuote}{\treEsSaldoAcqAcquistoPrezzoMedioEffettivo}
    = \Eur{\treEsSaldoAcqSaldoPrezzoMedioEffettivo}
  \end{equation*}
\item il \emph{prezzo medio di carico nel saldo} si possa calcolare anche come media ponderata tra i
  \emph{prezzi medi di carico degli acquisti}:
  \begin{equation*}
    \MediaPonderataTre
    {\unoEsSaldoAcqAcquistoNumeroQuote}{\unoEsSaldoAcqAcquistoPrezzoMedioCarico}
    {\dueEsSaldoAcqAcquistoNumeroQuote}{\dueEsSaldoAcqAcquistoPrezzoMedioCarico}
    {\treEsSaldoAcqAcquistoNumeroQuote}{\treEsSaldoAcqAcquistoPrezzoMedioCarico}
    = \Eur{\treEsSaldoAcqSaldoPrezzoMedioCarico}
  \end{equation*}
\item il costo totale di tutte le operazioni si possa calcolare sia come somma tra tutti i costi:
  \begin{equation*}
    \num{\unoEsSaldoAcqAcquistoCostoOperazione} +
    \num{\dueEsSaldoAcqAcquistoCostoOperazione} +
    \num{\treEsSaldoAcqAcquistoCostoOperazione}
    = \Eur{\treEsSaldoAcqSaldoCostoTotaleQuote}
  \end{equation*}
  che come prodotto tra il numero di quote nel saldo e il \emph{costo medio di acquisto per quota in
     carico}:
  \begin{equation*}
    \num{\treEsSaldoAcqSaldoNumeroQuote} \times{}
    \num{\treEsSaldoAcqSaldoCostoMedio}
    = \Eur{\treEsSaldoAcqSaldoCostoTotaleQuote}
  \end{equation*}
\end{itemize}
dopo la prima operazione di vendita: queste proprietà non valgono piú!

%page
\section{Descrizione di un'operazione di vendita}

\input{conto-intesa--calcoli-esempio--operazione-vendita.inc}

\Define{\UnoNumeroQuote}{20}
\Define{\DueNumeroQuote}{30}
\Define{\TreNumeroQuote}{50}
\Define{\UnoPrezzoEseguito}{52,00}
\Define{\DuePrezzoEseguito}{53,00}
\Define{\TrePrezzoEseguito}{55,00}


Quando si vendono quote di un titolo  \Parentesi{anche non \Etf{}} è generalmente meglio specificare
la strategia di  acquisto \Virgolette{con prezzo limite} e selezionare  esplicitamente il mercato su
cui si vogliono  eseguire le operazioni, per  esempio, nel caso degli  \Etf{}: \emph{Borsa Italiana,
   segmento ETFplus}.

Quando l'operazione è di \textbf{vendita}: il  \Virgolette{prezzo limite} è un \textbf{prezzo minimo
   a cui si vendono le  quote}.  Perciò si inserisce l'ordine di vendita per  un numero di quote del
fondo a  un certo prezzo  minimo: per essere  conforme all'ordine inserito,  la banca che  agisce da
intermediario può eseguire  l'ordine vendendo a qualsiasi prezzo \emph{maggiore  o uguale} al limite
fissato.

L'ordine può essere eseguito in piú fasi in cui solo una parte delle quote è venduta, in ogni fase a
un prezzo diverso; è possibile che non tutte le quote siano vendute, nel qual caso ci interessa solo
la frazione \Virgolette{eseguita} dell'ordine.  Il  \textbf{\emph{prezzo medio eseguito di vendita}}
(nella  terminologia della  Banca), anche  detto \textbf{\emph{prezzo  medio effettivo  di vendita}}
(nella terminologia dell'Agenzia dell'Entrate), è la  media ponderata dei prezzi di vendita rispetto
al   numero   di  quote   vendute.    Per   esempio,  se   un   singolo   ordine  di   vendita   per
\num{\dueEsVenVenditaNumeroQuote{}} quote è eseguito nelle fasi:
\begin{enumerate}
\item vendita di \num{\UnoNumeroQuote} quote al prezzo di \Eur{\UnoPrezzoEseguito};
\item vendita di \num{\DueNumeroQuote} quote al prezzo di \Eur{\DuePrezzoEseguito};
\item vendita di \num{\TreNumeroQuote} quote al prezzo di \Eur{\TrePrezzoEseguito};
\end{enumerate}
allora il \emph{prezzo medio effettivo} risulta:
\begin{equation*}
  \MediaPonderataTre
  {\UnoNumeroQuote}{\UnoPrezzoEseguito}
  {\DueNumeroQuote}{\DuePrezzoEseguito}
  {\TreNumeroQuote}{\TrePrezzoEseguito}
  = \Eur{\dueEsVenVenditaPrezzoMedioEffettivo}
\end{equation*}

Il \textbf{controvalore  dell'operazione di  vendita} è il  prodotto tra il  numero totale  di quote
vendute    e    il    \emph{prezzo    medio    effettivo};    per    esempio,    se    si    vendono
\num{\dueEsVenVenditaNumeroQuote{}}     quote     al     \emph{prezzo    medio     effettivo}     di
\Eur{\dueEsVenVenditaPrezzoMedioEffettivo{}} il controvalore dell'operazione risulta:
\begin{equation*}
  \num{\dueEsVenVenditaNumeroQuote{}} \times{}
  \num{\dueEsVenVenditaPrezzoMedioEffettivo{}}
  = \Eur{\dueEsVenVenditaControvaloreOperazione{}}
\end{equation*}

La  circolare  dell'Agenzia dell'Entrate  specifica  che  il  \emph{reddito finanziario  per  quota}
derivante da un'operazione di vendita è la differenza tra \emph{prezzo medio effettivo di vendita} e
\emph{prezzo medio effettivo nel saldo} precedente la vendita stessa; peculiarmente:
\begin{itemize}
\item quando  la differenza tra  prezzi medi effettivi è  \textbf{positiva}: essa è  da considerarsi
  \emph{reddito da capitale}, quindi \textbf{non genera} una plusvalenza utilizzabile per compensare
  precedenti minusvalenze da altri investimenti;
\item quando la differenza tra prezzi medi  effettivi è \textbf{negativa}: essa è da considerarsi un
  \emph{reddito  diverso},  quindi  \textbf{genera}  una minusvalenza  compensabile  con  successive
  plusvalenze da altri investimenti.
\end{itemize}

Quando  il reddito  di  una vendita  è  positivo: occorre  pagare la  \textbf{tassa  sul reddito  da
   capitale}\footnote{Si deve  pagare anche il  bollo sull'estratto  conto trimestrale per  il Conto
   Titoli,  pari allo  \SI{0,2}{\percent} all'anno,  cioè  lo \SI{0,05}{\percent}  al trimestre  del
   controvalore alla data di chiusura dei conti  trimestrali; il bollo è addebitato direttamente sul
   Conto Corrente, perciò in questa guida lo si considera conteggiato a parte.}.  Per esempio, se si
vendono   \num{\dueEsVenVenditaNumeroQuote{}}   quote   al    \emph{prezzo   medio   effettivo}   di
\Eur{\dueEsVenVenditaPrezzoMedioEffettivo} e il \emph{prezzo medio effettivo nel saldo} precedente è
di \Eur{\unoEsVenSaldoPrezzoMedioEffettivo}, il \emph{reddito da capitale} risulta:
\begin{equation*}
  \num{\dueEsVenVenditaNumeroQuote} \times{} \left(
    \num{\dueEsVenVenditaPrezzoMedioEffettivo} - \num{\unoEsVenSaldoPrezzoMedioEffettivo}
  \right) = \Eur{\dueEsVenVenditaRedditoCapitale}
\end{equation*}
per fondi  che non  contengono Titoli  dello Stato Italiano,  e assimilati,  l'aliquota unica  è del
\SI{26}{\percent}; allora la \emph{tassa sul reddito da capitale} risulta:
\begin{equation*}
  \num{0,26} \times{} \num{\dueEsVenVenditaRedditoCapitale}
  = \Eur{\dueEsVenVenditaTassaRedditoCapitale}
\end{equation*}

Il \textbf{costo  dell'operazione di vendita}  è la somma tra  costi, spese e  commissioni associate
all'operazione; per ogni ordine di vendita eseguito, occorre pagare:
\begin{itemize}
\item  le commissioni  per  l'intermediario, fissate  a  \Eur{2,50} quale  che  sia il  controvalore
  dell'operazione eseguita  \Parentesi{la Borsa  di negoziazione  su cui  si eseguono  le operazioni
     vuole essere pagata per ogni eseguito};
\item le commissioni fisse per la Banca pari a \Eur{0,50};
\item le  commissioni variabili per  la Banca  pari allo \SI{0,24}{\percent}  del \emph{controvalore
     dell'operazione}.
\end{itemize}
se un ordine non è eseguito: si paga nulla; se un ordine è eseguito in piú fasi: nella prima fase si
pagano  i costi  fissi  piú  la commissione  percentuale;  nelle fasi  successive  si  paga solo  la
commissione   percentuale.     Per   esempio,   se   il    \emph{controvalore   dell'operazione}   è
\Eur{\dueEsVenVenditaControvaloreOperazione} il \emph{costo dell'operazione} risulta:
\begin{equation*}
  \CalcoloCostoOperazione{\dueEsVenVenditaControvaloreOperazione}
  = \Eur{\dueEsVenVenditaCostoOperazione}
\end{equation*}

Da osservare  che: sia  il calcolo  della \emph{tassa  sul reddito  da capitale}  che i  calcoli del
\emph{costo  dell'operazione}  di acquisto  e  vendita  si  eseguono  usando il  \emph{prezzo  medio
   effettivo}; in pratica: si  pagano le tasse anche sul \emph{costo  dell'operazione} di acquisto e
vendita, che però sono redditi di qualcun altro, non dell'investitore!

La  circolare  dell'Agenzia  dell'Entrate  specifica  che   il  costo  delle  operazioni  genera  un
\emph{reddito diverso}, da  registrare come minusvalenza.  Per esempio, si  supponga di possedere un
Conto Titoli con:
\begin{itemize}
\item numero di quote in carico: \num{\dueEsVenVenditaNumeroQuote};
\item \emph{prezzo medio effettivo}: \Eur{\unoEsVenSaldoPrezzoMedioEffettivo{}};
\item \emph{prezzo medio di carico}: \Eur{\unoEsVenSaldoPrezzoMedioCarico{}};
\item \emph{costo medio di acquisto per ogni quota in carico}: \Eur{\unoEsVenSaldoCostoMedio{}};
\end{itemize}
si vendano tutte le quote; l'operazione di vendita sia descritta da:
\begin{itemize}
\item numero di quote vendute: \num{\dueEsVenVenditaNumeroQuote};
\item \emph{prezzo medio effettivo di vendita}: \Eur{\dueEsVenVenditaControvaloreOperazione{}};
\item \emph{reddito da capitale}: \Eur{\dueEsVenVenditaRedditoCapitale{}};
\item \emph{costo dell'operazione di vendita}: \Eur{\dueEsVenVenditaCostoOperazione{}};
\end{itemize}
ogni   quota   venduta   era  stata   acquistata   con   un   \emph{costo   medio  per   quota}   di
\Eur{\unoEsVenSaldoCostoMedio}, perciò il \emph{costo di acquisto delle quote vendute} risulta:
\begin{equation*}
  \num{\dueEsVenVenditaNumeroQuote} \times{} \num{\unoEsVenSaldoCostoMedio}
  = \Eur{\dueEsVenVenditaCostoAcquistoQuoteVendute}
\end{equation*}
allo scopo di far risultare un numero negativo, la circolare specifica che il \emph{reddito diverso}
associato alla vendita deve essere calcolato come:
\begin{equation*}
  \left[
    \num{\dueEsVenVenditaRedditoCapitale} - \left(
      \num{\dueEsVenVenditaCostoOperazione} + \num{\dueEsVenVenditaCostoAcquistoQuoteVendute}
    \right)
  \right] - \num{\dueEsVenVenditaRedditoCapitale}
  = - \left(
    \num{\dueEsVenVenditaCostoOperazione} + \num{\dueEsVenVenditaCostoAcquistoQuoteVendute}
  \right)
  = \Eur{\dueEsVenVenditaRedditoDiversoTotale}
\end{equation*}
questa  minusvalenza  viene  automaticamente  registrata  dalla  Banca  nella  \Virgolette{Posizione
   Fiscale} del Conto Titoli; da notare che  \textbf{le minusvalenze derivanti dai costi di acquisto
   e vendita  sono registrate nella Posizione  Fiscale solo al  momento della vendita di  quote}, al
momento dell'acquisto nulla viene registrato.

Il   \textbf{controvalore  totale   di  vendita}   è   la  differenza   tra  il   \emph{controvalore
   dell'operazione}  e la  somma  tra  \emph{costo dell'operazione}  e  \emph{tassa  sul reddito  da
   capitale};  è  la  quantità di  denaro  accreditata  sul  Conto  Corrente.  Per  esempio,  se  il
\emph{controvalore dell'operazione}  è \Eur{\dueEsVenVenditaControvaloreOperazione},  il \emph{costo
   dell'operazione}  è  \Eur{\dueEsVenVenditaCostoOperazione}  e   la  \emph{tassa  sul  reddito  da
   capitale} è \Eur{\dueEsVenVenditaTassaRedditoCapitale}, il \emph{controvalore totale} risulta:
\begin{equation*}
  \num{\dueEsVenVenditaControvaloreOperazione} -
  \left( \num{\dueEsVenVenditaCostoOperazione} + \num{\dueEsVenVenditaTassaRedditoCapitale} \right)
  = \Eur{\dueEsVenVenditaControvaloreTotale}
\end{equation*}

In un'operazione  di vendita le  quote vengono \Virgolette{scaricate}  dal Conto Titoli:  la vendita
modifica solo  il numero di quote  in carico nel saldo,  il \emph{prezzo medio di  carico nel saldo}
resta invariato da prima a dopo la  vendita.  Allo scopo di calcolare il rendimento dell'operazione,
si definisce il  \textbf{prezzo medio netto di  vendita} pari al rapporto  tra il \emph{controvalore
   totale}   e    il   numero   di   quote    vendute.    Per   esempio,   se    si   sono   vendute
\num{\dueEsVenVenditaNumeroQuote}       quote      al       \emph{controvalore      totale}       di
\Eur{\dueEsVenVenditaControvaloreTotale}, il \emph{prezzo medio netto di vendita} risulta:
\begin{equation*}
  \num{\dueEsVenVenditaControvaloreTotale} / \num{\dueEsVenVenditaNumeroQuote}
  = \Eur{\dueEsVenVenditaPrezzoMedioNettoVendita}
\end{equation*}

Il \textbf{rendimento  \underline{percentuale} dell'operazione di  vendita}, rispetto alle  quote in
carico nel  saldo precedente,  si calcola  considerando il  \emph{prezzo medio  di carico  nel saldo
   precedente} e il \emph{prezzo medio netto di vendita};  in caso di guadagno è un numero positivo,
in caso di perdita è un numero negativo.  Per  esempio, se il \emph{prezzo medio di carico nel saldo
   precedente} è \Eur{\unoEsVenSaldoPrezzoMedioCarico}  e il \emph{prezzo medio netto  di vendita} è
\Eur{\dueEsVenVenditaPrezzoMedioNettoVendita}, il \emph{rendimento percentuale} risulta:
\begin{equation*}
  \CalcoloRendimentoPercentuale{\dueEsVenVenditaPrezzoMedioNettoVendita}{\unoEsVenSaldoPrezzoMedioCarico}
  = \Perc{\dueEsVenVenditaRendimentoPercentuale}
\end{equation*}

Il  \textbf{rendimento \underline{in  valuta} dell'operazione  di vendita},  rispetto alle  quote in
carico nel  saldo precedente,  si calcola  considerando il  \emph{prezzo medio  di carico  nel saldo
   precedente} e il \emph{prezzo medio netto di vendita};  in caso di guadagno è un numero positivo,
in    caso   di    perdita    è   un    numero    negativo.    Per    esempio,    se   si    vendono
\num{\dueEsVenVenditaNumeroQuote} quote con \emph{prezzo medio  di carico nel saldo precedente} pari
a   \Eur{\unoEsVenSaldoPrezzoMedioCarico}  e   \emph{prezzo   medio  netto   di   vendita}  pari   a
\Eur{\dueEsVenVenditaPrezzoMedioNettoVendita}, il \emph{rendimento in valuta} risulta:
\begin{equation*}
  \num{\dueEsVenVenditaNumeroQuote} \times{} \left(
    \num{\dueEsVenVenditaPrezzoMedioNettoVendita} - \num{\unoEsVenSaldoPrezzoMedioCarico}
  \right) = \Eur{\dueEsVenVenditaRendimentoInValuta}
\end{equation*}

%page
\section{Aggiornamento del saldo dopo una vendita}


Dopo  l'esecuzione di  ogni operazione  di vendita  occorre ricalcolare  il saldo.   Il saldo  di un
investimento in quote di \Etf{} è rappresentato dalle tre quantità:
\begin{itemize}
\item numero di quote di quell'\Etf{} in carico nel Conto Titoli;
\item \emph{prezzo medio effettivo nel saldo} per ogni quota in carico;
\item \emph{prezzo medio di carico nel saldo} per ogni quota in carico;
\end{itemize}
sia  il \emph{prezzo  medio effettivo  nel saldo}  che il  \emph{prezzo medio  di carico  nel saldo}
restano  \textbf{immutati} dopo  una  vendita: essa  modifica  solo  il numero  di  quote in  carico
diminuendolo del numero di quote vendute.

Per esempio, si supponga di avere un Conto Titoli con saldo iniziale:
\begin{itemize}
\item numero di quote: \num{100};
\item \emph{prezzo medio effettivo nel saldo}: \Eur{50,00};
\item \emph{prezzo medio di carico nel saldo}: \Eur{50,15};
\end{itemize}
se si vendessero \num{90} quote, non importa a quale prezzo, il saldo finale risulterebbe:
\begin{itemize}
\item numero di quote: \(\num{100} - \num{90} = \num{10}\);
\item \emph{prezzo medio effettivo nel saldo}: \Eur{50,00}, immutato;
\item \emph{prezzo medio di carico nel saldo}: \Eur{50,15}, immutato.
\end{itemize}

%page
\section{Rendimento indicato nel riepilogo del patrimonio}


Nel sito di  banca telematica è indicato, per  ogni fondo nel Conto Titoli, il  saldo costituito dal
numero  di quote  in  carico e  dal  \emph{prezzo  medio di  carico};  in piú  è  visibile la  stima
dell'\emph{Utile  o Perdita},  nell'ipotesi  di vendita  di  tutte le  quote  al prezzo  dell'ultimo
contratto di compravendita sulla Borsa di negoziazione del titolo:
\begin{itemize}
\item l'\emph{Utile} mostrato è  al lordo sia del \emph{costo dell'operazione  di vendita} che della
  \emph{tassa sul reddito da capitale};
\item la \emph{Perdita} mostrata è al lordo del \emph{costo dell'operazione} di vendita.
\end{itemize}

Si supponga di possedere \num{100} quote al \emph{prezzo medio di carico nel saldo} di \Eur{50,00}:
\begin{itemize}
\item  se l'ultimo  contratto  di  compravendita è  stato  chiuso  al prezzo  di  una  quota pari  a
  \Eur{52,00}, l'\emph{Utile percentuale} mostrato nella tabella del patrimonio risulterebbe:
  \begin{equation*}
    \CalcoloRendimentoPercentuale{52,00}{50,00} = \SI{+3,6889}{\percent}
  \end{equation*}
  mentre l'\emph{Utile in valuta} risulterebbe:
  \begin{equation*}
    \num{100} \times{} (\num{52,00} - \num{50,00}) = \Eur{+185,00}
  \end{equation*}
  in realtà, vendendo tutte le quote al \emph{prezzo medio effettivo} di \Eur{52,00}, risulterebbe:
  \begin{itemize}
  \item \emph{prezzo medio netto}: \Eur{51,33};
  \item \emph{rendimento percentuale}: \SI{+2,3434}{\percent};
  \item \emph{rendimento in valuta}: \Eur{+117,52};
  \end{itemize}

\item  se l'ultimo  contratto  di  compravendita è  stato  chiuso  al prezzo  di  una  quota pari  a
  \Eur{48,00}, la \emph{Perdita percentuale} mostrata nella tabella del patrimonio risulterebbe:
  \begin{equation*}
    \CalcoloRendimentoPercentuale{48,00}{50,00} = \SI{-4,2871}{\percent}
  \end{equation*}
  mentre la \emph{Perdita in valuta} risulterebbe:
  \begin{equation*}
    \num{100} \times{} (\num{48,00} - \num{50,00}) = \Eur{-215,00}
  \end{equation*}
  in realtà, vendendo tutte le quote al \emph{prezzo medio effettivo} di \Eur{48,00}, risulterebbe:
  \begin{itemize}
  \item \emph{prezzo medio netto}: \Eur{47,85};
  \item \emph{perdita percentuale}: \SI{-4,5767}{\percent};
  \item \emph{perdita in valuta}: \Eur{-229,52}.
  \end{itemize}
\end{itemize}

%page
\section{Strategie per la scelta dei prezzi di vendita}


Si cerca  di vendere a  un prezzo maggiore  del prezzo di acquisto;  se occorresse liquidità  per le
proprie spese: si potrebbe essere costretti a disinvestire, vendendo a prezzi inferiori.

%page
\subsection{Metodo delle medie ponderate}

\input{conto-intesa--calcoli-esempio--strategie-prezzi--metodo-media-ponderata.inc}


Si considerano tutte le quote di uno stesso  \Etf{}, acquistate a qualsiasi prezzo, come parte dello
stesso insieme.  Si cerca  di vendere le quote a un prezzo talmente  maggiore del prezzo di acquisto
che, recuperati i costi e pagate le tasse, si realizza un guadagno.
\begin{itemize}
\item Se  il \emph{prezzo medio netto  di vendita} è maggiore  del \emph{prezzo medio di  carico nel
     saldo} precedente  la vendita: si recuperano  i costi di  acquisto e vendita; si  recuperano le
  tasse; si chiude con un guadagno.

\item Se il \emph{prezzo medio netto di vendita} è uguale al \emph{prezzo medio di carico nel saldo}
  precedente la  vendita: si recuperano i  costi di acquisto e  vendita; si recuperano le  tasse; ma
  resta nessun guadagno, si chiude in pareggio.

\item Se  il \emph{prezzo  medio netto di  vendita} è  minore del \emph{prezzo  medio di  carico nel
     saldo} precedente la  vendita: tolti i costi  di acquisto e vendita ed  eventualmente le tasse,
  l'operazione genera una perdita.
\end{itemize}
In ogni  caso, restano le  \emph{minusvalenze} accumulate nella Posizione  Fiscale che, se  e quando
possibile, potranno essere  usate per ridurre le  tasse da pagare in future  operazioni che generano
\emph{redditi diversi}.

Si supponga di avere un Conto Titoli con il seguente saldo:
\begin{itemize}
\item numero quote: \num{\unoStratMediaEsUnoSaldoNumeroQuote};
\item \emph{prezzo medio effettivo}: \Eur{\unoStratMediaEsUnoSaldoPrezzoMedioEffettivo};
\item \emph{prezzo medio di carico}: \Eur{\unoStratMediaEsUnoSaldoPrezzoMedioCarico};
\item \emph{costo medio di acquisto per quota in carico}: \Eur{\unoStratMediaEsUnoSaldoCostoMedio};
\item \emph{controvalore di carico}: \Eur{\unoStratMediaEsUnoSaldoControvaloreCarico};
\item \emph{minusvalenze accumulate}: \Eur{\unoStratMediaEsUnoSaldoMinusvalenza};
\end{itemize}
si vogliano vendere tutte le quote.

%% ------------------------------------------------------------------------

Come esempio di guadagno, si venda come segue:
\begin{itemize}
\item numero quote: \num{\dueStratMediaEsUnoVenditaNumeroQuote};
\item \emph{prezzo medio eseguito}: \Eur{\dueStratMediaEsUnoVenditaPrezzoMedioEffettivo};
\item \emph{controvalore dell'operazione}: \Eur{\dueStratMediaEsUnoVenditaControvaloreOperazione};

\item \emph{reddito da capitale}: \Eur{\dueStratMediaEsUnoVenditaRedditoCapitale};
\item \emph{tasse sul reddito da capitale}: \Eur{\dueStratMediaEsUnoVenditaTassaRedditoCapitale};

\item \emph{costo dell'operazione di vendita}: \Eur{\dueStratMediaEsUnoVenditaCostoOperazione};
\item \emph{costo medio di acquisto delle quote vendute}: \Eur{\dueStratMediaEsUnoVenditaCostoAcquistoQuoteVendute};
\item \emph{reddito diverso da minusvalenza sul capitale}: \Eur{\dueStratMediaEsUnoVenditaRedditoDiversoMinusvalenzaCapitale};
\item \emph{reddito diverso da pagamento delle commissioni}: \Eur{\dueStratMediaEsUnoVenditaRedditoDiversoPagamentoCommissioni};
\item \emph{reddito diverso totale}: \Eur{\dueStratMediaEsUnoVenditaRedditoDiversoTotale};

\item \emph{controvalore totale}: \Eur{\dueStratMediaEsUnoVenditaControvaloreTotale};

\item \emph{prezzo medio netto di vendita}: \Eur{\dueStratMediaEsUnoVenditaPrezzoMedioNettoVendita};
\item \emph{rendimento percentuale}: \SI{\dueStratMediaEsUnoVenditaRendimentoPercentuale}{\percent};
\item \emph{rendimento in valuta}: \Eur{\dueStratMediaEsUnoVenditaRendimentoInValuta};
\end{itemize}
il         \emph{prezzo        medio         netto         di         vendita}        pari         a
\Eur{\dueStratMediaEsUnoVenditaPrezzoMedioNettoVendita} è maggiore del  \emph{prezzo medio di carico
   nel  saldo} pari  a \Eur{\unoStratMediaEsUnoSaldoPrezzoMedioCarico},  quindi, recuperati  costi e
tasse, si  realizza un  guadagno; il  reddito diverso è  una minusvalenza  da compensare  con future
operazioni.

%% ------------------------------------------------------------------------

Come esempio di perdita, si venda come segue:
\begin{itemize}
\item numero quote: \num{\dueStratMediaEsDueVenditaNumeroQuote};
\item \emph{prezzo medio eseguito}: \Eur{\dueStratMediaEsDueVenditaPrezzoMedioEffettivo};
\item \emph{controvalore dell'operazione}: \Eur{\dueStratMediaEsDueVenditaControvaloreOperazione};

\item \emph{reddito da capitale}: \Eur{\dueStratMediaEsDueVenditaRedditoCapitale};
\item \emph{tasse sul reddito da capitale}: \Eur{\dueStratMediaEsDueVenditaTassaRedditoCapitale};

\item \emph{costo dell'operazione di vendita}: \Eur{\dueStratMediaEsDueVenditaCostoOperazione};
\item \emph{costo medio di acquisto delle quote vendute}: \Eur{\dueStratMediaEsDueVenditaCostoAcquistoQuoteVendute};
\item \emph{reddito diverso da minusvalenza sul capitale}: \Eur{\dueStratMediaEsDueVenditaRedditoDiversoMinusvalenzaCapitale};
\item \emph{reddito diverso da pagamento delle commissioni}: \Eur{\dueStratMediaEsDueVenditaRedditoDiversoPagamentoCommissioni};
\item \emph{reddito diverso totale}: \Eur{\dueStratMediaEsDueVenditaRedditoDiversoTotale};

\item \emph{controvalore totale}: \Eur{\dueStratMediaEsDueVenditaControvaloreTotale};

\item \emph{prezzo medio netto di vendita}: \Eur{\dueStratMediaEsDueVenditaPrezzoMedioNettoVendita};
\item \emph{rendimento percentuale}: \SI{\dueStratMediaEsDueVenditaRendimentoPercentuale}{\percent};
\item \emph{rendimento in valuta}: \Eur{\dueStratMediaEsDueVenditaRendimentoInValuta};
\end{itemize}
nonostante          il         \emph{prezzo          medio         eseguito}          pari         a
\Eur{\dueStratMediaEsDueVenditaPrezzoMedioEffettivo} sia  maggiore del \emph{prezzo medio  di carico
   nel saldo} pari a \Eur{\unoStratMediaEsDueSaldoPrezzoMedioCarico}, il \emph{prezzo medio netto di
   vendita} di \Eur{\dueStratMediaEsDueVenditaPrezzoMedioNettoVendita} è minore, quindi, tolti costi
e tasse,  si realizza una  perdita; il  reddito diverso è  una minusvalenza compensabile  con future
operazioni.

%% ------------------------------------------------------------------------

Come ulteriore esempio di perdita, si venda come segue:
\begin{itemize}
\item numero quote: \num{\dueStratMediaEsTreVenditaNumeroQuote};
\item \emph{prezzo medio eseguito}: \Eur{\dueStratMediaEsTreVenditaPrezzoMedioEffettivo};
\item \emph{controvalore dell'operazione}: \Eur{\dueStratMediaEsTreVenditaControvaloreOperazione};

\item \emph{reddito da capitale}: \Eur{\dueStratMediaEsTreVenditaRedditoCapitale};
\item \emph{tasse sul reddito da capitale}: \Eur{\dueStratMediaEsTreVenditaTassaRedditoCapitale};

\item \emph{costo dell'operazione di vendita}: \Eur{\dueStratMediaEsTreVenditaCostoOperazione};
\item \emph{costo medio di acquisto delle quote vendute}: \Eur{\dueStratMediaEsTreVenditaCostoAcquistoQuoteVendute};
\item \emph{reddito diverso da minusvalenza sul capitale}: \Eur{\dueStratMediaEsTreVenditaRedditoDiversoMinusvalenzaCapitale};
\item \emph{reddito diverso da pagamento delle commissioni}: \Eur{\dueStratMediaEsTreVenditaRedditoDiversoPagamentoCommissioni};
\item \emph{reddito diverso totale}: \Eur{\dueStratMediaEsTreVenditaRedditoDiversoTotale};

\item \emph{controvalore totale}: \Eur{\dueStratMediaEsTreVenditaControvaloreTotale};

\item \emph{prezzo medio netto di vendita}: \Eur{\dueStratMediaEsTreVenditaPrezzoMedioNettoVendita};
\item \emph{rendimento percentuale}: \SI{\dueStratMediaEsTreVenditaRendimentoPercentuale}{\percent};
\item \emph{rendimento in valuta}: \Eur{\dueStratMediaEsTreVenditaRendimentoInValuta};
\end{itemize}
il         \emph{prezzo        medio         netto         di         vendita}        pari         a
\Eur{\dueStratMediaEsTreVenditaPrezzoMedioNettoVendita} è  minore del  \emph{prezzo medio  di carico
   nel  saldo}  pari  a  \Eur{\unoStratMediaEsTreSaldoPrezzoMedioCarico},  quindi  si  realizza  una
perdita; si osserva che:
\begin{itemize}
\item il \emph{reddito da capitale} è zero e le tasse sono nulle;
\item la differenza tra il \emph{controvalore dell'operazione di vendita} e il \emph{controvalore di
     carico nel saldo}:
  \begin{equation*}
    \num{\dueStratMediaEsTreVenditaControvaloreOperazione} -
    \num{\unoStratMediaEsTreSaldoControvaloreCarico}
    = \Eur{\dueStratMediaEsTreVenditaRedditoDiversoMinusvalenzaCapitale}
  \end{equation*}
  rappresenta la perdita di  reddito sul capitale dovuta alla vendita, è  negativa e contribuisce al
  calcolo del \emph{reddito diverso};
\item la differenza tra  il \emph{controvalore totale di vendita} e  il \emph{controvalore di carico
     nel saldo}:
  \begin{equation*}
    \num{\dueStratMediaEsTreVenditaControvaloreTotale} -
    \num{\unoStratMediaEsTreSaldoControvaloreCarico}
    = \Eur{\dueStratMediaEsTreVenditaRedditoDiversoTotale}
  \end{equation*}
  è negativa  e rappresenta l'intero  \emph{reddito diverso}, includendo  sia la perdita  di reddito
  dovuta alla vendita che il costo delle operazioni;
\end{itemize}
il \emph{reddito diverso totale} è una \emph{minusvalenza} da compensare con future operazioni.

%% ------------------------------------------------------------------------

La domanda  fondamentale è:  \textbf{possedendo un  numero di  quote a  un certo  \emph{prezzo medio
      effettivo} e un  certo \emph{prezzo medio di  carico} nel saldo, qual'è  il \emph{prezzo medio
      eseguito di vendita} che permette di realizzare un guadagno?}

Per esempio, si  posseggano \num{\unoStratMediaEsUnoSaldoNumeroQuote} quote a  un \emph{prezzo medio
   effettivo   nel  saldo}   pari  a   \Eur{\unoStratMediaEsUnoSaldoPrezzoMedioEffettivo}  e   a  un
\emph{prezzo medio di  carico nel saldo} pari  a \Eur{\unoStratMediaEsUnoSaldoPrezzoMedioCarico}, si
voglia determinare  il \emph{prezzo medio  \underline{eseguito} di vendita}  per tutte le  quote che
realizzi un guadagno.  Detto \(X\) tale valore, risulta:
\begin{itemize}
\item \emph{controvalore dell'operazione di vendita}: \(\num{\unoStratMediaEsUnoSaldoNumeroQuote} \times{} X\);
\item \emph{costo dell'operazione di vendita}:
  \begin{align*}
    \num{0,50} + \num{2,50} + \num{0,0024} \times{} (\num{\unoStratMediaEsUnoSaldoNumeroQuote} \times{} X)
    &= \num{3} + \num{0,0024} \times{} \num{\unoStratMediaEsUnoSaldoNumeroQuote} \times{} X = \\
    &= \num{3} + \num{0,24} \times{} X
  \end{align*}
\item \emph{tassa sul reddito da capitale}:
  \begin{align*}
    \num{0,26} \times{} \num{\unoStratMediaEsUnoSaldoNumeroQuote} \times{} (X - \num{\unoStratMediaEsUnoSaldoPrezzoMedioEffettivo})
    &= \num{26} \times{} X - \num{26} \times{} \num{50} = \\
    &= \num{26} \times{} X - \num{1300}
  \end{align*}
\item \emph{controvalore  totale di vendita}, la  quantità di denaro accreditata  sul Conto Corrente
  dopo la vendita:
  \begin{align*}
    (\num{\unoStratMediaEsUnoSaldoNumeroQuote} \times{} X)
    &- (\num{3} + \num{0,24} \times{} X)
      - (\num{26} \times{} X - \num{1300})
      = \\
    &= \num{\unoStratMediaEsUnoSaldoNumeroQuote} \times{} X
      - \num{3} - \num{0,24} \times{} X
      - \num{26} \times{} X + \num{1300}
      = \\
    &= (\num{\unoStratMediaEsUnoSaldoNumeroQuote} - \num{0,24} - \num{26}) \times{} X + (\num{1300} - \num{3})
      = \\
    &= \num{73,76} \times{} X + \num{1297}
  \end{align*}
\item \emph{prezzo medio netto di vendita}:
  \begin{equation*}
    \frac{\num{73,76} \times{} X + \num{1297}}{\unoStratMediaEsUnoSaldoNumeroQuote}
  \end{equation*}
\end{itemize}
il  pareggio  si  raggiungerebbe  vendendo  le  quote a  un  \emph{prezzo  medio  netto}  uguale  al
\emph{prezzo medio di carico nel saldo} pari a \Eur{\unoStratMediaEsUnoSaldoPrezzoMedioCarico}; quindi:
\begin{equation*}
  \frac{\num{73,76} \times{} X + \num{1297}}{\unoStratMediaEsUnoSaldoNumeroQuote} = \num{\unoStratMediaEsUnoSaldoPrezzoMedioCarico}
\end{equation*}
risolvendo rispetto a \(X\):
\begin{equation*}
  X = \frac{\num{\unoStratMediaEsUnoSaldoNumeroQuote} \times{} \num{\unoStratMediaEsUnoSaldoPrezzoMedioCarico} - \num{1297}}{\num{73,76}}
  = \Eur{50,4067}
\end{equation*}
vendendo con un \emph{prezzo medio eseguito} maggiore di \Eur{50,4067} si realizza un guadagno.

%page
\subsection{Metodo delle linee di investimento}

\input{conto-intesa--calcoli-esempio--strategie-prezzi--metodo-linee-investimento.inc}


Si è visto  come, dal punto di  vista contabile: tutte le quote  di uno stesso \Etf{}  in carico nel
Conto Titoli siano  parte dello stesso insieme;  in particolare, per ogni operazione  di vendita: le
\emph{tasse sul reddito da capitale} si calcolano  rispetto al valore attuale del \emph{prezzo medio
   effettivo nel  saldo}, indipendentemente da  quanti acquisti si siano  eseguiti in passato  e dal
loro \emph{prezzo medio effettivo di acquisto}.

Nonostante ciò, si immagini questo scenario:
\begin{itemize}
\item si disponga di un capitale di \Eur{15000,00};
\item si vogliano  definire tre \Virgolette{linee di investimento} formalmente  separate, dal nostro
  punto di vista; a ciascuna linea si allochino \Eur{\unoStratInvLineeCapitaleIniziale{}};
\item come primo investimento, per ciascuna linea, si acquistino quote dello stesso \Etf{};
\item in futuro si  prevede di vendere queste quote e acquistare, per  ogni linea, prodotti diversi,
  mantenendo    separato    il    destino    di    ognuno   dei    tre    capitali    iniziali    da
  \Eur{\unoStratInvLineeCapitaleIniziale{}};
\item si vuole che ogni linea di investimento risulti redditizia.
\end{itemize}

Le operazioni di acquisto siano come segue:
\begin{enumerate}
\item  per   la  prima  linea  di   investimento  si  acquisti   quando  il  prezzo  di   mercato  è
  \Eur{\unoStratInvLineeAcquistoPrezzoMedioEffettivo{}} per  quota; il numero di  quote acquistabili
  risulta:
  \begin{equation*}
    \num{\unoStratInvLineeCapitaleIniziale{}} /
    \num{\unoStratInvLineeAcquistoPrezzoMedioEffettivo{}}
    = \num{\unoStratInvLineeAcquistoNumeroQuote{}}
  \end{equation*}
  il riassunto dell'operazione di acquisto per la prima linea è:
  \begin{itemize}
  \item numero quote \num{\unoStratInvLineeAcquistoNumeroQuote{}};
  \item \emph{prezzo medio eseguito} \Eur{\unoStratInvLineeAcquistoPrezzoMedioEffettivo{}};
  \item \emph{controvalore dell'operazione} \Eur{\unoStratInvLineeAcquistoControvaloreOperazione{}};
  \item \emph{costo dell'operazione} \Eur{\unoStratInvLineeAcquistoCostoOperazione{}};
  \item \emph{controvalore totale} \Eur{\unoStratInvLineeAcquistoControvaloreTotale};
  \item \emph{prezzo medio di carico} \Eur{\unoStratInvLineeAcquistoPrezzoMedioCarico};
  \end{itemize}

\item  per  la  seconda   linea  di  investimento  si  acquisti  quando  il   prezzo  di  mercato  è
  \Eur{\dueStratInvLineeAcquistoPrezzoMedioEffettivo{}} per  quota; il numero di  quote acquistabili
  risulta:
  \begin{equation*}
    \num{\dueStratInvLineeCapitaleIniziale{}} /
    \num{\dueStratInvLineeAcquistoPrezzoMedioEffettivo{}}
    = \num{\dueStratInvLineeAcquistoNumeroQuote{}}
  \end{equation*}
  il riassunto dell'operazione di acquisto per la seconda linea è:
  \begin{itemize}
  \item numero quote \num{\dueStratInvLineeAcquistoNumeroQuote{}};
  \item \emph{prezzo medio eseguito} \Eur{\dueStratInvLineeAcquistoPrezzoMedioEffettivo{}};
  \item \emph{controvalore dell'operazione} \Eur{\dueStratInvLineeAcquistoControvaloreOperazione{}};
  \item \emph{costo dell'operazione} \Eur{\dueStratInvLineeAcquistoCostoOperazione{}};
  \item \emph{controvalore totale} \Eur{\dueStratInvLineeAcquistoControvaloreTotale};
  \item \emph{prezzo medio di carico} \Eur{\dueStratInvLineeAcquistoPrezzoMedioCarico};
  \end{itemize}

\item  per   la  terza  linea  di   investimento  si  acquisti   quando  il  prezzo  di   mercato  è
  \Eur{\treStratInvLineeAcquistoPrezzoMedioEffettivo{}} per  quota; il numero di  quote acquistabili
  risulta:
  \begin{equation*}
    \num{\unoStratInvLineeCapitaleIniziale{}} /
    \num{\treStratInvLineeAcquistoPrezzoMedioEffettivo{}}
    \simeq{} \num{\treStratInvLineeNumeroQuoteNonArrotondato{}}
    \simeq{} \num{\treStratInvLineeAcquistoNumeroQuote{}}
  \end{equation*}
  il riassunto dell'operazione di acquisto per la terza linea è:
  \begin{itemize}
  \item numero quote \num{\treStratInvLineeAcquistoNumeroQuote{}};
  \item \emph{prezzo medio eseguito} \Eur{\treStratInvLineeAcquistoPrezzoMedioEffettivo{}};
  \item \emph{controvalore dell'operazione} \Eur{\treStratInvLineeAcquistoControvaloreOperazione{}};
  \item \emph{costo dell'operazione} \Eur{\treStratInvLineeAcquistoCostoOperazione{}};
  \item \emph{controvalore totale} \Eur{\treStratInvLineeAcquistoControvaloreTotale};
  \item \emph{prezzo medio di carico} \Eur{\treStratInvLineeAcquistoPrezzoMedioCarico};
  \end{itemize}
\end{enumerate}

Nel Conto Titoli compare il saldo per il numero totale di quote acquistate:
\begin{equation*}
  \num{\unoStratInvLineeAcquistoNumeroQuote{}} +
  \num{\dueStratInvLineeAcquistoNumeroQuote{}} +
  \num{\treStratInvLineeAcquistoNumeroQuote{}} =
  \num{\treStratInvLineeSaldoNumeroQuote{}}
\end{equation*}
il  \emph{prezzo medio  effettivo nel  saldo} risulta  dalla media  ponderata dei  \emph{prezzi medi
   effettivi di acquisto}:
\begin{equation*}
  \frac{
     \num{\unoStratInvLineeAcquistoNumeroQuote{}} \times{} \num{\unoStratInvLineeAcquistoPrezzoMedioEffettivo{}} +
     \num{\dueStratInvLineeAcquistoNumeroQuote{}} \times{} \num{\dueStratInvLineeAcquistoPrezzoMedioEffettivo{}} +
     \num{\treStratInvLineeAcquistoNumeroQuote{}} \times{} \num{\treStratInvLineeAcquistoPrezzoMedioEffettivo{}}}
  {\treStratInvLineeSaldoNumeroQuote{}}
  = \Eur{\treStratInvLineeSaldoPrezzoMedioEffettivo{}}
\end{equation*}
il \emph{prezzo medio  di carico nel saldo}  risulta dalla media ponderata dei  \emph{prezzi medi di
   carico di acquisto}:
\begin{equation*}
  \frac{
     \num{\unoStratInvLineeAcquistoNumeroQuote{}} \times{} \num{\unoStratInvLineeAcquistoPrezzoMedioCarico{}} +
     \num{\dueStratInvLineeAcquistoNumeroQuote{}} \times{} \num{\dueStratInvLineeAcquistoPrezzoMedioCarico{}} +
     \num{\treStratInvLineeAcquistoNumeroQuote{}} \times{} \num{\treStratInvLineeAcquistoPrezzoMedioCarico{}}}
  {\treStratInvLineeSaldoNumeroQuote{}}
  = \Eur{\treStratInvLineeSaldoPrezzoMedioCarico{}}
\end{equation*}

Si è osservato  in precedenza che quando si  esegue una vendita parziale delle quote  in carico: nel
saldo del  Conto Titoli l'unico numero  che cambia è il  numero di quote in  carico, il \emph{prezzo
   medio effettivo nel saldo}  e il \emph{prezzo medio di carico nel  saldo} restano invariati!  Ciò
significa che: se si eseguono prima tutti gli  acquisti e poi tutte le vendite, \textbf{l'ordine con
   cui si liquidano  le quote delle tre linee  di investimento è ininfluente ai  fini dei rendimenti
   ottenuti}.

%page
\paragraph{Vendita delle quote della prima linea di investimento.}
Si   liquidino  le   \num{\quattroStratInvLineeVenditaNumeroQuote}  quote   della  prima   linea  di
investimento       al       \emph{prezzo       medio      effettivo       di       vendita}       di
\Eur{\quattroStratInvLineeVenditaPrezzoMedioEffettivo}, maggiore del \emph{prezzo medio di carico di
   acquisto}  delle quote  stesse  \Eur{\unoStratInvLineeAcquistoPrezzoMedioCarico},  ma minore  del
\emph{prezzo medio di carico nel saldo} \Eur{\treStratInvLineeSaldoPrezzoMedioCarico}.  Il riassunto
della contabilità fiscale come calcolato dalla Banca è il seguente:
\begin{itemize}
\item numero quote \num{\quattroStratInvLineeVenditaNumeroQuote};
\item \emph{prezzo medio effettivo} \Eur{\quattroStratInvLineeVenditaPrezzoMedioEffettivo};
\item \emph{controvalore dell'operazione} \Eur{\quattroStratInvLineeVenditaControvaloreOperazione};
\item \emph{reddito da capitale da plusvalenza sul capitale} \Eur{\quattroStratInvLineeVenditaRedditoCapitale};
\item \emph{tassa sul reddito da capitale} \Eur{\quattroStratInvLineeVenditaTassaRedditoCapitale};
\item \emph{costo dell'operazione} \Eur{\quattroStratInvLineeVenditaCostoOperazione};
\item \emph{costo di acquisto delle quote vendute} \Eur{\quattroStratInvLineeVenditaCostoAcquistoQuoteVendute};
\item \emph{reddito diverso da minusvalenza sul capitale} \Eur{\quattroStratInvLineeVenditaRedditoDiversoMinusvalenzaCapitale};
\item \emph{reddito diverso da pagamento commissioni} \Eur{\quattroStratInvLineeVenditaRedditoDiversoPagamentoCommissioni};
\item \emph{reddito diverso totale} \Eur{\quattroStratInvLineeVenditaRedditoDiversoTotale};
\item \emph{minusvalenza} registrata nella Posizione Fiscale \Eur{\quattroStratInvLineeVenditaMinusvalenza};
\item \emph{controvalore totale}  \Eur{\quattroStratInvLineeVenditaControvaloreTotale}, sono i soldi
  accreditati sul Conto Corrente;
\item \emph{prezzo medio netto} \Eur{\quattroStratInvLineeVenditaPrezzoMedioNettoVendita};
\item  \emph{rendimento   percentuale}  rispetto  al   \emph{prezzo  medio  di  carico   nel  saldo}
  \SI{\quattroStratInvLineeVenditaRendimentoPercentuale}{\percent};
\item  \emph{rendimento   in  valuta}  rispetto   al  \emph{prezzo   medio  di  carico   nel  saldo}
  \Eur{\quattroStratInvLineeVenditaRendimentoInValuta}.
\end{itemize}
Il risultato di  questa operazione dal punto di  vista della prima linea di  investimento può essere
calcolato in due modi:
\begin{enumerate}
\item con  la quantità  di soldi  addebitata sul  Conto Corrente  per acquistare  le quote,  cioè il
  \emph{controvalore  totale di  acquisto}  \Eur{\unoStratInvLineeAcquistoControvaloreTotale}, e  la
  quantità  di  soldi  accreditata  sul  Conto  Corrente  dopo  la  vendita  delle  quote,  cioè  il
  \emph{controvalore totale di vendita} \Eur{\quattroStratInvLineeVenditaControvaloreTotale};
\item       con       il      \emph{prezzo       medio       di       carico      di       acquisto}
  \Eur{\unoStratInvLineeAcquistoPrezzoMedioCarico}  e  il  \emph{prezzo   medio  netto  di  vendita}
  \Eur{\quattroStratInvLineeVenditaPrezzoMedioNettoVendita};
\end{enumerate}
il rendimento della compravendita risulta:
\begin{equation*}
  \CalcoloRendimentoPercentuale
  {\quattroStratInvLineeVenditaControvaloreTotale}
  {\unoStratInvLineeAcquistoControvaloreTotale} =
  \CalcoloRendimentoPercentuale
  {\quattroStratInvLineeVenditaPrezzoMedioNettoVendita}
  {\unoStratInvLineeAcquistoPrezzoMedioCarico} =
  \SI{\quattroStratInvLineeRendimentoPercentualeLinea}{\percent}
\end{equation*}

%page
\paragraph{Vendita delle quote della seconda linea di investimento.}
Si  liquidino   le  \num{\cinqueStratInvLineeVenditaNumeroQuote}   quote  della  seconda   linea  di
investimento       al       \emph{prezzo       medio      effettivo       di       vendita}       di
\Eur{\cinqueStratInvLineeVenditaPrezzoMedioEffettivo}, minore  del \emph{prezzo  medio di  carico di
   acquisto} delle  quote stesse  \Eur{\dueStratInvLineeAcquistoPrezzoMedioCarico}, ma  maggiore del
\emph{prezzo medio di carico nel saldo} \Eur{\treStratInvLineeSaldoPrezzoMedioCarico}.  Il riassunto
della contabilità fiscale come calcolato dalla Banca è il seguente:
\begin{itemize}
\item numero quote \num{\cinqueStratInvLineeVenditaNumeroQuote};
\item \emph{prezzo medio effettivo} \Eur{\cinqueStratInvLineeVenditaPrezzoMedioEffettivo};
\item \emph{controvalore dell'operazione} \Eur{\cinqueStratInvLineeVenditaControvaloreOperazione};
\item \emph{reddito da capitale da plusvalenza sul capitale} \Eur{\cinqueStratInvLineeVenditaRedditoCapitale};
\item \emph{tassa sul reddito da capitale} \Eur{\cinqueStratInvLineeVenditaTassaRedditoCapitale};
\item \emph{costo dell'operazione} \Eur{\cinqueStratInvLineeVenditaCostoOperazione};
\item \emph{costo di acquisto delle quote vendute} \Eur{\cinqueStratInvLineeVenditaCostoAcquistoQuoteVendute};
\item \emph{reddito diverso da minusvalenza sul capitale} \Eur{\cinqueStratInvLineeVenditaRedditoDiversoMinusvalenzaCapitale};
\item \emph{reddito diverso da pagamento commissioni} \Eur{\cinqueStratInvLineeVenditaRedditoDiversoPagamentoCommissioni};
\item \emph{reddito diverso totale} \Eur{\cinqueStratInvLineeVenditaRedditoDiversoTotale};
\item \emph{minusvalenza} registrata nella Posizione Fiscale \Eur{\cinqueStratInvLineeVenditaMinusvalenza};
\item \emph{controvalore totale}  \Eur{\cinqueStratInvLineeVenditaControvaloreTotale}, sono i soldi
  accreditati sul Conto Corrente;
\item \emph{prezzo medio netto} \Eur{\cinqueStratInvLineeVenditaPrezzoMedioNettoVendita};
\item  \emph{rendimento   percentuale}  rispetto  al   \emph{prezzo  medio  di  carico   nel  saldo}
  \SI{\cinqueStratInvLineeVenditaRendimentoPercentuale}{\percent};
\item  \emph{rendimento   in  valuta}  rispetto   al  \emph{prezzo   medio  di  carico   nel  saldo}
  \Eur{\cinqueStratInvLineeVenditaRendimentoInValuta}.
\end{itemize}
Il risultato di questa operazione dal punto di  vista della seconda linea di investimento può essere
calcolato in due modi:
\begin{enumerate}
\item con  la quantità  di soldi  addebitata sul  Conto Corrente  per acquistare  le quote,  cioè il
  \emph{controvalore  totale di  acquisto}  \Eur{\dueStratInvLineeAcquistoControvaloreTotale}, e  la
  quantità  di  soldi  accreditata  sul  Conto  Corrente  dopo  la  vendita  delle  quote,  cioè  il
  \emph{controvalore totale di vendita} \Eur{\cinqueStratInvLineeVenditaControvaloreTotale};
\item       con       il      \emph{prezzo       medio       di       carico      di       acquisto}
  \Eur{\dueStratInvLineeAcquistoPrezzoMedioCarico}  e  il  \emph{prezzo   medio  netto  di  vendita}
  \Eur{\cinqueStratInvLineeVenditaPrezzoMedioNettoVendita};
\end{enumerate}
il rendimento della compravendita risulta:
\begin{equation*}
  \CalcoloRendimentoPercentuale
  {\cinqueStratInvLineeVenditaControvaloreTotale}
  {\dueStratInvLineeAcquistoControvaloreTotale} =
  \CalcoloRendimentoPercentuale
  {\cinqueStratInvLineeVenditaPrezzoMedioNettoVendita}
  {\dueStratInvLineeAcquistoPrezzoMedioCarico} =
  \SI{\cinqueStratInvLineeRendimentoPercentualeLinea}{\percent}
\end{equation*}

%page
\paragraph{Vendita delle quote della terza linea di investimento.}
Si liquidino le \num{\seiStratInvLineeVenditaNumeroQuote} quote della terza linea di investimento al
\emph{prezzo  medio  effettivo  di vendita}  di  \Eur{\seiStratInvLineeVenditaPrezzoMedioEffettivo},
maggiore    del   \emph{prezzo    medio    di    carico   di    acquisto}    delle   quote    stesse
\Eur{\treStratInvLineeAcquistoPrezzoMedioCarico}, e  maggiore del  \emph{prezzo medio di  carico nel
   saldo}  \Eur{\treStratInvLineeSaldoPrezzoMedioCarico}.  Il  riassunto  della contabilità  fiscale
come calcolato dalla Banca è il seguente:
\begin{itemize}
\item numero quote \num{\seiStratInvLineeVenditaNumeroQuote};
\item \emph{prezzo medio effettivo} \Eur{\seiStratInvLineeVenditaPrezzoMedioEffettivo};
\item \emph{controvalore dell'operazione} \Eur{\seiStratInvLineeVenditaControvaloreOperazione};
\item \emph{reddito da capitale da plusvalenza sul capitale} \Eur{\seiStratInvLineeVenditaRedditoCapitale};
\item \emph{tassa sul reddito da capitale} \Eur{\seiStratInvLineeVenditaTassaRedditoCapitale};
\item \emph{costo dell'operazione} \Eur{\seiStratInvLineeVenditaCostoOperazione};
\item \emph{costo di acquisto delle quote vendute} \Eur{\seiStratInvLineeVenditaCostoAcquistoQuoteVendute};
\item \emph{reddito diverso da minusvalenza sul capitale} \Eur{\seiStratInvLineeVenditaRedditoDiversoMinusvalenzaCapitale};
\item \emph{reddito diverso da pagamento commissioni} \Eur{\seiStratInvLineeVenditaRedditoDiversoPagamentoCommissioni};
\item \emph{reddito diverso totale} \Eur{\seiStratInvLineeVenditaRedditoDiversoTotale};
\item \emph{minusvalenza} registrata nella Posizione Fiscale \Eur{\seiStratInvLineeVenditaMinusvalenza};
\item \emph{controvalore totale}  \Eur{\seiStratInvLineeVenditaControvaloreTotale}, sono i soldi
  accreditati sul Conto Corrente;
\item \emph{prezzo medio netto} \Eur{\seiStratInvLineeVenditaPrezzoMedioNettoVendita};
\item  \emph{rendimento   percentuale}  rispetto  al   \emph{prezzo  medio  di  carico   nel  saldo}
  \SI{\seiStratInvLineeVenditaRendimentoPercentuale}{\percent};
\item  \emph{rendimento   in  valuta}  rispetto   al  \emph{prezzo   medio  di  carico   nel  saldo}
  \Eur{\seiStratInvLineeVenditaRendimentoInValuta}.
\end{itemize}
Il risultato di  questa operazione dal punto di  vista della terza linea di  investimento può essere
calcolato in due modi:
\begin{enumerate}
\item con  la quantità  di soldi  addebitata sul  Conto Corrente  per acquistare  le quote,  cioè il
  \emph{controvalore  totale di  acquisto}  \Eur{\treStratInvLineeAcquistoControvaloreTotale}, e  la
  quantità  di  soldi  accreditata  sul  Conto  Corrente  dopo  la  vendita  delle  quote,  cioè  il
  \emph{controvalore totale di vendita} \Eur{\seiStratInvLineeVenditaControvaloreTotale};
\item       con       il      \emph{prezzo       medio       di       carico      di       acquisto}
  \Eur{\treStratInvLineeAcquistoPrezzoMedioCarico}  e  il  \emph{prezzo   medio  netto  di  vendita}
  \Eur{\seiStratInvLineeVenditaPrezzoMedioNettoVendita};
\end{enumerate}
il rendimento della compravendita risulta:
\begin{equation*}
  \CalcoloRendimentoPercentuale
  {\seiStratInvLineeVenditaControvaloreTotale}
  {\treStratInvLineeAcquistoControvaloreTotale} =
  \CalcoloRendimentoPercentuale
  {\seiStratInvLineeVenditaPrezzoMedioNettoVendita}
  {\treStratInvLineeAcquistoPrezzoMedioCarico} =
  \SI{\seiStratInvLineeRendimentoPercentualeLinea}{\percent}
\end{equation*}

%page
%% ------------------------------------------------------------
%% Fine.
%% ------------------------------------------------------------

%%% fdl-1.3.tex --

\appendix

\section{\rlap{GNU Free Documentation License}}

\begin{center}
  Version 1.3, 3 November 2008


  Copyright \copyright{} 2000, 2001, 2002, 2007, 2008  Free Software Foundation, Inc.

  \bigskip

  \texttt{<https://fsf.org/>}

  \bigskip

  Everyone is permitted  to copy and distribute verbatim  copies of this
  license document, but changing it is not allowed.
\end{center}


\begin{center}
  {\bf\large Preamble}
\end{center}

The purpose of this License is to make a manual, textbook, or other
functional and useful document ``free'' in the sense of freedom: to
assure everyone the effective freedom to copy and redistribute it,
with or without modifying it, either commercially or noncommercially.
Secondarily, this License preserves for the author and publisher a way
to get credit for their work, while not being considered responsible
for modifications made by others.

This License is a kind of ``copyleft'', which means that derivative
works of the document must themselves be free in the same sense.  It
complements the GNU General Public License, which is a copyleft
license designed for free software.

We have designed this License in order to use it for manuals for free
software, because free software needs free documentation: a free
program should come with manuals providing the same freedoms that the
software does.  But this License is not limited to software manuals;
it can be used for any textual work, regardless of subject matter or
whether it is published as a printed book.  We recommend this License
principally for works whose purpose is instruction or reference.


\subsection{APPLICABILITY AND DEFINITIONS}

% \begin{center}
%   {\Large\bf 1. APPLICABILITY AND DEFINITIONS\par}
%   \phantomsection
%   \addcontentsline{toc}{section}{1. APPLICABILITY AND DEFINITIONS}
% \end{center}

This License applies to any manual or other work, in any medium, that
contains a notice placed by the copyright holder saying it can be
distributed under the terms of this License.  Such a notice grants a
world-wide, royalty-free license, unlimited in duration, to use that
work under the conditions stated herein.  The ``\textbf{Document}'', below,
refers to any such manual or work.  Any member of the public is a
licensee, and is addressed as ``\textbf{you}''.  You accept the license if you
copy, modify or distribute the work in a way requiring permission
under copyright law.

A ``\textbf{Modified Version}'' of the Document means any work containing the
Document or a portion of it, either copied verbatim, or with
modifications and/or translated into another language.

A ``\textbf{Secondary Section}'' is a named appendix or a front-matter section of
the Document that deals exclusively with the relationship of the
publishers or authors of the Document to the Document's overall subject
(or to related matters) and contains nothing that could fall directly
within that overall subject.  (Thus, if the Document is in part a
textbook of mathematics, a Secondary Section may not explain any
mathematics.)  The relationship could be a matter of historical
connection with the subject or with related matters, or of legal,
commercial, philosophical, ethical or political position regarding
them.

The ``\textbf{Invariant Sections}'' are certain Secondary Sections whose titles
are designated, as being those of Invariant Sections, in the notice
that says that the Document is released under this License.  If a
section does not fit the above definition of Secondary then it is not
allowed to be designated as Invariant.  The Document may contain zero
Invariant Sections.  If the Document does not identify any Invariant
Sections then there are none.

The ``\textbf{Cover Texts}'' are certain short passages of text that are listed,
as Front-Cover Texts or Back-Cover Texts, in the notice that says that
the Document is released under this License.  A Front-Cover Text may
be at most 5 words, and a Back-Cover Text may be at most 25 words.

A ``\textbf{Transparent}'' copy of the Document means a machine-readable copy,
represented in a format whose specification is available to the
general public, that is suitable for revising the document
straightforwardly with generic text editors or (for images composed of
pixels) generic paint programs or (for drawings) some widely available
drawing editor, and that is suitable for input to text formatters or
for automatic translation to a variety of formats suitable for input
to text formatters.  A copy made in an otherwise Transparent file
format whose markup, or absence of markup, has been arranged to thwart
or discourage subsequent modification by readers is not Transparent.
An image format is not Transparent if used for any substantial amount
of text.  A copy that is not ``Transparent'' is called ``\textbf{Opaque}''.

Examples of suitable formats for Transparent copies include plain
ASCII without markup, Texinfo input format, LaTeX input format, SGML
or XML using a publicly available DTD, and standard-conforming simple
HTML, PostScript or PDF designed for human modification.  Examples of
transparent image formats include PNG, XCF and JPG.  Opaque formats
include proprietary formats that can be read and edited only by
proprietary word processors, SGML or XML for which the DTD and/or
processing tools are not generally available, and the
machine-generated HTML, PostScript or PDF produced by some word
processors for output purposes only.

The ``\textbf{Title Page}'' means, for a printed book, the title page itself,
plus such following pages as are needed to hold, legibly, the material
this License requires to appear in the title page.  For works in
formats which do not have any title page as such, ``Title Page'' means
the text near the most prominent appearance of the work's title,
preceding the beginning of the body of the text.

The ``\textbf{publisher}'' means any person or entity that distributes
copies of the Document to the public.

A section ``\textbf{Entitled XYZ}'' means a named subunit of the Document whose
title either is precisely XYZ or contains XYZ in parentheses following
text that translates XYZ in another language.  (Here XYZ stands for a
specific section name mentioned below, such as ``\textbf{Acknowledgements}'',
``\textbf{Dedications}'', ``\textbf{Endorsements}'', or ``\textbf{History}''.)
To ``\textbf{Preserve the Title}''
of such a section when you modify the Document means that it remains a
section ``Entitled XYZ'' according to this definition.

The Document may include Warranty Disclaimers next to the notice which
states that this License applies to the Document.  These Warranty
Disclaimers are considered to be included by reference in this
License, but only as regards disclaiming warranties: any other
implication that these Warranty Disclaimers may have is void and has
no effect on the meaning of this License.


\subsection{VERBATIM COPYING}
% \begin{center}
%   {\Large\bf 2. VERBATIM COPYING\par}
%   \phantomsection
%   \addcontentsline{toc}{section}{2. VERBATIM COPYING}
% \end{center}

You may copy and distribute the Document in any medium, either
commercially or noncommercially, provided that this License, the
copyright notices, and the license notice saying this License applies
to the Document are reproduced in all copies, and that you add no other
conditions whatsoever to those of this License.  You may not use
technical measures to obstruct or control the reading or further
copying of the copies you make or distribute.  However, you may accept
compensation in exchange for copies.  If you distribute a large enough
number of copies you must also follow the conditions in section~3.

You may also lend copies, under the same conditions stated above, and
you may publicly display copies.


\subsection{COPYING IN QUANTITY}
% \begin{center}
%   {\Large\bf 3. COPYING IN QUANTITY\par}
%   \phantomsection
%   \addcontentsline{toc}{section}{3. COPYING IN QUANTITY}
% \end{center}


If you publish printed copies (or copies in media that commonly have
printed covers) of the Document, numbering more than 100, and the
Document's license notice requires Cover Texts, you must enclose the
copies in covers that carry, clearly and legibly, all these Cover
Texts: Front-Cover Texts on the front cover, and Back-Cover Texts on
the back cover.  Both covers must also clearly and legibly identify
you as the publisher of these copies.  The front cover must present
the full title with all words of the title equally prominent and
visible.  You may add other material on the covers in addition.
Copying with changes limited to the covers, as long as they preserve
the title of the Document and satisfy these conditions, can be treated
as verbatim copying in other respects.

If the required texts for either cover are too voluminous to fit
legibly, you should put the first ones listed (as many as fit
reasonably) on the actual cover, and continue the rest onto adjacent
pages.

If you publish or distribute Opaque copies of the Document numbering
more than 100, you must either include a machine-readable Transparent
copy along with each Opaque copy, or state in or with each Opaque copy
a computer-network location from which the general network-using
public has access to download using public-standard network protocols
a complete Transparent copy of the Document, free of added material.
If you use the latter option, you must take reasonably prudent steps,
when you begin distribution of Opaque copies in quantity, to ensure
that this Transparent copy will remain thus accessible at the stated
location until at least one year after the last time you distribute an
Opaque copy (directly or through your agents or retailers) of that
edition to the public.

It is requested, but not required, that you contact the authors of the
Document well before redistributing any large number of copies, to give
them a chance to provide you with an updated version of the Document.


\subsection{MODIFICATIONS}
% \begin{center}
%   {\Large\bf 4. MODIFICATIONS\par}
%   \phantomsection
%   \addcontentsline{toc}{section}{4. MODIFICATIONS}
% \end{center}

You may copy and distribute a Modified Version of the Document under
the conditions of sections 2 and 3 above, provided that you release
the Modified Version under precisely this License, with the Modified
Version filling the role of the Document, thus licensing distribution
and modification of the Modified Version to whoever possesses a copy
of it.  In addition, you must do these things in the Modified Version:

\begin{itemize}
\item[A.]
  Use in the Title Page (and on the covers, if any) a title distinct
  from that of the Document, and from those of previous versions
  (which should, if there were any, be listed in the History section
  of the Document).  You may use the same title as a previous version
  if the original publisher of that version gives permission.

\item[B.]
  List on the Title Page, as authors, one or more persons or entities
  responsible for authorship of the modifications in the Modified
  Version, together with at least five of the principal authors of the
  Document (all of its principal authors, if it has fewer than five),
  unless they release you from this requirement.

\item[C.]
  State on the Title page the name of the publisher of the
  Modified Version, as the publisher.

\item[D.]
  Preserve all the copyright notices of the Document.

\item[E.]
  Add an appropriate copyright notice for your modifications
  adjacent to the other copyright notices.

\item[F.]
  Include, immediately after the copyright notices, a license notice
  giving the public permission to use the Modified Version under the
  terms of this License, in the form shown in the Addendum below.

\item[G.]
  Preserve in that license notice the full lists of Invariant Sections
  and required Cover Texts given in the Document's license notice.

\item[H.]
  Include an unaltered copy of this License.

\item[I.]
  Preserve the section Entitled ``History'', Preserve its Title, and add
  to it an item stating at least the title, year, new authors, and
  publisher of the Modified Version as given on the Title Page.  If
  there is no section Entitled ``History'' in the Document, create one
  stating the title, year, authors, and publisher of the Document as
  given on its Title Page, then add an item describing the Modified
  Version as stated in the previous sentence.

\item[J.]
  Preserve the network location, if any, given in the Document for
  public access to a Transparent copy of the Document, and likewise
  the network locations given in the Document for previous versions
  it was based on.  These may be placed in the ``History'' section.
  You may omit a network location for a work that was published at
  least four years before the Document itself, or if the original
  publisher of the version it refers to gives permission.

\item[K.]
  For any section Entitled ``Acknowledgements'' or ``Dedications'',
  Preserve the Title of the section, and preserve in the section all
  the substance and tone of each of the contributor acknowledgements
  and/or dedications given therein.

\item[L.]
  Preserve all the Invariant Sections of the Document,
  unaltered in their text and in their titles.  Section numbers
  or the equivalent are not considered part of the section titles.

\item[M.]
  Delete any section Entitled ``Endorsements''.  Such a section
  may not be included in the Modified Version.

\item[N.]
  Do not retitle any existing section to be Entitled ``Endorsements''
  or to conflict in title with any Invariant Section.

\item[O.]
  Preserve any Warranty Disclaimers.
\end{itemize}

If the Modified Version includes new front-matter sections or
appendices that qualify as Secondary Sections and contain no material
copied from the Document, you may at your option designate some or all
of these sections as invariant.  To do this, add their titles to the
list of Invariant Sections in the Modified Version's license notice.
These titles must be distinct from any other section titles.

You may add a section Entitled ``Endorsements'', provided it contains
nothing but endorsements of your Modified Version by various
parties---for example, statements of peer review or that the text has
been approved by an organization as the authoritative definition of a
standard.

You may add a passage of up to five words as a Front-Cover Text, and a
passage of up to 25 words as a Back-Cover Text, to the end of the list
of Cover Texts in the Modified Version.  Only one passage of
Front-Cover Text and one of Back-Cover Text may be added by (or
through arrangements made by) any one entity.  If the Document already
includes a cover text for the same cover, previously added by you or
by arrangement made by the same entity you are acting on behalf of,
you may not add another; but you may replace the old one, on explicit
permission from the previous publisher that added the old one.

The author(s) and publisher(s) of the Document do not by this License
give permission to use their names for publicity for or to assert or
imply endorsement of any Modified Version.


\subsection{COMBINING DOCUMENTS}
% \begin{center}
%   {\Large\bf 5. COMBINING DOCUMENTS\par}
%   \phantomsection
%   \addcontentsline{toc}{section}{5. COMBINING DOCUMENTS}
% \end{center}


You may combine the Document with other documents released under this
License, under the terms defined in section~4 above for modified
versions, provided that you include in the combination all of the
Invariant Sections of all of the original documents, unmodified, and
list them all as Invariant Sections of your combined work in its
license notice, and that you preserve all their Warranty Disclaimers.

The combined work need only contain one copy of this License, and
multiple identical Invariant Sections may be replaced with a single
copy.  If there are multiple Invariant Sections with the same name but
different contents, make the title of each such section unique by
adding at the end of it, in parentheses, the name of the original
author or publisher of that section if known, or else a unique number.
Make the same adjustment to the section titles in the list of
Invariant Sections in the license notice of the combined work.

In the combination, you must combine any sections Entitled ``History''
in the various original documents, forming one section Entitled
``History''; likewise combine any sections Entitled ``Acknowledgements'',
and any sections Entitled ``Dedications''.  You must delete all sections
Entitled ``Endorsements''.

\subsection{COLLECTIONS OF DOCUMENTS}
% \begin{center}
%   {\Large\bf 6. COLLECTIONS OF DOCUMENTS\par}
%   \phantomsection
%   \addcontentsline{toc}{section}{6. COLLECTIONS OF DOCUMENTS}
% \end{center}

You may make a collection consisting of the Document and other documents
released under this License, and replace the individual copies of this
License in the various documents with a single copy that is included in
the collection, provided that you follow the rules of this License for
verbatim copying of each of the documents in all other respects.

You may extract a single document from such a collection, and distribute
it individually under this License, provided you insert a copy of this
License into the extracted document, and follow this License in all
other respects regarding verbatim copying of that document.


\subsection{AGGREGATION WITH INDEPENDENT WORKS}
% \begin{center}
%   {\Large\bf 7. AGGREGATION WITH INDEPENDENT WORKS\par}
%   \phantomsection
%   \addcontentsline{toc}{section}{7. AGGREGATION WITH INDEPENDENT WORKS}
% \end{center}


A compilation of the Document or its derivatives with other separate
and independent documents or works, in or on a volume of a storage or
distribution medium, is called an ``aggregate'' if the copyright
resulting from the compilation is not used to limit the legal rights
of the compilation's users beyond what the individual works permit.
When the Document is included in an aggregate, this License does not
apply to the other works in the aggregate which are not themselves
derivative works of the Document.

If the Cover Text requirement of section~3 is applicable to these
copies of the Document, then if the Document is less than one half of
the entire aggregate, the Document's Cover Texts may be placed on
covers that bracket the Document within the aggregate, or the
electronic equivalent of covers if the Document is in electronic form.
Otherwise they must appear on printed covers that bracket the whole
aggregate.


\subsection{TRANSLATION}
% \begin{center}
%   {\Large\bf 8. TRANSLATION\par}
%   \phantomsection
%   \addcontentsline{toc}{section}{8. TRANSLATION}
% \end{center}


Translation is considered a kind of modification, so you may
distribute translations of the Document under the terms of section~4.
Replacing Invariant Sections with translations requires special
permission from their copyright holders, but you may include
translations of some or all Invariant Sections in addition to the
original versions of these Invariant Sections.  You may include a
translation of this License, and all the license notices in the
Document, and any Warranty Disclaimers, provided that you also include
the original English version of this License and the original versions
of those notices and disclaimers.  In case of a disagreement between
the translation and the original version of this License or a notice
or disclaimer, the original version will prevail.

If a section in the Document is Entitled ``Acknowledgements'',
``Dedications'', or ``History'', the requirement (section~4) to Preserve
its Title (section~1) will typically require changing the actual
title.


\subsection{TERMINATION}
% \begin{center}
%   {\Large\bf 9. TERMINATION\par}
%   \phantomsection
%   \addcontentsline{toc}{section}{9. TERMINATION}
% \end{center}


You may not copy, modify, sublicense, or distribute the Document
except as expressly provided under this License.  Any attempt
otherwise to copy, modify, sublicense, or distribute it is void, and
will automatically terminate your rights under this License.

However, if you cease all violation of this License, then your license
from a particular copyright holder is reinstated (a) provisionally,
unless and until the copyright holder explicitly and finally
terminates your license, and (b) permanently, if the copyright holder
fails to notify you of the violation by some reasonable means prior to
60 days after the cessation.

Moreover, your license from a particular copyright holder is
reinstated permanently if the copyright holder notifies you of the
violation by some reasonable means, this is the first time you have
received notice of violation of this License (for any work) from that
copyright holder, and you cure the violation prior to 30 days after
your receipt of the notice.

Termination of your rights under this section does not terminate the
licenses of parties who have received copies or rights from you under
this License.  If your rights have been terminated and not permanently
reinstated, receipt of a copy of some or all of the same material does
not give you any rights to use it.


\subsection{FUTURE REVISIONS OF THIS LICENSE}
% \begin{center}
%   {\Large\bf 10. FUTURE REVISIONS OF THIS LICENSE\par}
%   \phantomsection
%   \addcontentsline{toc}{section}{10. FUTURE REVISIONS OF THIS LICENSE}
% \end{center}


The Free Software Foundation may publish new, revised versions
of the GNU Free Documentation License from time to time.  Such new
versions will be similar in spirit to the present version, but may
differ in detail to address new problems or concerns.  See
\texttt{https://www.gnu.org/licenses/}.

Each version of the License is given a distinguishing version number.
If the Document specifies that a particular numbered version of this
License ``or any later version'' applies to it, you have the option of
following the terms and conditions either of that specified version or
of any later version that has been published (not as a draft) by the
Free Software Foundation.  If the Document does not specify a version
number of this License, you may choose any version ever published (not
as a draft) by the Free Software Foundation.  If the Document
specifies that a proxy can decide which future versions of this
License can be used, that proxy's public statement of acceptance of a
version permanently authorizes you to choose that version for the
Document.


\subsection{RELICENSING}
% \begin{center}
%   {\Large\bf 11. RELICENSING\par}
%   \phantomsection
%   \addcontentsline{toc}{section}{11. RELICENSING}
% \end{center}


``Massive Multiauthor Collaboration Site'' (or ``MMC Site'') means any
World Wide Web server that publishes copyrightable works and also
provides prominent facilities for anybody to edit those works.  A
public wiki that anybody can edit is an example of such a server.  A
``Massive Multiauthor Collaboration'' (or ``MMC'') contained in the
site means any set of copyrightable works thus published on the MMC
site.

``CC-BY-SA'' means the Creative Commons Attribution-Share Alike 3.0
license published by Creative Commons Corporation, a not-for-profit
corporation with a principal place of business in San Francisco,
California, as well as future copyleft versions of that license
published by that same organization.

``Incorporate'' means to publish or republish a Document, in whole or
in part, as part of another Document.

An MMC is ``eligible for relicensing'' if it is licensed under this
License, and if all works that were first published under this License
somewhere other than this MMC, and subsequently incorporated in whole
or in part into the MMC, (1) had no cover texts or invariant sections,
and (2) were thus incorporated prior to November 1, 2008.

The operator of an MMC Site may republish an MMC contained in the site
under CC-BY-SA on the same site at any time before August 1, 2009,
provided the MMC is eligible for relicensing.


\subsection{ADDENDUM: How to use this License for your documents}
% \begin{center}
%   {\Large\bf ADDENDUM: How to use this License for your documents\par}
%   \phantomsection
%   \addcontentsline{toc}{section}{ADDENDUM: How to use this License for your documents}
% \end{center}

To use this License in a document you have written, include a copy of
the License in the document and put the following copyright and
license notices just after the title page:

\bigskip
\begin{quote}
  Copyright \copyright{}  YEAR  YOUR NAME.
  Permission is granted to copy, distribute and/or modify this document
  under the terms of the GNU Free Documentation License, Version 1.3
  or any later version published by the Free Software Foundation;
  with no Invariant Sections, no Front-Cover Texts, and no Back-Cover Texts.
  A copy of the license is included in the section entitled ``GNU
  Free Documentation License''.
\end{quote}
\bigskip

If you have Invariant Sections, Front-Cover Texts and Back-Cover Texts,
replace the ``with \dots\ Texts.''\ line with this:

\bigskip
\begin{quote}
  with the Invariant Sections being LIST THEIR TITLES, with the
  Front-Cover Texts being LIST, and with the Back-Cover Texts being LIST.
\end{quote}
\bigskip

If you have Invariant Sections without Cover Texts, or some other
combination of the three, merge those two alternatives to suit the
situation.

If your document contains nontrivial examples of program code, we
recommend releasing these examples in parallel under your choice of
free software license, such as the GNU General Public License,
to permit their use in free software.

%%% end of file


%page
%% ------------------------------------------------------------
%% Fine.
%% ------------------------------------------------------------

\end{document}

%%% end of file
% Local Variables:
% mode: latex
% TeX-master: t
% ispell-local-dictionary: "italiano"
% End:

% conto-intesa-guida-etf.tex -*- coding: utf-8-unix -*-

\documentclass[12pt,a4paper]{article}
\renewcommand{\rmdefault}{ptm} % Times Roman font, for PDF output
\usepackage[italian]{babel}
\usepackage[utf8]{inputenc}
\usepackage[T1]{fontenc}
\usepackage[]{textcomp}
\usepackage[]{eurosym}
\usepackage[]{amsmath}
\usepackage[]{amsfonts}
\usepackage[]{amsthm}
\usepackage[]{amssymb}
\usepackage[]{syntonly}
\usepackage[copy-decimal-marker,retain-explicit-plus]{siunitx}
\usepackage[hidelinks]{hyperref} % To generate PDFs with hyperlinks

\pagestyle{headings}

%page
%% ------------------------------------------------------------
%% Comandi.
%% ------------------------------------------------------------

\newcommand{\Eur}[1]{\SI{#1}{\text{\euro{}}}}

% \everytexdraw{
%    \setgray 0 \linewd 0.01
%    \arrowheadsize l:0.12 w:0.04 \arrowheadtype t:F
% }

%page
%% ------------------------------------------------------------
%% Intestazione
%% ------------------------------------------------------------

\author{Marco Maggi}
\title{Pericolosa e incompleta guida agli investimenti in ETF con \emph{Banca Intesa Sanpaolo}}

\begin{document}

\maketitle

\begin{abstract}
  Questa guida è un  aiuto per chi volesse ricostruire i calcoli  di tenuta del Conto
  Titoli  di \emph{Banca  Intesa  Sanpaolo} per  gli  investimenti in  \emph{Exchange
     Traded Funds} (ETF).  Questa guida è incompleta:  non copre tutto ciò che c'è da
  sapere sugli ETF,  né l'operatività del sito di \emph{home  banking} di \emph{Banca
     Intesa Sanpaolo}.   Questa guida  è pericolosa:  ognuno la usa  a suo  rischio e
  pericolo.
\end{abstract}

\tableofcontents

\newpage{}

\noindent
Copyright \copyright{} 2017 Marco Maggi \texttt{<marco.maggi-ipsu@poste.it>}.

Permission is granted to copy, distribute and/or modify this document under the terms
of the GNU Free Documentation License, Version  1.3 or any later version published by
the Free Software  Foundation; with no Invariant Sections, no  Front-Cover Texts, and
no Back-Cover Texts.  A copy of the license is included in the section entitled ``GNU
Free Documentation License''.

\newpage{}

%page
\section{Dove reperire le informazioni}


Per  ogni  operazione di  compravendita  eseguita  sul  Conto Titoli:  nella  sezione
documenti del sito  di \emph{home banking} è disponibile una  \emph{nota di eseguito}
con  tutti  i  dati  dell'operazione.   Nei calcoli  qui  illustrati:  alcuni  valori
riportati  nelle note  di eseguito  sono  i dati  di  partenza; altri  valori sono  i
risultati ricalcolati per verifica.

%page
\section{Media aritmetica e media ponderata}


Nei  calcoli  di tenuta  Conto  Titoli  si  usa  spesso l'operazione  di  \emph{media
   aritmetica} e  in particolare  la sua riscrittura  come \emph{media  ponderata}; è
utile richiamare queste idee e alcune delle loro proprietà.

Si  ricorda che  la  \textbf{media  aritmetica} tra  i  numeri  \num{11}, \num{22}  e
\num{33} si scrive:
\begin{equation*}
  \frac{\num{11} + \num{22} + \num{33}}{3} = \num{22}
\end{equation*}
il denominatore è \num{3} perché al numeratore ci sono \num{3} numeri.

L'operazione si  costruisce nello stesso  modo se qualche  numero compare piú  di una
volta;  la media  aritmetica tra  i numeri:  \num{11}, \num{11},  \num{11}, \num{22},
\num{33} e \num{33} si scrive:
\begin{equation*}
  \frac{\num{11} + \num{11} + \num{11} + \num{22} + \num{33} + \num{33}}{6}
  = \frac{121}{6} \simeq \num{20,17}
\end{equation*}
il denominatore è \num{6} perché:
\begin{itemize}
\item \num{11} compare \num{3} volte;
\item \num{22} compare \num{1} volta;
\item \num{33} compare \num{2} volte;
\end{itemize}
quindi \(\num{3} + \num{1} + \num{2} = \num{6}\).

Il ``numero  di volte  in cui  un numero  compare'' si  chiama \textbf{molteplicità};
allora si può dire che:
\begin{itemize}
\item \num{11} compare con \emph{molteplicità} \num{3};
\item \num{22} compare con \emph{molteplicità} \num{1};
\item \num{33} compare con \emph{molteplicità} \num{2};
\end{itemize}
evidenziando le molteplicità e considerando le semplici identità:
\begin{align*}
  \num{11} + \num{11} + \num{11} &= \num{3} \times{} \num{11} &&&
  \num{22} &= \num{1} \times{} \num{22} &&&
  \num{33} + \num{33} &= \num{2} \times{} \num{33}
\end{align*}
l'espressione della \emph{media aritmetica} si può riscrivere:
\begin{align*}
  &\frac{\num{11} + \num{11} + \num{11} + \num{22} + \num{33} + \num{33}}{6}
  = \frac{\num{3} \times{} \num{11}
    + \num{1} \times{} \num{22}
    + \num{2} \times{} \num{33}}
    {6} = \\
  &=
    \frac{\num{3} \times{} \num{11}}{6} +
    \frac{\num{1} \times{} \num{22}}{6} +
    \frac{\num{2} \times{} \num{33}}{6}
  =
    \frac{\num{3}}{6} \times{} \num{11} +
    \frac{\num{1}}{6} \times{} \num{22} +
    \frac{\num{2}}{6} \times{} \num{33}
\end{align*}
in cui si evidenzia come:
\begin{itemize}
\item il numero \num{11} compaia con ``peso'' \num{3} rispetto al totale \num{6};
\item il numero \num{22} compaia con ``peso'' \num{1} rispetto al totale \num{6};
\item il numero \num{33} compaia con ``peso'' \num{2} rispetto al totale \num{6}.
\end{itemize}

In questo modo: la \emph{media aritmetica} tra i numeri \num{11}, \num{11}, \num{11},
\num{22}, \num{33} e  \num{33} è riscritta come media \textbf{media  ponderata} tra i
numeri \num{11}, \num{22} e \num{33} rispetto alle loro molteplicità \num{3}, \num{1}
e \num{2}.

Nei  calcoli di  tenuta  Conto Titoli:  si  eseguono medie  ponderate  tra prezzi  di
acquisto o vendita  di quote di fondi,  rispetto al numero di  quote acquistate; tale
numero di quote assume il ruolo di ``molteplicità di un prezzo''.

%page
\subsection{Calcolo incrementale della media ponderata}


L'attività di  investire in  un ETF  è costituita  da una  sequenza di  operazioni di
acquisto e vendita di quote del fondo; acquisti e vendite possono essere intercalati;
il  prezzo  del  totale  delle   quote  possedute  viene  calcolato  aggiornando  una
\emph{media ponderata} dopo  ogni operazione.  È utile richiamare come  il calcolo di
una \emph{media ponderata} possa essere eseguito passo dopo passo.

Si considerino  ancora i  numeri \num{11}, \num{11},  \num{11}, \num{22},  \num{33} e
\num{33}; si  è calcolato che  la loro  \emph{media aritmetica} può  essere riscritta
come \emph{media ponderata} e risulta:
\begin{equation*}
  \frac{\num{11} + \num{11} + \num{11} + \num{22} + \num{33} + \num{33}}{6}
  = \frac{\num{3} \times{} \num{11}
     + \num{1} \times{} \num{22}
     + \num{2} \times{} \num{33}}
  {6} \simeq \num{20,17}
\end{equation*}
si considerino  prima i  numeri \num{11},  \num{22} e \num{33};  poi si  aggiungano i
numeri \num{11} e \num{11}; infine si aggiunga il numero \num{33}:
\begin{enumerate}
\item la \emph{media ponderata} tra i numeri \num{11}, \num{22} e \num{33} risulta:
  \begin{equation*}
    \frac{\num{1} \times{} \num{11}
       + \num{1} \times{} \num{22}
       + \num{1} \times{} \num{33}}{\num{3}}
    = \num{22}
  \end{equation*}

\item  ora si  aggiungano  i numeri  \num{11} e  \num{11};  la \emph{media  ponderata
     aggiornata} risulta:
  \begin{equation*}
    \frac{\num{3} \times{} \num{11}
       + \num{1} \times{} \num{22}
       + \num{1} \times{} \num{33}}{\num{5}}
    = \frac{\num{88}}{\num{5}} \simeq \num{17,6}
  \end{equation*}
  e si può scrivere:
  \begin{align*}
    \frac{\num{3} \times{} \num{11}
    + \num{1} \times{} \num{22}
    + \num{1} \times{} \num{33}}{\num{5}}
    = \frac{\num{2} \times{} \num{11}}{\num{5}}
    + \frac{\num{1} \times{} \num{11}
    + \num{1} \times{} \num{22}
    + \num{1} \times{} \num{33}}{\num{5}}
  \end{align*}
  la seconda frazione al membro di destra si può riscrivere:
  \begin{align*}
    \frac{\num{3}}{\num{5}} \times \frac{\num{5}}{\num{3}} \times
    \frac{\num{1} \times{} \num{11}
    + \num{1} \times{} \num{22}
    + \num{1} \times{} \num{33}}{\num{5}}
    &= \frac{\num{3}}{\num{5}} \times
      \frac{\num{1} \times{} \num{11}
      + \num{1} \times{} \num{22}
      + \num{1} \times{} \num{33}}{\num{3}} \\
    &\simeq \frac{\num{3}}{\num{5}} \times{} \num{20,17}
      = \frac{\num{3} \times{} \num{20,17}}{\num{5}}
  \end{align*}
  e quindi la \emph{media ponderata aggiornata} risulta:
  \begin{align*}
    \frac{\num{2} \times{} \num{11}}{\num{5}}
    + \frac{\num{3} \times{} \num{20,17}}{\num{5}}
    = \frac{\num{2} \times{} \num{11} + \num{3} \times{} \num{20,17}}{\num{5}}
    \simeq \num{17,6}
  \end{align*}
  cioè la \emph{media ponderata aggiornata} è pari alla \emph{media ponderata} tra:
  \begin{itemize}
  \item la \emph{media ponderata precedente}, con la sua molteplicità;
  \item il nuovo numero aggiunto, con la sua molteplicità;
  \end{itemize}

\item  infine  si  aggiunga  il  numero  \num{33};  la  nuova  \emph{media  ponderata
     aggiornata} è la \emph{media ponderata totale} e procedendo come prima risulta:
  \begin{equation*}
    \frac{\num{1} \times{} \num{33}
       + \num{5} \times{} \num{17,6}}{\num{6}}
    \simeq \num{20,17}
  \end{equation*}
\end{enumerate}

Nei calcoli  di tenuta  Conto Titoli:  dopo ogni acquisto  di quote  di un  fondo, si
calcola  il  prezzo  medio  di   ogni  quota  posseduta  come  \emph{media  ponderata
   aggiornata} tra:
\begin{itemize}
\item  la  \emph{media ponderata  precedente},  con  il  precedente numero  di  quote
  possedute come molteplicità;
\item il prezzo delle nuove quote acquistate,  con il numero di quote acquistate come
  molteplicità.
\end{itemize}


%page
\section{Direzione dei trasferimenti di denaro}


Con  abusiva  semplificazione, le  somme  di  denaro  descritte  in questa  guida  si
ritengono scambiate tra il Conto Corrente e il Conto Titoli:
\begin{itemize}
\item le somme di denaro \emph{investite} in  quote di fondi sono in \emph{uscita dal
     Conto Corrente} e in \emph{entrata nel Conto Titoli};
\item le somme di denaro \emph{disinvestite}  da quote di fondi sono in \emph{entrata
     nel Conto Corrente} e in \emph{uscita dal Conto Titoli}.
\end{itemize}

Per esempio,  si supponga  di acquistare  quote di un  fondo per  \Eur{1000,00} tutto
incluso; questa quantità di denaro esce dal Conto Corrente ed entra nel Conto Titoli;
si considerino i due casi:
\begin{itemize}
\item si vendono le quote al prezzo netto di \Eur{1200,00}; questa quantità di denaro
  esce dal Conto Titoli ed entra nel  Conto Corrente; il bilancio delle operazioni si
  esprime con la differenza algebrica:
  \begin{equation*}
    \num{1200,00} - \num{1000,00} = \Eur{+200,00}
  \end{equation*}
  cioè, nel  saldo finale, \Eur{200,00} ``appaiono''  nel Conto Titoli, vi  escono ed
  entrano nel Conto Corrente: un guadagno;
\item si vendono le quote al prezzo  netto di \Eur{900,00}; questa quantità di denaro
  esce dal Conto Titoli ed entra nel  Conto Corrente; il bilancio delle operazioni si
  esprime con la differenza algebrica:
  \begin{equation*}
    \num{900,00} - \num{1000,00} = \Eur{-100,00}
  \end{equation*}
  cioè, nel saldo finale, \Eur{100,00} escono dal Conto Corrente ed entrano nel Conto
  Titoli in cui poi ``spariscono'': una perdita;
\end{itemize}
le differenze algebriche hanno due interpretazioni:
\begin{itemize}
\item \emph{uscite dal Conto Titoli} meno \emph{entrate nel Conto Titoli};
\item \emph{entrate nel Conto Corrente} meno \emph{uscite dal Conto Corrente}.
\end{itemize}

%page
\section{Descrizione di un'operazione di acquisto}


Scelta la strategia di  acquisto con prezzo limite: in una certa data,  e a una certa
ora, si inserisce  l'ordine di acquisto per un  numero di quote del fondo  a un certo
prezzo massimo (si può acquistare a qualsiasi prezzo al di sotto del limite fissato).
L'ordine  può essere  eseguito  in piú  fasi  in cui  solo una  parte  delle quote  è
acquistata, in  ogni fase a  un prezzo  diverso; è possibile  che non tutte  le quote
siano acquistate, nel qual caso ci interessa solo la parte ``eseguita'' dell'ordine.

Il \textbf{prezzo medio eseguito  di un acquisto} è la media  ponderata dei prezzi di
acquisto rispetto al  numero di quote acquistate.  Per esempio,  se un singolo ordine
di acquisto per \num{90} quote è eseguito nelle tre fasi:
\begin{enumerate}
\item acquisto di \num{20} quote al prezzo di \Eur{22,00};
\item acquisto di \num{30} quote al prezzo di \Eur{33,00};
\item acquisto di \num{40} quote al prezzo di \Eur{44,00};
\end{enumerate}
allora il \emph{prezzo medio eseguito} risulta:
\begin{equation*}
  \frac{20 \times{} \num{22,00}
     + 30 \times{} \num{33,00}
     + 40 \times \num{44,00}}{20 + 30 + 40}
  = \Eur{35,44}
\end{equation*}

Il  \textbf{controvalore dell'operazione  di acquisto}  è il  prodotto tra  il numero
totale di  quote acquistate e  il \emph{prezzo medio  eseguito}.  Per esempio,  se si
acquistano  \num{90}  quote  al  \emph{prezzo   medio  eseguito}  di  \Eur{35,44}  il
controvalore dell'operazione risulta:
\begin{equation*}
  90 \times{} \num{35,44} = \Eur{3189,60}
\end{equation*}

Il  \textbf{costo  dell'operazione  di  acquisto}  è la  somma  tra  costi,  spese  e
commissioni associate all'operazione.  Per ogni  ordine di acquisto eseguito, occorre
pagare:
\begin{itemize}
\item le spese per la banca pari a un fisso di \Eur{0,50};
\item i costi di intermediario pari a un fisso di \Eur{2,50};
\item   le   commissioni   per   la   banca   pari   allo   \SI{0,24}{\percent}   del
  \emph{controvalore dell'operazione}.
\end{itemize}
Se un  ordine non è eseguito:  si paga nulla.  Se  un ordine è eseguito  in piú fasi:
nella prima fase si  pagano i costi fissi piú la  commissione percentuale; nelle fasi
successive  si   paga  solo   la  commissione  percentuale.    Per  esempio,   se  il
\emph{controvalore dell'operazione}  è \Eur{3189,60} il  \emph{costo dell'operazione}
risulta:
\begin{equation*}
  \num{0,50} + \num{2,50} + \num{0,0024} \times{} \num{3189,60}
  = \Eur{10,65}
\end{equation*}

Il  \textbf{controvalore totale  di un  acquisto} è  la somma  tra \emph{controvalore
   dell'operazione} e \emph{costo dell'operazione}; è la quantità di denaro in uscita
dal  Conto   Corrente  e  in   entrata  nel  Conto   Titoli.   Per  esempio,   se  il
\emph{controvalore dell'operazione} è \Eur{3189,60} e il \emph{costo dell'operazione}
è pari a \Eur{10,65}, il \emph{controvalore totale} risulta:
\begin{equation*}
  \num{3189,60} + \num{10,65} = \Eur{3200,25}
\end{equation*}

Il  \textbf{prezzo   medio  di  carico  di   un  acquisto}  é  il   rapporto  tra  il
\emph{controvalore  totale}  e  il  numero  di   quote  acquistate  ed  è  usato  per
l'aggiornamento del saldo:  è il prezzo medio di una  singola quota acquistata, tutto
incluso.  Per  esempio, se  si sono acquistate  \num{90} quote  al \emph{controvalore
   totale} di \Eur{3200,25}, il \emph{prezzo medio di carico} risulta:
\begin{equation*}
  \num{3200,25} / \num{90} = \Eur{35,56}
\end{equation*}

%page
\section{Aggiornamento del saldo dopo un'operazione di acquisto}


Il saldo di un investimento in quote di ETF è rappresentato dalle tre quantità:
\begin{itemize}
\item numero di quote in carico nel Conto Titoli;
\item prezzo medio di carico nel saldo;
\item \emph{Net Asset Value} (NAV) del prezzo  medio di carico (da non confondere con
  il NAV dell'attività  sottostante il fondo);
\end{itemize}
dopo l'esecuzione di ogni operazione di acquisto occorre ricalcolare il saldo.

Il \textbf{prezzo medio di carico nel saldo} è la media ponderata tra il \emph{prezzo
   medio di \underline{carico} del nuovo acquisto} e il precedente \emph{prezzo medio
   di carico nel saldo}.

Il \textbf{NAV  del prezzo medio  di carico  nel saldo} è  la media ponderata  tra il
\emph{prezzo medio \underline{eseguito} del nuovo acquisto} e il precedente \emph{NAV
   del  prezzo medio  di carico  nel saldo};  questo valore  è usato  per il  calcolo
dell'imponibile per la tassazione sui redditi da capitale.

Per esempio, si supponga di eseguire le operazioni:
\begin{enumerate}
\item acquisto di  \num{2} quote al \emph{prezzo medio eseguito}  di \Eur{20,00} e al
  \emph{prezzo medio di carico} di:
  \begin{equation*}
    (\num{2} \times{} \num{20,00}) + [\num{0,5} + \num{2,5}
    + \num{0,0024} \times{} (\num{2} \times{} \num{20,00})] = \Eur{21,54}
  \end{equation*}
\item acquisto di  \num{3} quote al \emph{prezzo medio eseguito}  di \Eur{30,00} e al
  \emph{prezzo medio di carico} di:
  \begin{equation*}
    (\num{3} \times{} \num{30,00}) + [\num{0,5} + \num{2,5}
    + \num{0,0024} \times{} (\num{3} \times{} \num{30,00})] = \Eur{31,07}
  \end{equation*}
\item acquisto di  \num{4} quote al \emph{prezzo medio eseguito}  di \Eur{40,00} e al
  \emph{prezzo medio di carico} di:
  \begin{equation*}
    (\num{4} \times{} \num{40,00}) + [\num{0,5} + \num{2,5}
    + \num{0,0024} \times{} (\num{4} \times{} \num{40,00})] = \Eur{40,85}
  \end{equation*}
\end{enumerate}
all'inizio si  consideri un Conto  Titoli vuoto,  con saldo convenzionale  di \num{0}
quote al prezzo di \Eur{0}.  L'aggiornamento del saldo avviene nel seguente modo.

\begin{enumerate}
\item  Dopo il  primo acquisto:  il numero  di  quote in  carico nel  Conto Titoli  è
  \num{2}; il \emph{prezzo medio di carico nel saldo} è \Eur{21,54}; il \emph{NAV del
     prezzo medio di carico nel saldo} è \Eur{20,00}.

\item Dopo  il secondo  acquisto: il  numero di quote  in carico  nel Conto  Titoli è
  \(2 + 3 = 5\); il \emph{prezzo medio di carico nel saldo} è la media ponderata:
  \begin{equation*}
    \frac{\num{2} \times{} \num{21,54} + \num{3} \times{} \num{31,07}}{2 + 3}
    = \Eur{27,23}
  \end{equation*}
 il \emph{NAV del prezzo medio di carico nel saldo} è la media ponderata:
  \begin{equation*}
    \frac{\num{2} \times{} \num{20,00} + \num{3} \times{} \num{30,00}}{2 + 3}
    = \Eur{26,00}
  \end{equation*}

\item  Dopo il  terzo acquisto:  il numero  di  quote in  carico nel  Conto Titoli  è
  \(5 + 4 = 9\); il \emph{prezzo medio di carico nel saldo} è la media ponderata:
  \begin{equation*}
    \frac{5 \times{} \num{27,23} + 4 \times{} \num{40,85}}{5 + 4}
    = \Eur{33,28}
  \end{equation*}
 il \emph{NAV del prezzo medio di carico nel saldo} è la media ponderata:
  \begin{equation*}
    \frac{\num{5} \times{} \num{26,00} + \num{4} \times{} \num{40}}{5 + 4}
    = \Eur{32,22}
  \end{equation*}
\end{enumerate}

Evidenziando il \emph{prezzo medio di carico  nel saldo}: ogni nuova quota acquistata
viene ``buttata nel secchio'' insieme  alle vecchie, indipendentemente dal suo prezzo
di acquisto; è come se ogni quota  fosse stata acquistata allo stesso prezzo, pari al
\emph{prezzo medio di carico nel saldo}.

%page
\section{Descrizione di un'operazione di vendita}


Scelta la strategia  di vendita con prezzo limite:  in una certa data, e  a una certa
ora, si  inserisce l'ordine di vendita  per un numero di  quote del fondo a  un certo
prezzo minimo  (si può vendere  a qualsiasi prezzo al  di sopra del  limite fissato).
L'ordine può essere eseguito in piú fasi in cui solo una parte delle quote è venduta,
in ogni fase a  un prezzo diverso; è possibile che non tutte  le quote siano vendute,
nel qual caso ci interessa solo la parte ``eseguita'' dell'ordine.

Il \textbf{prezzo medio eseguito  di una vendita} è la media  ponderata dei prezzi di
vendita rispetto al  numero di quote vendute.   Per esempio, se un  singolo ordine di
vendita per \num{90} quote è eseguito nelle fasi:
\begin{enumerate}
\item vendita di \num{20} quote al prezzo di \Eur{22,00};
\item vendita di \num{30} quote al prezzo di \Eur{33,00};
\item vendita di \num{40} quote al prezzo di \Eur{44,00};
\end{enumerate}
allora il \emph{prezzo medio eseguito} risulta:
\begin{equation*}
  \frac{20 \times{} \num{22,00}
     + 30 \times{} \num{33,00}
     + 40 \times \num{44,00}}{20 + 30 + 40}
  = \Eur{35,44}
\end{equation*}

Il  \textbf{controvalore dell'operazione  di vendita}  è  il prodotto  tra il  numero
totale  di quote  vendute e  il  \emph{prezzo medio  eseguito}.  Per  esempio, se  si
vendono \num{90} quote al \emph{prezzo medio eseguito} di \Eur{35,44} il controvalore
dell'operazione risulta:
\begin{equation*}
  90 \times{} \num{35,44} = \Eur{3189,60}
\end{equation*}

Per ogni vendita occorre pagare le \textbf{tasse sul reddito da capitale}\footnote{Si
   deve pagare  anche il bollo sull'estratto  conto trimestrale per il  Conto Titoli,
   pari  allo \SI{0,2}{\percent}  all'anno; il  bollo è  addebitato direttamente  sul
   Conto Corrente, perciò in questa guida  lo si considera conteggiato a parte.}; per
fondi che non contengono Titoli dello  Stato Italiano, e assimilati, l'aliquota unica
è  del  \SI{26}{\percent}.  L'imponibile  per  ogni  quota  è  la differenza  tra  il
\emph{prezzo medio eseguito della vendita} e  il \emph{NAV del prezzo medio di carico
   nel  saldo}.   Per  esempio,  se  il  numero  di  quote  vendute  è  \num{90},  il
\emph{prezzo medio  eseguito} è  di \Eur{35,44}  e il \emph{NAV  del prezzo  medio di
   carico} è \Eur{34,70} il calcolo delle tasse si scrive:
\begin{equation*}
  \left(\num{35,44} - \num{34,70}\right) \times{} \num{90} \times{} \num{0,26}
  = \Eur{17,32}
\end{equation*}
che, evidenziando il \emph{controvalore dell'operazione}, si riscrive:
\begin{equation*}
  [\num{3189,60} - (\num{90} \times{} \num{34,70})] \times{} \num{0,26}
  = \Eur{17,32}
\end{equation*}

Il  \textbf{costo  dell'operazione  di  vendita}  è  la  somma  tra  costi,  spese  e
commissioni associate all'operazione.   Per ogni ordine di  vendita eseguito, occorre
pagare:
\begin{itemize}
\item le spese per la banca pari a un fisso di \Eur{0,50};
\item i costi di intermediario pari a un fisso di \Eur{2,50};
\item   le   commissioni   per   la   banca   pari   allo   \SI{0,24}{\percent}   del
  \emph{controvalore dell'operazione}.
\end{itemize}
Se un  ordine non è eseguito:  si paga nulla.  Se  un ordine è eseguito  in piú fasi:
nella prima fase si  pagano i costi fissi piú la  commissione percentuale; nelle fasi
successive  si   paga  solo   la  commissione  percentuale.    Per  esempio,   se  il
\emph{controvalore dell'operazione}  è \Eur{3189,60} il  \emph{costo dell'operazione}
risulta:
\begin{equation*}
  \num{0,50} + \num{2,50} + \num{0,0024} \times{} \num{3189,60}
  = \Eur{10,65}
\end{equation*}

Da osservare  che: sia il  calcolo delle \emph{tasse sul  reddito da capitale}  che i
calcoli del \emph{costo dell'operazione} di acquisto  e vendita si eseguono usando il
\emph{prezzo medio  eseguito}; in pratica: si  pagano le tasse anche  sul \emph{costo
   dell'operazione} di  acquisto e vendita, che  però sono redditi di  qualcun altro,
non dell'investitore!

Il \textbf{controvalore totale di vendita}  è la differenza tra il \emph{controvalore
   dell'operazione} e  la somma  tra \emph{costo  dell'operazione} e  \emph{tasse sul
   reddito da capitale}; è  la quantità di denaro in entrata nel  Conto Corrente e in
uscita dal  Conto Titoli.  Per  esempio, se il \emph{controvalore  dell'operazione} è
\Eur{3189,60}, il \emph{costo  dell'operazione} è pari a \Eur{10,65} e  le tasse sono
pari a \Eur{17,32}, il \emph{controvalore totale} risulta:
\begin{equation*}
  \num{3189,60} - \num{10,65} - \num{17,32} = \Eur{3161,63}
\end{equation*}

Per un'operazione di vendita: \textbf{non esiste} un \textbf{prezzo medio di carico};
la vendita modifica solo il numero di quote in carico nel saldo.  Invece si definisce
il \textbf{prezzo medio netto di vendita}  pari al rapporto tra il \emph{controvalore
   totale} e il  numero di quote vendute.   Per esempio, se si  sono vendute \num{90}
quote al \emph{controvalore totale} di  \Eur{3161,63}, il \emph{prezzo medio netto di
   vendita} risulta:
\begin{equation*}
  \num{3161,63} / \num{90} = \Eur{35,13}
\end{equation*}

%page
\section{Aggiornamento del saldo dopo un'operazione di vendita}


Il saldo di un investimento in quote di ETF è rappresentato dalle tre quantità:
\begin{itemize}
\item numero di quote in carico nel Conto Titoli;
\item prezzo medio di carico nel saldo;
\item \emph{Net Asset Value} (NAV) del prezzo  medio di carico (da non confondere con
  il NAV dell'attività  sottostante il fondo);
\end{itemize}
dopo l'esecuzione di ogni operazione di vendita occorre ricalcolare il saldo.

Sia il \textbf{prezzo medio di carico nel  saldo} che il \textbf{NAV del prezzo medio
   di carico  nel saldo}  restano immutati  dopo una vendita:  essa modifica  solo il
numero di quote in carico.  Per esempio, si supponga di eseguire le operazioni:
\begin{enumerate}
\item acquisto di \num{20} quote al  \emph{prezzo medio eseguito} di \Eur{20,00} e al
  \emph{prezzo medio di carico} di \Eur{21,54};
\item vendita di \num{15} quote; ai fini del saldo: non importa a quale prezzo;
\item acquisto di \num{30} quote al  \emph{prezzo medio eseguito} di \Eur{30,00} e al
  \emph{prezzo medio di carico} di \Eur{31,07};
\item vendita di \num{5} quote; ai fini del saldo: non importa a quale prezzo.
\end{enumerate}
all'inizio il si consideri un Conto  Titoli vuoto, con saldo convenzionale di \num{0}
quote al prezzo di \Eur{0}; l'aggiornamento del saldo avviene nel seguente modo.

\begin{enumerate}
\item Dopo la prima  operazione (acquisto): il numero di quote  in carico è \num{20};
  il \emph{prezzo medio  di carico nel saldo} è \Eur{21,54};  il \emph{NAV del prezzo
     medio di carico nel saldo} è \Eur{20,00}.

\item  Dopo  la  seconda  operazione  (vendita):  il numero  di  quote  in  carico  è
  \(20  - 15  =  5\);  il \emph{prezzo  medio  di carico  nel  saldo}  è invariato  a
  \Eur{21,54};  il \emph{NAV  del prezzo  medio di  carico nel  saldo} è  invariato a
  \Eur{20,00}.

\item  Dopo  il  terza  operazione  (acquisto):  il  numero  di  quote  in  carico  è
  \(5 + 30 = 35\); il \emph{prezzo medio di carico nel saldo} è la media ponderata:
  \begin{equation*}
    \frac{5 \times{} \num{21,54} + 30 \times{} \num{31,07}}{5 + 30}
    = \Eur{29,71}
  \end{equation*}
  il \emph{NAV del prezzo medio di carico nel saldo} è la media ponderata:
  \begin{equation*}
    \frac{5 \times{} \num{20,00} + 30 \times{} \num{30,00}}{5 + 30}
    = \Eur{28,57}
  \end{equation*}

\item  Dopo  la  quarta  operazione  (vendita):  il  numero  di  quote  in  carico  è
  \(35  - 5  =  30\);  il \emph{prezzo  medio  di carico  nel  saldo}  è invariato  a
  \Eur{29,71}; il \emph{prezzo medio di carico nel saldo} è invariato a \Eur{28,57}.
\end{enumerate}

%page
\section{Rendimento indicato nel riepilogo del patrimonio}


Nel sito  di \emph{home banking}  di Intesa Sanpaolo è  indicato, per ogni  fondo nel
Conto Titoli, il  saldo costituito dal numero  di quote in carico  e dal \emph{prezzo
   medio di  carico}; in piú è  visibile la stima dell'\emph{Utile  o Perdita} (U/P),
nell'ipotesi di vendita  di tutte le quote al prezzo  dell'ultima quotazione rilevata
sulla Borsa di  negoziazione del titolo.  L'\emph{Utile} mostrato è  al lordo sia dei
costi di vendita che  delle tasse sui redditi; la \emph{Perdita}  mostrata è al lordo
dei costi vendita.

Si supponga  di avere in  carico \num{16}  quote di un  ETF al \emph{prezzo  medio di
   carico nel saldo} di \Eur{128,98}:
\begin{itemize}
\item  se  l'ultima  quotazione  rilevata  per  il  prezzo  di  una  quota  fosse  di
  \Eur{132,00},  l'\emph{Utile percentuale}  mostrato  nella  tabella del  patrimonio
  risulterebbe:
  \begin{equation*}
    \frac{\num{132,00} - \num{128,98}}{\num{128,98}} \times{} 100 =
    \SI{+2,3414}{\percent}
  \end{equation*}
  mentre l'\emph{Utile in valuta} risulterebbe:
  \begin{equation*}
    16 \times{} (\num{132,00} - \num{128,98}) = \Eur{48,32}
  \end{equation*}
\item  se  l'ultima  quotazione  rilevata  per  il  prezzo  di  una  quota  fosse  di
  \Eur{120,00}, la  \emph{Perdita percentuale} mostrata nella  tabella del patrimonio
  risulterebbe:
  \begin{equation*}
    \frac{\num{120,00} - \num{128,98}}{\num{128,98}} \times{} 100 =
    \SI{-6.9623}{\percent}
  \end{equation*}
  mentre la \emph{Perdita in valuta} risulterebbe:
  \begin{equation*}
    16 \times{} (\num{120,00} - \num{128,98}) = \Eur{-143.68}
  \end{equation*}
\end{itemize}

%page
\section{Strategia per la scelta dei prezzi di vendita}


In generale  si cerca di  vendere a  un prezzo maggiore  del prezzo di  acquisto.  Se
occorre  liquidità: si  potrebbe essere  costretti a  disinvestire vendendo  a prezzi
inferiori.

%page
\subsection{Accoppiamento diretto tra acquisti e vendite}

Si cerca  di vendere  ogni singola quota  a un \emph{prezzo  medio netto  di vendita}
maggiore del suo \emph{prezzo medio di carico di acquisto}.  Per esempio, si supponga
di eseguire le operazioni:
\begin{enumerate}
\item acquisto di \num{20} quote al \emph{prezzo  medio di carico di acquisto} pari a
  \Eur{22,00};
\item acquisto di \num{30} quote al \emph{prezzo  medio di carico di acquisto} pari a
  \Eur{33,00};
\item acquisto di \num{40} quote al \emph{prezzo  medio di carico di acquisto} pari a
  \Eur{44,00};
\item vendita  di \num{20}  quote a un  \emph{prezzo medio netto  di vendita}  pari a
  \Eur{28,00};
\item vendita  di \num{30}  quote a un  \emph{prezzo medio netto  di vendita}  pari a
  \Eur{38,00};
\item vendita  di \num{40}  quote a un  \emph{prezzo medio netto  di vendita}  pari a
  \Eur{48,00};
\end{enumerate}
per ogni gruppo di quote acquistate e poi vendute si realizzano i guadagni:
\begin{itemize}
\item       accoppiando       le       operazioni      \num{1}       e       \num{4}:
  \(\num{20} \times (\num{28,00} - \num{22,00}) = \Eur{120,00}\)
\item       accoppiando       le       operazioni      \num{2}       e       \num{5}:
  \(\num{30} \times (\num{38,00} - \num{33,00}) = \Eur{150,00}\);
\item       accoppiando       le       operazioni      \num{3}       e       \num{6}:
  \(\num{40} \times (\num{48,00} - \num{44,00}) = \Eur{160,00}\);
\end{itemize}
in  cui i  risultati  rappresentano uscite  dal  Conto Titoli  ed  entrate nel  Conto
Corrente.  Quindi il rendimento totale in valuta risulta:
\begin{align*}
  \num{120,00} + \num{150,00} + \num{160,00} = \Eur{430,00}
\end{align*}
e in percentuale:
\begin{align*}
  100 \times{} \frac{\num{430,00}}
  {20 \times{} \num{22,00} + 30 \times{} \num{33,00} + \num{40} \times{} \num{44,00}}
  = \SI{13,48}{\percent}
\end{align*}
Siccome  si sono  eseguiti  prima tutti  gli  acquisti  e poi  tutte  le vendite,  il
\emph{prezzo medio di carico nel saldo} risulta:
\begin{equation*}
  \frac{20 \times{} \num{22,00} + 30 \times{} \num{33,00} + 40 \times{} \num{44,00}}
  {20 + 30 + 40} = \Eur{35,44}
\end{equation*}
la media ponderata dei \emph{prezzi medi netti di vendita} risulta:
\begin{equation*}
  \frac{20 \times{} \num{28,00} + 30 \times{} \num{38,00} + 40 \times{} \num{48,00}}
  {20 + 30 + 40} = \Eur{40,22}
\end{equation*}
quindi il rendimento totale percentuale si può calcolare anche come:
\begin{align*}
  100 \times{} \frac{\num{40,22} - \num{35,44}}{35,44} = \SI{13,48}{\percent}
\end{align*}
e in valuta:
\begin{align*}
  (20 + 30 + 40) \times{} (\num{40,22} - \num{35,44}) = \Eur{430,20}
\end{align*}

%page
\subsection{Metodo delle medie ponderate}

Si cerca  di vendere le quote  a un prezzo  talmente maggiore del prezzo  di acquisto
che, recuperati i costi e pagate le tasse, si realizza un guadagno.
\begin{itemize}
\item Se il \emph{prezzo medio netto di vendita} è maggiore del \emph{prezzo medio di
     carico nel saldo} in  quel momento: si recuperano i costi  di acquisto e vendita
  e, dopo aver pagato le tasse, si realizza un guadagno.

\item Se il  \emph{prezzo medio netto di  vendita} è uguale al  \emph{prezzo medio di
     carico nel saldo}: l'operazione di vendita recupera i costi ma, dopo aver pagato
  le tasse, resta nessun guadagno.

\item Se il \emph{prezzo  medio netto di vendita} è minore  del \emph{prezzo medio di
     carico nel saldo}: tolti i costi  ed eventualmente le tasse, l'operazione genera
  una perdita.
\end{itemize}

Come esempio  di guadagno,  si posseggano  \num{100} quote  al \emph{prezzo  medio di
   carico} di \Eur{88,00}; si vendano  \num{50} quote al \emph{prezzo medio eseguito}
di \Eur{95,00}; risulta:
\begin{itemize}
\item \emph{controvalore dell'operazione}: \(\num{50} \times{} \num{95,00} = \Eur{4750}\);
\item \emph{costo dell'operazione di vendita}: \(\num{0,50} + \num{2,50} + \num{0,0024} \times{} \num{4750} = \Eur{14,40}\);
\item \emph{tasse sul reddito da capitale}: \(\num{0,26} \times{} \num{50} \times{} (\num{95,00} - \num{88,00}) = \Eur{91,00}\);
\item \emph{controvalore totale di vendita}: \(\num{4750} - \num{91,00} - \num{14,40} = \Eur{4644,60}\);
\item \emph{prezzo medio netto di vendita}: \(\num{4644,60} / \num{50} = \Eur{92,89}\);
\end{itemize}
quindi un rendimento percentuale di:
\begin{equation*}
  100 \times{} \frac{\num{92,89} - \num{88,00}}{\num{88,00}} = \SI{+5,56}{\percent}
\end{equation*}

Come  esempio di  perdita, si  posseggano \num{100}  quote al  \emph{prezzo medio  di
   carico nel saldo} di \Eur{88,00}; si  vendano \num{50} quote al \emph{prezzo medio
   eseguito} di \Eur{88,20}; risulta:
\begin{itemize}
\item \emph{controvalore dell'operazione}: \(\num{50} \times{} \num{88,20} = \Eur{4410,00}\);
\item \emph{costo dell'operazione di vendita}: \(\num{0,50} + \num{2,50} + \num{0,0024} \times{} \num{4410,00} = \Eur{13,58}\);
\item \emph{tasse sul reddito da capitale}: \(\num{0,26} \times{} \num{50} \times{} (\num{88,20} - \num{88,00}) = \Eur{2,60}\);
\item \emph{controvalore totale di vendita}: \(\num{4410,00} - \num{2,60} - \num{13,58} = \Eur{4393,82}\);
\item \emph{prezzo medio netto di vendita}: \(\num{4393,82} / \num{50} = \Eur{87,88}\);
\end{itemize}
quindi una perdita percentuale di:
\begin{equation*}
  100 \times{} \frac{\num{87,88} - \num{88,00}}{\num{88,00}} = \SI{-0.1364}{\percent}
\end{equation*}
nonostante il  \emph{prezzo medio  eseguito} sia maggiore  del \emph{prezzo  medio di
   carico nel  saldo}: l'operazione può  andare in  perdita se il  \emph{prezzo medio
   netto di vendita} risulta minore.

Se si  considerano operazioni di vendita  che liquidano tutte le  quote possedute: si
può cercare di recuperare una vendita in perdita con altre in guadagno.  Per esempio,
si supponga di eseguire le operazioni:
\begin{enumerate}
\item acquisto di \num{20} quote al \emph{prezzo  medio di carico di acquisto} pari a
  \Eur{22,00};
\item acquisto di \num{30} quote al \emph{prezzo  medio di carico di acquisto} pari a
  \Eur{33,00};
\item acquisto di \num{40} quote al \emph{prezzo  medio di carico di acquisto} pari a
  \Eur{44,00};
\item vendita  di \num{40}  quote a un  \emph{prezzo medio netto  di vendita}  pari a
  \Eur{33,00};
\item vendita  di \num{50}  quote a un  \emph{prezzo medio netto  di vendita}  pari a
  \Eur{38,50};
\end{enumerate}
avendo eseguito  prima tutti  gli acquisti  e poi tutte  le vendite,  il \emph{prezzo
   medio di carico nel saldo} si calcola con la media ponderata:
\begin{equation*}
  \frac{\num{20} \times{} \num{22,00}
     + \num{30} \times{} \num{33,00}
     + \num{40} \times{} \num{44,00}}
  {\num{20} + \num{30} + \num{40}} = \Eur{35,44}
\end{equation*}
l'operazione  di vendita  al prezzo  di \Eur{33,00}  è in  perdita, mentre  quella al
prezzo di \Eur{38,50}  è in guadagno; la media ponderata  dei \emph{prezzi medi netti
   di vendita} risulta:
\begin{equation*}
  \frac{\num{40} \times{} \num{33,00} + \num{50} \times{} \num{38,50}}
  {40 + 50}
  = \Eur{36,06}
\end{equation*}
maggiore del \emph{prezzo medio di carico nel saldo} e quindi, alla fine, si realizza
un guadagno di:
\begin{equation*}
  \num{90}  \times{} (\num{36,06}  -  \num{35,44}) =  \Eur{55,80}
\end{equation*}

La  domanda  fondamentale  è:  \textbf{possedendo  un numero  di  quote  a  un  certo
   \emph{prezzo medio di carico nel saldo},  qual'è il \emph{prezzo medio eseguito di
      vendita} che permette di andare in guadagno?}

Come esempio, si posseggano \num{50} quote al \emph{prezzo medio di carico nel saldo}
di \Eur{88,00}; si  vuole determinare il \emph{prezzo medio eseguito  di vendita} per
tutte le quote che permette il guadagno.  Detto \(X\) tale valore, risulta:
\begin{itemize}
\item \emph{controvalore dell'operazione}: \(\num{50} \times{} X\);
\item \emph{costo dell'operazione di vendita}:
  \begin{align*}
    \num{0,50} + \num{2,50} + \num{0,0024} \times{} (\num{50} \times{} X)
    &= \num{3,00} + \num{0,0024} \times{} \num{50} \times{} X = \\
    &= \num{3,00} + \num{0,12} \times{} X
  \end{align*}
\item \emph{tasse sul reddito da capitale}:
  \begin{align*}
    \num{0,26} \times{} \num{50} \times{} (X - \num{88,00})
    &= \num{13} \times{} X - \num{13} \times{} \num{88,00} = \\
    &= \num{13} \times{} X - \num{1144,00}
  \end{align*}
\item \emph{controvalore totale di vendita}:
  \begin{align*}
    (\num{50} \times{} X)
    &- (\num{3,00} + \num{0,12} \times{} X)
      - (\num{13} \times{} X - \num{1144,00})
      = \\
    &= \num{50} \times{} X
      - \num{3,00} - \num{0,12} \times{} X
      - \num{13} \times{} X + \num{1144,00}
      = \\
    &= (\num{50} - \num{0,12} - \num{13}) \times{} X + (\num{1144,00} - \num{3,00})
      = \\
    &= \num{36,88} \times{} X + \num{1141,00}
  \end{align*}
\item \emph{prezzo medio netto di vendita}:
  \begin{equation*}
    \frac{\num{36,88} \times{} X + \num{1141,00}}{50}
  \end{equation*}
\end{itemize}
il pareggio si raggiungerebbe vendendo le quote a un \emph{prezzo medio netto} uguale
al \emph{prezzo medio di carico nel saldo} di \Eur{88,00}; quindi:
\begin{equation*}
  \frac{\num{36,88} \times{} X + \num{1141,00}}{50} = \num{88,00}
\end{equation*}
risolvendo rispetto a \(X\):
\begin{equation*}
  X = \frac{\num{50} \times{} \num{88,00} - \num{1141,00}}{\num{36,88}}
  = \Eur{88,37}
\end{equation*}
vendendo con un  \emph{prezzo medio eseguito} maggiore di \Eur{88,37}  si realizza un
guadagno.

%page
%% ------------------------------------------------------------
%% Fine.
%% ------------------------------------------------------------

\include{fdl-1.3}

%page
%% ------------------------------------------------------------
%% Fine.
%% ------------------------------------------------------------

\end{document}

%%% end of file
% Local Variables:
% mode: latex
% page-delimiter: "^%page"
% TeX-master: t
% ispell-local-dictionary: "italiano"
% fill-column: 85
% End:

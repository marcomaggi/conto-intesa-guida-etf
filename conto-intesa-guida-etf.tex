% conto-intesa-guida-etf.tex -*- coding: utf-8-unix -*-

\documentclass[12pt,a4paper]{article}
\renewcommand{\rmdefault}{ptm} % Times Roman font, for PDF output
\usepackage[italian]{babel}
\usepackage[utf8]{inputenc}
\usepackage[T1]{fontenc}
\usepackage[]{textcomp}
\usepackage[]{eurosym}
\usepackage[]{amsmath}
\usepackage[]{amsfonts}
\usepackage[]{amsthm}
\usepackage[]{amssymb}
\usepackage[]{syntonly}
\usepackage[copy-decimal-marker,retain-explicit-plus]{siunitx}
\usepackage[hidelinks]{hyperref} % To generate PDFs with hyperlinks

\pagestyle{headings}

%page
%% ------------------------------------------------------------
%% Comandi.
%% ------------------------------------------------------------

\newcommand{\Undefine}[1]{\let#1\CustomUndefined}
\newcommand{\Define}[2]{%
\Undefine{#1}
\newcommand{#1}{#2}
}

\newcommand{\Perc}[1]{\SI{#1}{\percent}}
\newcommand{\Eur}[1]{\SI{#1}{\text{\euro{}}}}

\newcommand{\MediaPonderataDue}[4]{\frac{\num{#1} \times{} \num{#2} + \num{#3} \times{} \num{#4}}{\num{#1} + \num{#3}}}
\newcommand{\MediaPonderataTre}[6]{\frac{\num{#1} \times{} \num{#2} + \num{#3} \times{} \num{#4} + \num{#5} \times{} \num{#6}}{\num{#1} + \num{#3} + \num{#5}}}

\newcommand{\CalcoloCostoOperazione}[1]{\num{0,50} + \num{2,50} + \num{0,0024} \times{} \num{#1}}

\newcommand{\CalcoloRendimentoPercentuale}[2]{\frac{\num{#1} - \num{#2}}{#2} \times{} \num{100}}

\newcommand{\Parentesi}[1]{(#1)}
\newcommand{\Virgolette}[1]{``#1''}
\newcommand{\Etf}[1]{\textsc{etf}}

\newcommand{\Pma}{\emph{prezzo medio di acquisto}}

\newcommand{\PrezzoMedioEseguitoAcquisto}{\emph{prezzo medio eseguito di acquisto}}
\newcommand{\PrezzoMedioEseguitoVendita}{\emph{prezzo medio eseguito di vendita}}

\newcommand{\PrezzoMedioEffettivoAcquisto}{\emph{prezzo medio effettivo di acquisto}}
\newcommand{\PrezzoMedioEffettivoVendita}{\emph{prezzo medio effettivo di vendita}}

\newcommand{\PrezzoMedioCaricoAcquisto}{\emph{prezzo medio di carico di acquisto}}
\newcommand{\PrezzoMedioEffettivoNelSaldo}{\emph{prezzo medio effettivo nel saldo}}
\newcommand{\PrezzoMedioCaricoSaldo}{\emph{prezzo medio di carico nel saldo}}

\newcommand{\PrezzoMedioNettoVendita}{\emph{prezzo medio netto di vendita}}

\newcommand{\ControvaloreTotaleVendita}{\emph{controvalore totale di vendita}}
\newcommand{\ControvaloreTotaleAcquisto}{\emph{controvalore totale di acquisto}}

\newcommand{\RedditoCapitale}{\emph{reddito da capitale}}
\newcommand{\RedditoDiverso}{\emph{reddito diverso}}
\newcommand{\TasseRedditoCapitale}{\emph{tasse sul reddito da capitale}}
\newcommand{\UtilePercentuale}{\emph{utile percentuale}}
\newcommand{\UtileValuta}{\emph{utile in valuta}}
\newcommand{\PerditaPercentuale}{\emph{perdita percentuale}}
\newcommand{\PerditaValuta}{\emph{perdita in valuta}}
\newcommand{\RendimentoPercentuale}{\emph{rendimento percentuale}}
\newcommand{\RendimentoValuta}{\emph{rendimento in valuta}}

%% ------------------------------------------------------------------------

% Prezzo medio di effettivo, prezzo medio eseguito
\newcommand{\Pme}[1]{P_{\mathrm{me}#1}}

% Prezzo medio effettivo di acquisto
\newcommand{\Pmea}[1]{\Pme{\mathrm{a}#1}}

% Prezzo medio effettivo di vendita
\newcommand{\Pmev}[1]{\Pme{\mathrm{v}#1}}

% Prezzo medio effettivo nel saldo
\newcommand{\Pmes}[1]{\Pme{\mathrm{s}#1}}

% Prezzo medio di carico
\newcommand{\Pmc}[1]{P_{\mathrm{mc}#1}}

% Prezzo medio di carico di acquisto
\newcommand{\Pmca}[1]{\Pmc{\mathrm{a}#1}}

% Prezzo medio di carico nel saldo
\newcommand{\Pmcs}[1]{\Pmc{\mathrm{s}#1}}

%page
%% ------------------------------------------------------------
%% Intestazione
%% ------------------------------------------------------------

\author{Marco Maggi}
\title{Pericolosa e incompleta guida agli investimenti in \Etf{} con \emph{Banca Intesa Sanpaolo}}

\begin{document}

\maketitle

\begin{abstract}
  \noindent
  Questa guida è un  aiuto per chi volesse ricostruire i calcoli di  contabilità del Conto Titoli di
  \emph{Banca   Intesa   Sanpaolo}   per   gli  investimenti   in   \emph{Exchange   Traded   Funds}
  \Parentesi{\Etf{}}.   Questa guida  è incompleta:  non copre  tutto ciò  che c'è  da sapere  sugli
  \Etf{}, né l'operatività del sito di  \emph{home banking} di \emph{Banca Intesa Sanpaolo}.  Questa
  guida è pericolosa: ognuno la usa a suo rischio e pericolo.

  In questa guida si tenta di mantenere  il testo matematicamente ``facile''; si evita l'utilizzo di
  simboli matematici nelle  formule; chi ha frequentato  le Scuole Medie Inferiori, e  ne ricorda le
  basi del programma di matematica, dovrebbe farcela.
\end{abstract}

\tableofcontents

\newpage{}

\noindent
Copyright \copyright{} 2017, 2018, 2020 Marco Maggi \texttt{<mrc.mgg@gmail.com>}.

Permission is  granted to copy, distribute  and/or modify this document  under the terms of  the GNU
Free  Documentation License,  Version  1.3 or  any  later  version published  by  the Free  Software
Foundation; with no  Invariant Sections, no Front-Cover  Texts, and no Back-Cover Texts.   A copy of
the license is included in the section entitled ``GNU Free Documentation License''.

Il codice sorgente di questa guida è disponibile in Rete all'indirizzo:
\begin{center}
  \url{https://github.com/marcomaggi/conto-intesa-guida-etf}
\end{center}

\newpage{}

%page
\section{Introduzione}


Il calcolo  dei redditi da  capitale e dei  redditi diversi per  investimenti in \Etf{}  deve essere
eseguito come specificato nella ``Circolare dell'Agenzia dell'Entrate 21/E del 10 Luglio 2014''.  Il
calcolo di questi redditi permette sia la determinazione dei rendimenti che il calcolo delle tasse.

Per ogni operazione di compravendita eseguita sul  Conto Titoli: nella sezione documenti del sito di
\emph{home banking} è disponibile una \emph{nota di eseguito} con tutti i dati dell'operazione.  Nei
calcoli qui  illustrati: alcuni valori  riportati nelle  note di eseguito  sono i dati  di partenza;
altri valori sono i risultati ricalcolati per verifica.

Questa  guida utilizza  sia la  terminologia  della Banca  che quella  della circolare  dell'Agenzia
dell'Entrate.   Da notare  che i  prezzi di  acquisto  e vendita  sulle Borse  di negoziazione  sono
chiamati \emph{effettivi} dalla circolare.

Tutti i numeri in questa guida sono presentati arrotondati, il piú delle volte al centesimo di Euro;
l'esecuzione dei calcoli è invece eseguita con precisione maggiore.

%page
\section{Media aritmetica e media ponderata}


Nei calcoli di contabilità  Conto Titoli si usa spesso l'operazione di  \emph{media aritmetica} e in
particolare la sua riscrittura come \emph{media ponderata};  è utile richiamare queste idee e alcune
delle loro proprietà.

Si ricorda che la \textbf{media aritmetica} tra i numeri \num{11}, \num{22} e \num{33} si scrive:
\begin{equation*}
  \frac{\num{11} + \num{22} + \num{33}}{3} = \num{22}
\end{equation*}
il denominatore è \num{3} perché al numeratore  ci sono \num{3} numeri; la \emph{media aritmetica} è
una procedura di calcolo il cui risultato è il \emph{valore medio}.

L'operazione si costruisce  nello stesso modo se qualche  numero compare piú di una  volta; la media
aritmetica tra i numeri: \num{11}, \num{11}, \num{11}, \num{22}, \num{33} e \num{33} si scrive:
\begin{equation*}
  \frac{\num{11} + \num{11} + \num{11} + \num{22} + \num{33} + \num{33}}{6}
  = \frac{121}{6} \simeq \num{20,17}
\end{equation*}
il denominatore è \num{6} perché:
\begin{itemize}
\item \num{11} compare \num{3} volte;
\item \num{22} compare \num{1} volta;
\item \num{33} compare \num{2} volte;
\end{itemize}
quindi \(\num{3} + \num{1} + \num{2} = \num{6}\).

Il \Virgolette{numero di volte in cui un  numero compare} si chiama \textbf{molteplicità}; allora si
può dire che:
\begin{itemize}
\item \num{11} compare con \emph{molteplicità} \num{3};
\item \num{22} compare con \emph{molteplicità} \num{1};
\item \num{33} compare con \emph{molteplicità} \num{2};
\end{itemize}
evidenziando le molteplicità e considerando le semplici identità:
\begin{align*}
  \num{11} + \num{11} + \num{11} &= \num{3} \times{} \num{11} &&&
  \num{22} &= \num{1} \times{} \num{22} &&&
  \num{33} + \num{33} &= \num{2} \times{} \num{33}
\end{align*}
l'espressione della \emph{media aritmetica} si può riscrivere:
\begin{align*}
  &\frac{\num{11} + \num{11} + \num{11} + \num{22} + \num{33} + \num{33}}{6}
  = \frac{\num{3} \times{} \num{11}
    + \num{1} \times{} \num{22}
    + \num{2} \times{} \num{33}}
    {6} = \\
  &=
    \frac{\num{3} \times{} \num{11}}{6} +
    \frac{\num{1} \times{} \num{22}}{6} +
    \frac{\num{2} \times{} \num{33}}{6}
  =
    \frac{\num{3}}{6} \times{} \num{11} +
    \frac{\num{1}}{6} \times{} \num{22} +
    \frac{\num{2}}{6} \times{} \num{33}
\end{align*}
in cui si evidenzia come:
\begin{itemize}
\item il numero \num{11} compaia con ``peso'' \num{3} rispetto al totale \num{6};
\item il numero \num{22} compaia con ``peso'' \num{1} rispetto al totale \num{6};
\item il numero \num{33} compaia con ``peso'' \num{2} rispetto al totale \num{6}.
\end{itemize}

In questo  modo: la  \emph{media aritmetica}  tra i numeri  \num{11}, \num{11},  \num{11}, \num{22},
\num{33} e \num{33} è riscritta come media  \textbf{media ponderata} tra i numeri \num{11}, \num{22}
e \num{33} rispetto alle loro molteplicità \num{3}, \num{1} e \num{2}.

Il significato del \emph{valore medio} si può evidenziare osservando che se vale l'identità:
\begin{align*}
  \frac{\num{3} \times{} \num{11} + \num{1} \times{} \num{22}
  + \num{2} \times{} \num{33}}{\num{6}}
  \simeq \num{20,17}
\end{align*}
allora vale anche l'identità:
\begin{align*}
  \frac{\num{3} \times{} \num{11} + \num{1} \times{} \num{22}
  + \num{2} \times{} \num{33}}{\num{6}}
  \simeq \frac{\num{6} \times{} \num{20,17}}{\num{6}}
\end{align*}
perciò la \emph{media  ponderata} tra: \num{11} con molteplicità \num{3},  \num{22} con molteplicità
\num{1},  \num{33}  con  molteplicità  \num{2},   è  equivalente  alla  \emph{media  ponderata}  del
\emph{valore medio} \num{20,17} con molteplicità \num{6}.

%page
\subsection{Interpretazione della media ponderata come prezzo medio}


Nei calcoli di  contabilità del Conto Titoli: si  eseguono medie ponderate tra i  prezzi di acquisto
\Parentesi{\textbf{non}  di vendita}  di quote  di uno  stesso \Etf{}  rispetto al  numero di  quote
acquistate.

Negli esempi precedenti:  i numeri \num{11}, \num{22}, \num{33} si  possono interpretare come prezzi
di acquisto  di quote;  le molteplicità \num{3},  \num{1}, \num{2} si  possono interpretare  come il
numero di quote acquistate; cioè:
\begin{itemize}
\item \num{3} quote acquistate al prezzo di \Eur{11,00};
\item \num{1} quota acquistata al prezzo di \Eur{22,00};
\item \num{2} quote acquistate al prezzo di \Eur{33,00}.
\end{itemize}

Affermare che  la \emph{media ponderata}  dei prezzi è pari  a \Eur{20,17} significa  affermare che:
acquistare  \num{3} quote  di un  fondo al  prezzo di  \Eur{11,00}, poi  acquistare \num{1}  quota a
\Eur{22,00}, infine acquistare \num{2} quote a \Eur{33,00} è equivalente ad acquistare \num{6} quote
al \textbf{prezzo medio di acquisto} di \Eur{20,17}.

Il  numero  totale   di  quote  acquistate  \(\num{6}   =  \num{3}  +  \num{1}  +   \num{2}\)  è  la
\emph{molteplicità equivalente} del \Pma{} \Eur{20,17}, infatti si può scrivere:
\begin{align*}
  \frac{\num{3} \times{} \num{11} + \num{1} \times{} \num{22} + \num{2} \times{} \num{33}}{\num{6}}
  \simeq \frac{\num{6} \times{} \num{20,17}}{\num{6}}
  = \num{20,17}
\end{align*}

%page
\subsection{Calcolo incrementale della media ponderata}


L'attività di investimento in  un \Etf{} \Parentesi{cosí come in altri  titoli finanziari quotati} è
costituita da  una sequenza  di operazioni  di acquisto  e vendita  di quote  del fondo;  acquisti e
vendite possono  essere intercalati \Parentesi{prima un  acquisto, poi una vendita  parziale, poi un
   altro acquisto, eccetera}.  Dopo ogni acquisto: le nuove quote vengono messe insieme alle vecchie
già possedute.  Invece di conservare tutta la storia  della sequenza di acquisti e vendite: si tiene
la contabilità conservando solo il numero di quote attualmente possedute e il loro \Pma{}.

È utile richiamare  come il calcolo di  una \emph{media ponderata} possa essere  eseguito passo dopo
passo.

Si considerino  ancora i numeri  \num{11}, \num{11}, \num{11}, \num{22},  \num{33} e \num{33};  si è
calcolato che  la loro \emph{media  aritmetica} può essere  riscritta come \emph{media  ponderata} e
risulta:
\begin{equation*}
  \frac{\num{11} + \num{11} + \num{11} + \num{22} + \num{33} + \num{33}}{6}
  = \frac{\num{3} \times{} \num{11}
     + \num{1} \times{} \num{22}
     + \num{2} \times{} \num{33}}
  {6} \simeq \num{20,17}
\end{equation*}
Si  esegua lo  stesso calcolo  in piú  passi: si  considerino prima  i numeri  \num{11}, \num{22}  e
\num{33}; poi si aggiungano i numeri \num{11} e \num{11}; infine si aggiunga il numero \num{33}.
\begin{enumerate}
\item  La  \emph{media  ponderata}  tra  i  numeri  \num{11},  \num{22}  e  \num{33},  ciascuno  con
  molteplicità \num{1}, risulta:
  \begin{equation*}
    \frac{\num{1} \times{} \num{11}
       + \num{1} \times{} \num{22}
       + \num{1} \times{} \num{33}}{\num{3}}
    = \num{22}
  \end{equation*}
  il numero  totale di  numeri \(\num{3}  = \num{1} +  \num{1} +  \num{1}\) è  la \emph{molteplicità
     equivalente} del \emph{valore medio} \num{22}.

\item Ora si aggiungano  i numeri \num{11} e \num{11}; in totale:  \num{11} compare con molteplicità
  \num{3}; \num{22} compare con molteplicità \num{1}; \num{33} compare con molteplicità \num{1}.  La
  \emph{media ponderata aggiornata} risulta:
  \begin{equation*}
    \frac{\num{3} \times{} \num{11}
       + \num{1} \times{} \num{22}
       + \num{1} \times{} \num{33}}{\num{5}}
    = \frac{\num{88}}{\num{5}} \simeq \num{17,6}
  \end{equation*}
  il numero  totale di  numeri \(\num{5}  = \num{3} +  \num{1} +  \num{1}\) è  la \emph{molteplicità
     equivalente} del \emph{valore medio} \num{17,6}.

  È possibile calcolare  questo risultato partendo dalla media ponderata  del passo precedente, cioè
  \num{22}, combinandola con  in numeri \num{11} e  \num{11}?  La media ponderata  aggiornata si può
  scrivere:
  \begin{align*}
    \frac{\num{3} \times{} \num{11}
    + \num{1} \times{} \num{22}
    + \num{1} \times{} \num{33}}{\num{5}}
    = \frac{\num{2} \times{} \num{11}}{\num{5}}
    + \frac{\num{1} \times{} \num{11}
    + \num{1} \times{} \num{22}
    + \num{1} \times{} \num{33}}{\num{5}}
  \end{align*}
  la seconda frazione al membro di destra \Parentesi{il cui numeratore è uguale a quello della media
     ponderata al passo precedente} si può riscrivere:
  \begin{align*}
    \frac{\num{3}}{\num{5}} \times \frac{\num{5}}{\num{3}} \times
    \frac{\num{1} \times{} \num{11}
    + \num{1} \times{} \num{22}
    + \num{1} \times{} \num{33}}{\num{5}}
    &= \frac{\num{3}}{\num{5}} \times
      \frac{\num{1} \times{} \num{11}
      + \num{1} \times{} \num{22}
      + \num{1} \times{} \num{33}}{\num{3}}
  \end{align*}
  in cui il secondo termine della moltiplicazione al membro di destra è la media ponderata del passo
  precedente; quindi la \emph{media ponderata} aggiornata risulta:
  \begin{align*}
    \frac{\num{2} \times{} \num{11}}{\num{5}}
    + \frac{\num{3} \times{} \num{22}}{\num{5}}
    = \frac{\num{2} \times{} \num{11} + \num{3} \times{} \num{22}}{\num{5}}
    \simeq \num{17,6}
  \end{align*}
  cioè il \emph{valore medio} aggiornato \num{17,6} è pari alla \emph{media ponderata} tra:
  \begin{itemize}
  \item  il \emph{valore  medio} precedente  \num{22},  con la  sua \emph{molteplicità  equivalente}
    \num{3};
  \item il nuovo numero aggiunto \num{11}, con la sua molteplicità \num{2}.
  \end{itemize}

\item  Infine si  aggiunga il  numero  \num{33}; la  nuova  \emph{media ponderata}  aggiornata è  la
  \emph{media ponderata} totale e procedendo come prima risulta:
  \begin{equation*}
    \frac{\num{1} \times{} \num{33}
       + \num{5} \times{} \num{17,6}}{\num{6}}
    \simeq \num{20,17}
  \end{equation*}
\end{enumerate}

%page
\subsection{Intepretazione del calcolo incrementale della media ponderata}

Nei calcoli di contabilità del Conto Titoli: dopo ogni  acquisto di quote di un fondo, si calcola il
\Pma{} aggiornato, di ogni quota posseduta, come \emph{media ponderata} tra:
\begin{itemize}
\item il \Pma{} precedente, con il precedente numero di quote possedute come molteplicità;
\item il prezzo delle nuove quote acquistate, con il numero di quote acquistate come molteplicità.
\end{itemize}

Come si aggiorna il \Pma{} se si vendono  delle quote?  Esso resta invariato!  Si considerino ancora
i numeri  \num{11}, \num{11}, \num{11}, \num{22},  \num{33} e \num{33};  si è calcolato che  la loro
\emph{media ponderata} risulta:
\begin{equation*}
  \frac{\num{11} + \num{11} + \num{11} + \num{22} + \num{33} + \num{33}}{6}
  = \frac{\num{3} \times{} \num{11}
     + \num{1} \times{} \num{22}
     + \num{2} \times{} \num{33}}
  {6} \simeq \num{20,17}
\end{equation*}
in cui \num{20,17}  può essere interpretato come  \Pma{} di \num{6} quote.  Si  supponga di togliere
\num{2} quote  vendendole al prezzo  di \num{44}:  il \Pma{} delle  restanti \num{4} quote  è ancora
\num{20,17}.

%page
\section{Direzione dei trasferimenti di denaro}


Con abusiva semplificazione, le somme di denaro descritte in questa guida si ritengono scambiate tra
il Conto Corrente e il  Conto Titoli.  In realtà il Conto Titoli è  solo uno strumento contabile: un
registro in cui sono  annotate le quote di titoli finanziari che abbiamo  comprato e quindi sono nel
nostro \Virgolette{portafoglio  finanziario}.  Invece, in questa  guida, il Conto Titoli  è trattato
come un \Virgolette{contenitore} in cui vanno a finire i soldi che spendiamo per acquistare quote di
titoli  finanziari; questo  contenitore  \textbf{non}  conserva i  soldi:  quando estraiamo  denaro,
vendendo quote,  la quantità estratta  può essere  uguale, minore o  maggiore di quella  che abbiano
immesso al momento dell'acquisto.

Con questo sistema di idee:
\begin{itemize}
\item le somme di denaro \emph{investite} in quote di fondi sono in \emph{uscita dal Conto Corrente}
  e in \emph{entrata nel Conto Titoli};
\item le  somme di  denaro \emph{disinvestite}  da quote di  fondi sono  in \emph{entrata  nel Conto
     Corrente} e in \emph{uscita dal Conto Titoli}.
\end{itemize}

Per esempio,  si supponga di acquistare  quote di un  fondo per \Eur{1000,00} tutto  incluso; questa
quantità di denaro esce  dal Conto Corrente e viene utilizzata per comprare  le quote, le quote sono
\Virgolette{messe in  carico} nel Conto  Titoli con  valore pari al  loro prezzo totale  di acquisto
\Eur{1000,00}.  Si considerino i due casi:
\begin{enumerate}
\item si vendano le quote al prezzo netto di \Eur{1200,00}; questa quantità di denaro esce dal Conto
  Titoli ed entra nel Conto Corrente;
  \begin{itemize}
  \item  dal punto  di vista  del  Conto Titoli  \Eur{1000,00}  entrano e  \Eur{1200,00} escono,  il
    bilancio risulta:
    \begin{equation*}
      + \num{1000,00} - \num{1200,00} = \Eur{+200,00}
    \end{equation*}
  \item dal  punto di  vista del  Conto Corrente  \Eur{1000,00} escono  e \Eur{1200,00}  entrano, il
    bilancio risulta:
    \begin{equation*}
      - \num{1000,00} + \num{1200,00} = \Eur{+200,00}
    \end{equation*}
  \end{itemize}


  il bilancio delle operazioni,  dal punto di vista  del Conto
  Corrente, si esprime con la differenza algebrica:
  \begin{equation*}
    \num{1200,00} - \num{1000,00} = \Eur{+200,00}
  \end{equation*}
  cioè, nel saldo finale, \Eur{1000,00} inizialmente usciti rientrano

\Eur{200,00} entrano \Virgolette{appaiono} nel Conto Titoli, vi escono ed entrano
  nel Conto Corrente insieme ai \num{1000,00} iniziali: un guadagno;
\item si vendano le quote al prezzo netto  di \Eur{900,00}; questa quantità di denaro esce dal Conto
  Titoli ed  entra nel Conto  Corrente; il  bilancio delle operazioni  si esprime con  la differenza
  algebrica:
  \begin{equation*}
    \num{900,00} - \num{1000,00} = \Eur{-100,00}
  \end{equation*}
  cioè, nel saldo finale, \Eur{100,00} escono dal Conto  Corrente ed entrano nel Conto Titoli in cui
  poi \Virgolette{spariscono}: una perdita;
\end{enumerate}
le differenze algebriche hanno due interpretazioni:
\begin{itemize}
\item \emph{uscite dal Conto Titoli} meno \emph{entrate nel Conto Titoli};
\item \emph{entrate nel Conto Corrente} meno \emph{uscite dal Conto Corrente}.
\end{itemize}

%page
\section{Descrizione di un'operazione di acquisto}


\newcommand{\OneNumeroQuote}{100}
\newcommand{\OnePrezzoMedioEseguito}{53,80}
\newcommand{\OneControvaloreOperazione}{5380,00}
\newcommand{\OneCostoOperazione}{15,91}
\newcommand{\OneControvaloreTotale}{5395,91}
\newcommand{\OnePrezzoMedioCarico}{53,96}


Quando  si  acquistano  quote di  un  titolo  \Parentesi{anche  non  \Etf{}} è  generalmente  meglio
specificare la strategia di acquisto \Virgolette{con  prezzo limite} e selezionare esplicitamente il
mercato su cui si  vogliono eseguire le operazioni, per esempio, nel  caso degli \Etf{}: \emph{Borsa
   Italiana segmento ETFplus}.

Perciò in  una certa data, e  a una certa  ora, si inserisce l'ordine  di acquisto per un  numero di
quote del  fondo a un certo  prezzo massimo: per essere  conforme all'ordine inserito, la  banca che
agisce da intermediario può eseguire l'ordine  acquistando a qualsiasi prezzo \emph{minore o uguale}
al limite fissato.   L'ordine può essere eseguito  in piú fasi in  cui solo una parte  delle quote è
acquistata, in ogni fase  a un prezzo diverso; è possibile che non  tutte le quote siano acquistate,
nel qual caso ci interessa solo la frazione \Virgolette{eseguita} dell'ordine.

Il  \textbf{prezzo medio  eseguito di  un acquisto}  (nella terminologia  della Banca),  anche detto
\textbf{prezzo medio effettivo di un acquisto}  (nella terminologia dell'Agenzia dell'Entrate), è la
media ponderata dei prezzi  di acquisto rispetto al numero di quote acquistate.   Per esempio, se un
singolo ordine di acquisto per \num{\OneNumeroQuote} quote è eseguito nelle tre fasi:
\begin{enumerate}
\item acquisto di \num{20} quote al prezzo di \Eur{52,00};
\item acquisto di \num{30} quote al prezzo di \Eur{53,00};
\item acquisto di \num{50} quote al prezzo di \Eur{55,00};
\end{enumerate}
allora il \emph{prezzo medio eseguito} risulta:
\begin{equation*}
  \MediaPonderataTre{20}{52,00}{30}{53,00}{50}{55,00} = \Eur{\OnePrezzoMedioEseguito{}}
\end{equation*}

Il \textbf{controvalore  dell'operazione di acquisto}  è il prodotto tra  il numero totale  di quote
acquistate e il  \emph{prezzo medio eseguito}.  Per esempio, se  si acquistano \num{\OneNumeroQuote}
quote   al   \emph{prezzo  medio   eseguito}   di   \Eur{\OnePrezzoMedioEseguito}  il   controvalore
dell'operazione risulta:
\begin{equation*}
  \num{\OneNumeroQuote} \times{} \num{\OnePrezzoMedioEseguito{}}
  = \Eur{\OneControvaloreOperazione}
\end{equation*}

Il \textbf{costo dell'operazione  di acquisto} è la  somma tra costi, spese  e commissioni associate
all'operazione; per ogni ordine di acquisto eseguito, occorre pagare:
\begin{itemize}
\item  le commissioni  per  l'intermediario, fissate  a  \Eur{2,50} quale  che  sia il  controvalore
  dell'operazione eseguita  \Parentesi{la Borsa  di negoziazione  su cui  si eseguono  le operazioni
     vuole essere pagata per ogni eseguito};
\item le commissioni fisse per la Banca pari a \Eur{0,50};
\item le  commissioni variabili per  la Banca  pari allo \SI{0,24}{\percent}  del \emph{controvalore
     dell'operazione}.
\end{itemize}
Se un ordine non è  eseguito: si paga nulla.  Se un ordine è eseguito in  piú fasi: nella prima fase
si pagano  i costi  fissi piú  la commissione  percentuale; nelle  fasi successive  si paga  solo la
commissione   percentuale.     Per   esempio,   se   il    \emph{controvalore   dell'operazione}   è
\Eur{\OneControvaloreOperazione} il \emph{costo dell'operazione} risulta:
\begin{equation*}
  \CalcoloCostoOperazione{\OneControvaloreOperazione} = \Eur{\OneCostoOperazione}
\end{equation*}

Il \textbf{controvalore totale di un acquisto}  è la somma tra \emph{controvalore dell'operazione} e
\emph{costo dell'operazione} ed è  la quantità di denaro in uscita dal  Conto Corrente; per esempio,
se  il \emph{controvalore  dell'operazione}  è \Eur{\OneControvaloreOperazione{}}  e il  \emph{costo
   dell'operazione} è pari a \Eur{\OneCostoOperazione}, il \emph{controvalore totale} risulta:
\begin{equation*}
  \num{\OneControvaloreOperazione} + \num{\OneCostoOperazione}
  = \Eur{\OneControvaloreTotale}
\end{equation*}

Il \textbf{prezzo medio di carico di un acquisto}  é il rapporto tra il \emph{controvalore totale} e
il numero di  quote acquistate ed è  usato per l'aggiornamento del  saldo: è il prezzo  medio di una
singola quota acquistata,  tutto incluso.  Per esempio, se si  sono acquistate \num{\OneNumeroQuote}
quote  al  \emph{controvalore totale}  di  \Eur{\OneControvaloreTotale},  il \emph{prezzo  medio  di
   carico} risulta:
\begin{equation*}
  \num{\OneControvaloreTotale} / \num{\OneNumeroQuote} = \Eur{\OnePrezzoMedioCarico}
\end{equation*}

%page
\section{Aggiornamento del saldo dopo un'operazione di acquisto}


Dopo l'esecuzione di ogni operazione di acquisto  occorre ricalcolare il saldo del Conto Titoli; per
un investimento in quote di \Etf{}, la parte da aggiornare è rappresentata dalle tre quantità:
\begin{itemize}
\item numero di quote di quell'\Etf{} in carico nel Conto Titoli;
\item  \textbf{prezzo  medio  \underline{effettivo}  nel  saldo}:  è  il  prezzo  medio,  per  quota
  acquistata, degli ordini eseguiti sulla Borsa di negoziazione; si calcola come media ponderata tra
  il \emph{prezzo  medio \underline{effettivo} di  un nuovo  acquisto} e il  precedente \emph{prezzo
     medio  \underline{effettivo} nel  saldo}; nella  documentazione  della Banca:  questo valore  è
  chiamato \emph{Net  Asset Value} (NAV) del  prezzo medio di carico  (da non confondere con  il NAV
  dell'attività sottostante il fondo);
\item \textbf{prezzo medio di \underline{carico} nel saldo}: è il prezzo medio per quota acquistata,
  tutto incluso; si calcola come media ponderata  tra il \emph{prezzo medio di \underline{carico} di
     un nuovo acquisto} e il precedente \emph{prezzo medio di \underline{carico} nel saldo}.
\end{itemize}

Dai valori nel saldo  si può calcolare il \textbf{costo medio per quota  nel saldo}: si calcola come
differenza  tra il  \emph{prezzo medio  di  \underline{carico} nel  saldo} e  il \emph{prezzo  medio
   \underline{effettivo} nel saldo}.

Per esempio, si supponga di eseguire le operazioni:
\begin{enumerate}
\item acquisto di \num{101} quote al \emph{prezzo medio effettivo} di \Eur{51,00};
\item acquisto di \num{102} quote al \emph{prezzo medio effettivo} di \Eur{52,00};
\item acquisto di \num{103} quote al \emph{prezzo medio effettivo} di \Eur{53,00}.
\end{enumerate}
all'inizio  si consideri  un  Conto Titoli  vuoto,  con  saldo convenzionale  di:  \num{0} quote  di
quell'\Etf{}; \emph{prezzo  medio di carico} di  \Eur{0}; \emph{prezzo medio effettivo}  di \Eur{0}.
L'aggiornamento del saldo avviene come segue.
\begin{enumerate}
\item Per la prima operazione:
  \begin{itemize}
  \item numero di quote acquistate: \num{101};
  \item \emph{prezzo medio effettivo}: \Eur{51,00};
  \item \emph{controvalore dell'operazione}: \Eur{5151,00};
  \item \emph{costo dell'operazione}: \Eur{15,36};
  \item \emph{controvalore totale}: \Eur{5166,36};
  \item \emph{prezzo medio di carico}: \Eur{51,1521}.
  \end{itemize}

  Dopo la prima operazione:
  \begin{itemize}
  \item numero quote nel saldo: \num{101}; uguale al numero quote del primo acquisto;
  \item  \emph{prezzo  medio  effettivo  nel  saldo}:  \Eur{51,00};  uguale  al  \emph{prezzo  medio
       effettivo} del primo acquisto;
  \item \emph{prezzo medio carico nel saldo}: \Eur{51,1521}; uguale al \emph{prezzo medio di carico}
    del primo acquisto;
  \end{itemize}
  il \emph{costo medio per quota nel saldo} risulta:
  \begin{equation*}
    \num{51,1521} - \num{51,00} = \Eur{0,1521}
  \end{equation*}

\item Per la seconda operazione:
  \begin{itemize}
  \item numero di quote acquistate: \num{102};
  \item \emph{prezzo medio effettivo}: \Eur{52,00};
  \item \emph{controvalore dell'operazione}: \Eur{5304,00};
  \item \emph{costo dell'operazione}: \Eur{15,73};
  \item \emph{controvalore totale}: \Eur{5319,73};
  \item \emph{prezzo medio di carico}: \Eur{52,1542}.
  \end{itemize}

  Dopo la seconda operazione:
  \begin{itemize}
  \item numero quote nel saldo: \(101 + 102 = 203\);
  \item  il \emph{prezzo  medio effettivo  nel saldo}  è la  media ponderata  dei \emph{prezzi  medi
       effettivi} del saldo precedente e del nuovo acquisto:
    \begin{equation*}
      \MediaPonderataDue{101}{51,00}{102}{52,00} = \Eur{51,5025}
    \end{equation*}
  \item il  \emph{prezzo medio di carico  nel saldo} è la  media ponderata dei \emph{prezzi  medi di
       carico} del saldo precedente e del nuovo acquisto:
    \begin{equation*}
      \MediaPonderataDue{101}{51,1521}{102}{52,1542} = \Eur{51,6556}
    \end{equation*}
  \end{itemize}
  il \emph{costo medio per quota nel saldo} risulta:
  \begin{equation*}
    \num{51,6556} - \num{51,5025} = \Eur{0,1532}
  \end{equation*}

\item Per la terza operazione:
  \begin{itemize}
  \item numero di quote acquistate: \num{103};
  \item \emph{prezzo medio effettivo}: \Eur{53,00};
  \item \emph{controvalore dell'operazione}: \Eur{5459,00};
  \item \emph{costo dell'operazione}: \Eur{16,10};
  \item \emph{controvalore totale}: \Eur{5475,10};
  \item \emph{prezzo medio di carico}: \Eur{53,1563}.
  \end{itemize}

  Dopo la terza operazione:
  \begin{itemize}
  \item numero quote nel saldo: \(203 + 103 = 306\);
  \item  il \emph{prezzo  medio effettivo  nel saldo}  è la  media ponderata  dei \emph{prezzi  medi
       effettivi} del saldo precedente e del nuovo acquisto:
    \begin{equation*}
      \MediaPonderataDue{203}{51,5025}{103}{53,00} = \Eur{52,0065}
    \end{equation*}
  \item il \emph{prezzo medio di carico carico nel saldo} è la media ponderata dei \emph{prezzi medi
       di carico} del saldo precedente e del nuovo acquisto:
    \begin{equation*}
      \MediaPonderataDue{203}{51,6556}{103}{53,1563} = \Eur{52,1608}
    \end{equation*}
  \end{itemize}
  il \emph{costo medio per quota nel saldo} risulta:
  \begin{equation*}
    \num{52,1608} - \num{52,0065} = \Eur{0,1542}
  \end{equation*}
\end{enumerate}

Evidenziando  i prezzi  medi: ogni  nuova quota  acquistata viene  \Virgolette{buttata nel  secchio}
insieme alle vecchie, indipendentemente dal suo prezzo di acquisto; è come se ogni quota fosse stata
acquistata allo stesso prezzo, pari al prezzo medio.

Si  osservi come,  \textbf{solo quando  si sono  eseguiti degli  acquisti ma  nessuna
   vendita}:
\begin{itemize}
\item il \emph{prezzo medio effettivo nel saldo} si possa calcolare anche come media ponderata tra i
  \emph{prezzi medi effettivi degli acquisti}:
  \begin{equation*}
    \MediaPonderataTre{101}{51,00}{102}{52,00}{103}{53,00} = \Eur{52,01}
  \end{equation*}
\item il \emph{prezzo medio di carico nel saldo} si possa calcolare anche come media ponderata tra i
  \emph{prezzi medi di carico degli acquisti}:
  \begin{equation*}
    \MediaPonderataTre{101}{51,1521}{102}{52,1542}{103}{53,1563} = \Eur{52,1608}
  \end{equation*}
\item il costo totale di tutte le operazioni si possa calcolare sia come somma tra tutti i costi:
  \begin{equation*}
    \num{15,36} + \num{15,73} + \num{16,10} = \Eur{47,19}
  \end{equation*}
  che come  prodotto tra  il numero  di quote nel  saldo e  il \emph{costo  medio per
     quota}:
  \begin{equation*}
    \num{306} \times{} \num{0,1542} = \Eur{47,19}
  \end{equation*}
\end{itemize}
dopo la prima operazione di vendita: queste proprietà non valgono piú!

%page
\section{Descrizione di un'operazione di vendita}


\Define{\UnoNumeroQuote}{20}
\Define{\DueNumeroQuote}{30}
\Define{\TreNumeroQuote}{50}
\Define{\UnoPrezzoEseguito}{52,00}
\Define{\DuePrezzoEseguito}{53,00}
\Define{\TrePrezzoEseguito}{55,00}

\Define{\NumeroQuote}{100}
\Define{\PrezzoMedioEffettivo}{53,80}
\Define{\ControvaloreOperazione}{5380,00}

\Define{\PrezzoMedioEffettivoNelSaldoPrecedente}{50,00}
\Define{\PrezzoMedioCaricoNelSaldoPrecedente}{50,15}
\Define{\RedditoDaCapitale}{380,00}
\Define{\TassaSulRedditoDaCapitale}{98,80}

\Define{\CostoOperazione}{15,91}
\Define{\CostoMedioPerQuotaNelSaldoPrecedente}{0,15}
\Define{\CostoAcquistoQuoteVendute}{15,00}
\Define{\RedditoDiverso}{-30,91}

\Define{\ControvaloreTotale}{5265,29}
\Define{\PrezzoMedioNetto}{52,6529}
\Define{\RendimentoPercentuale}{+4,9908}
\Define{\RendimentoInValuta}{250,29}


Quando si vendono quote di un titolo  \Parentesi{anche non \Etf{}} è generalmente meglio specificare
la strategia di  acquisto \Virgolette{con prezzo limite} e selezionare  esplicitamente il mercato su
cui si  vogliono eseguire le  operazioni, per esempio, nel  caso degli \Etf{}:  \emph{Borsa Italiana
   segmento ETFplus}.

Perciò in una certa data, e a una certa ora, si inserisce l'ordine di vendita per un numero di quote
del fondo a un certo prezzo minimo: per  essere conforme all'ordine inserito, la banca che agisce da
intermediario può eseguire  l'ordine vendendo a qualsiasi prezzo \emph{maggiore  o uguale} al limite
fissato.  L'ordine può essere eseguito  in piú fasi in cui solo una parte  delle quote è venduta, in
ogni fase a  un prezzo diverso; è possibile che  non tutte le quote siano vendute,  nel qual caso ci
interessa solo la frazione \Virgolette{eseguita} dell'ordine.

Il  \textbf{prezzo  medio  eseguito  di  vendita} (nella  terminologia  della  Banca),  anche  detto
\textbf{prezzo  medio effettivo  di vendita}  (nella terminologia  dell'Agenzia dell'Entrate),  è la
media ponderata  dei prezzi  di vendita rispetto  al numero  di quote vendute.   Per esempio,  se un
singolo ordine di vendita per \num{\NumeroQuote} quote è eseguito nelle fasi:
\begin{enumerate}
\item vendita di \num{\UnoNumeroQuote} quote al prezzo di \Eur{\UnoPrezzoEseguito};
\item vendita di \num{\DueNumeroQuote} quote al prezzo di \Eur{\DuePrezzoEseguito};
\item vendita di \num{\TreNumeroQuote} quote al prezzo di \Eur{\TrePrezzoEseguito};
\end{enumerate}
allora il \emph{prezzo medio effettivo} risulta:
\begin{equation*}
  \MediaPonderataTre{\UnoNumeroQuote}{\UnoPrezzoEseguito}{\DueNumeroQuote}{\DuePrezzoEseguito}{\TreNumeroQuote}{\TrePrezzoEseguito}
  = \Eur{\PrezzoMedioEffettivo}
\end{equation*}

Il \textbf{controvalore  dell'operazione di  vendita} è il  prodotto tra il  numero totale  di quote
vendute e il \emph{prezzo medio effettivo}.  Per  esempio, se si vendono \num{\NumeroQuote} quote al
\emph{prezzo  medio  effettivo}  di   \Eur{\PrezzoMedioEffettivo}  il  controvalore  dell'operazione
risulta:
\begin{equation*}
  \num{\NumeroQuote} \times{} \num{\PrezzoMedioEffettivo}
  = \Eur{\ControvaloreOperazione}
\end{equation*}

La  circolare dell'Agenzia  dell'Entrate  specifica che  il \emph{reddito  per  quota} derivante  da
un'operazione  di  vendita  è la  differenza  tra  \emph{prezzo  medio  effettivo della  vendita}  e
\emph{prezzo medio effettivo nel saldo}; peculiarmente:
\begin{itemize}
\item quando  la differenza tra  prezzi medi effettivi è  \textbf{positiva}: essa è  da considerarsi
  \emph{reddito da capitale}, quindi \textbf{non genera} una plusvalenza utilizzabile per compensare
  precedenti minusvalenze da altri investimenti;
\item quando la differenza tra prezzi medi  effettivi è \textbf{negativa}: essa è da considerarsi un
  \emph{reddito  diverso},  quindi  \textbf{genera}  una minusvalenza  compensabile  con  successive
  plusvalenze da altri investimenti.
\end{itemize}

Quando  il reddito  di  una vendita  è  positivo: occorre  pagare la  \textbf{tassa  sul reddito  da
   capitale}\footnote{Si deve  pagare anche il  bollo sull'estratto  conto trimestrale per  il Conto
   Titoli,  pari allo  \SI{0,2}{\percent} all'anno,  cioè  lo \SI{0,05}{\percent}  al trimestre  del
   controvalore alla data di chiusura dei conti  trimestrali; il bollo è addebitato direttamente sul
   Conto Corrente, perciò in questa guida lo si considera conteggiato a parte.}.  Per esempio, se si
vendono \num{\NumeroQuote} quote  al \emph{prezzo medio effettivo}  di \Eur{\PrezzoMedioEffettivo} e
il      \emph{prezzo       medio      effettivo      nel      saldo}       precedente      è      di
\Eur{\PrezzoMedioEffettivoNelSaldoPrecedente}, il \emph{reddito da capitale} risulta:
\begin{equation*}
  \num{\NumeroQuote} \times{} \left(
    \num{\PrezzoMedioEffettivo} - \num{\PrezzoMedioEffettivoNelSaldoPrecedente}
  \right) = \Eur{\RedditoDaCapitale}
\end{equation*}
per fondi  che non  contengono Titoli  dello Stato Italiano,  e assimilati,  l'aliquota unica  è del
\SI{26}{\percent}; allora la \emph{tassa sul reddito da capitale} risulta:
\begin{equation*}
  \num{0,26} \times{} \num{\RedditoDaCapitale} = \Eur{\TassaSulRedditoDaCapitale}
\end{equation*}

Il \textbf{costo  dell'operazione di vendita}  è la somma tra  costi, spese e  commissioni associate
all'operazione; per ogni ordine di vendita eseguito, occorre pagare:
\begin{itemize}
\item  le commissioni  per  l'intermediario, fissate  a  \Eur{2,50} quale  che  sia il  controvalore
  dell'operazione eseguita  \Parentesi{la Borsa  di negoziazione  su cui  si eseguono  le operazioni
     vuole essere pagata per ogni eseguito};
\item le commissioni fisse per la Banca pari a \Eur{0,50};
\item le  commissioni variabili per  la Banca  pari allo \SI{0,24}{\percent}  del \emph{controvalore
     dell'operazione}.
\end{itemize}
se un ordine non è eseguito: si paga nulla; se un ordine è eseguito in piú fasi: nella prima fase si
pagano  i costi  fissi  piú  la commissione  percentuale;  nelle fasi  successive  si  paga solo  la
commissione   percentuale.     Per   esempio,   se   il    \emph{controvalore   dell'operazione}   è
\Eur{\ControvaloreOperazione} il \emph{costo dell'operazione} risulta:
\begin{equation*}
  \CalcoloCostoOperazione{\ControvaloreOperazione} = \Eur{\CostoOperazione}
\end{equation*}

Da osservare  che: sia  il calcolo  della \emph{tassa  sul reddito  da capitale}  che i  calcoli del
\emph{costo  dell'operazione}  di acquisto  e  vendita  si  eseguono  usando il  \emph{prezzo  medio
   effettivo}; in pratica: si  pagano le tasse anche sul \emph{costo  dell'operazione} di acquisto e
vendita, che però sono redditi di qualcun altro, non dell'investitore!

La circolare  dell'Agenzia dell'Entrate  specifica che  il \emph{costo  delle operazioni}  genera un
\emph{reddito diverso}, da  registrare come minusvalenza.  Per esempio, si  supponga di possedere un
Conto Titoli con:
\begin{itemize}
\item numero quote nel saldo precedente alla vendita: \num{\NumeroQuote};
\item     \emph{prezzo     medio     effettivo     nel    saldo}     precedente     alla     vendita
  \Eur{\PrezzoMedioEffettivoNelSaldoPrecedente};
\item    \emph{prezzo     medio    di    carico     nel    saldo}    precedente     alla    vendita:
  \Eur{\PrezzoMedioCaricoNelSaldoPrecedente};
\item    \emph{costo     medio    per     quota    nel     saldo}    precedente     alla    vendita:
  \Eur{\CostoMedioPerQuotaNelSaldoPrecedente};
\end{itemize}
si vendano tutte le quote; l'operazione di vendita sia descritta da:
\begin{itemize}
\item numero quote vendute: \num{\NumeroQuote};
\item \emph{prezzo medio effettivo di vendita} \Eur{\PrezzoMedioEffettivo};
\item \emph{reddito da capitale}: \Eur{\RedditoDaCapitale};
\item \emph{costo dell'operazione di vendita}: \Eur{\CostoOperazione};
\end{itemize}
ogni   quota   venduta   era  stata   acquistata   con   un   \emph{costo   medio  per   quota}   di
\Eur{\CostoMedioPerQuotaNelSaldoPrecedente}, perciò il \emph{costo  di acquisto delle quote vendute}
risulta:
\begin{equation*}
  \num{\NumeroQuote} \times{} \num{\CostoMedioPerQuotaNelSaldoPrecedente}
  = \Eur{\CostoAcquistoQuoteVendute}
\end{equation*}
allo scopo di far risultare un numero negativo, la circolare specifica che il \emph{reddito diverso}
associato alla vendita deve essere calcolato come:
\begin{equation*}
  \left[
    \num{\RedditoDaCapitale} - \left(
      \num{\CostoOperazione} + \num{\CostoAcquistoQuoteVendute}
    \right)
  \right] - \num{\RedditoDaCapitale}
  = - \left( \num{\CostoOperazione} + \num{\CostoAcquistoQuoteVendute} \right)
  = \Eur{\RedditoDiverso}
\end{equation*}
questa  minusvalenza  viene  automaticamente  registrata  dalla  Banca  nella  \Virgolette{Posizione
   Fiscale} del Conto Titoli; da notare che  \textbf{le minusvalenze derivanti dai costi di acquisto
   e vendita  sono registrate nella Posizione  Fiscale solo al  momento della vendita di  quote}, al
momento dell'acquisto nulla viene registrato.

Il   \textbf{controvalore  totale   di  vendita}   è   la  differenza   tra  il   \emph{controvalore
   dell'operazione}  e la  somma  tra  \emph{costo dell'operazione}  e  \emph{tassa  sul reddito  da
   capitale};  è  la  quantità di  denaro  in  entrata  nel  Conto  Corrente.  Per  esempio,  se  il
\emph{controvalore dell'operazione} è \Eur{\ControvaloreOperazione}, il \emph{costo dell'operazione}
è     \Eur{\CostoOperazione}    e     la    \emph{tassa     sul    reddito     da    capitale}     è
\Eur{\TassaSulRedditoDaCapitale}, il \emph{controvalore totale} risulta:
\begin{equation*}
  \num{\ControvaloreOperazione} -
  \left( \num{\CostoOperazione} + \num{\TassaSulRedditoDaCapitale} \right)
  = \Eur{\ControvaloreTotale}
\end{equation*}

In un'operazione  di vendita le  quote vengono \Virgolette{scaricate}  dal Conto Titoli:  la vendita
modifica solo  il numero di quote  in carico nel saldo,  il \emph{prezzo medio di  carico nel saldo}
resta invariato da prima a dopo la  vendita.  Allo scopo di calcolare il rendimento dell'operazione,
si definisce il  \textbf{prezzo medio netto di  vendita} pari al rapporto  tra il \emph{controvalore
   totale} e il numero  di quote vendute.  Per esempio, se si  sono vendute \num{\NumeroQuote} quote
al \emph{controvalore totale} di \Eur{\ControvaloreTotale},  il \emph{prezzo medio netto di vendita}
risulta:
\begin{equation*}
  \num{\ControvaloreTotale} / \num{\NumeroQuote} = \Eur{\PrezzoMedioNetto}
\end{equation*}

Il \textbf{rendimento  \underline{percentuale} dell'operazione di  vendita}, rispetto alle  quote in
carico nel  saldo precedente,  si calcola  considerando il  \emph{prezzo medio  di carico  nel saldo
   precedente} e il \emph{prezzo medio netto di vendita};  in caso di guadagno è un numero positivo,
in caso di perdita è un numero negativo.  Per  esempio, se il \emph{prezzo medio di carico nel saldo
   precedente}  è  \Eur{\PrezzoMedioCaricoNelSaldoPrecedente}  e  il  \emph{prezzo  medio  netto  di
   vendita} è \Eur{\PrezzoMedioNetto}, il \emph{rendimento percentuale} risulta:
\begin{equation*}
  \CalcoloRendimentoPercentuale{\PrezzoMedioNetto}{\PrezzoMedioCaricoNelSaldoPrecedente}
  = \Perc{\RendimentoPercentuale}
\end{equation*}

Il  \textbf{rendimento \underline{in  valuta} dell'operazione  di vendita},  rispetto alle  quote in
carico nel  saldo precedente,  si calcola  considerando il  \emph{prezzo medio  di carico  nel saldo
   precedente} e il \emph{prezzo medio netto di vendita};  in caso di guadagno è un numero positivo,
in caso di  perdita è un numero negativo.   Per esempio, se si vendono  \num{\NumeroQuote} quote con
\emph{prezzo medio di carico nel saldo precedente} pari a \Eur{\PrezzoMedioCaricoNelSaldoPrecedente}
e  \emph{prezzo medio  netto  di vendita}  pari a  \Eur{\PrezzoMedioNetto},  il \emph{rendimento  in
   valuta} risulta:
\begin{equation*}
  \num{\NumeroQuote} \times{} \left(
    \num{\PrezzoMedioNetto} - \num{\PrezzoMedioCaricoNelSaldoPrecedente}
  \right) = \Eur{\RendimentoInValuta}
\end{equation*}

%page
\section{Aggiornamento del saldo dopo un'operazione di vendita}


Dopo  l'esecuzione di  ogni operazione  di vendita  occorre ricalcolare  il saldo.   Il saldo  di un
investimento in quote di \Etf{} è rappresentato dalle tre quantità:
\begin{itemize}
\item numero di quote di quell'\Etf{} in carico nel Conto Titoli;
\item \emph{prezzo medio effettivo nel saldo} per ogni quota in carico;
\item \emph{prezzo medio di carico nel saldo} per ogni quota in carico;
\end{itemize}
sia  il \emph{prezzo  medio effettivo  nel saldo}  che il  \emph{prezzo medio  di carico  nel saldo}
restano  \textbf{immutati} dopo  una  vendita: essa  modifica  solo  il numero  di  quote in  carico
diminuendolo del numero di quote vendute.

Per esempio, si supponga di avere un Conto Titoli con saldo iniziale:
\begin{itemize}
\item numero di quote: \num{100};
\item \emph{prezzo medio effettivo nel saldo}: \Eur{50,00};
\item \emph{prezzo medio di carico nel saldo}: \Eur{50,15};
\end{itemize}
se si vendessero \num{90} quote, non importa a quale prezzo, il saldo finale risulterebbe:
\begin{itemize}
\item numero di quote: \(\num{100} - \num{90} = \num{10}\);
\item \emph{prezzo medio effettivo nel saldo}: \Eur{50,00}, immutato;
\item \emph{prezzo medio di carico nel saldo}: \Eur{50,15}, immutato.
\end{itemize}

%page
\section{Rendimento indicato nel riepilogo del patrimonio}


Nel sito di  banca telematica è indicato, per  ogni fondo nel Conto Titoli, il  saldo costituito dal
numero  di quote  in  carico e  dal  \emph{prezzo  medio di  carico};  in piú  è  visibile la  stima
dell'\emph{Utile  o Perdita},  nell'ipotesi  di vendita  di  tutte le  quote  al prezzo  dell'ultimo
contratto di compravendita sulla Borsa di negoziazione del titolo:
\begin{itemize}
\item l'\emph{Utile} mostrato è  al lordo sia del \emph{costo dell'operazione  di vendita} che della
  \emph{tassa sul reddito da capitale};
\item la \emph{Perdita} mostrata è al lordo del \emph{costo dell'operazione} di vendita.
\end{itemize}

Si supponga di possedere \num{100} quote al \emph{prezzo medio di carico nel saldo} di \Eur{50,00}:
\begin{itemize}
\item  se l'ultimo  contratto  di  compravendita è  stato  chiuso  al prezzo  di  una  quota pari  a
  \Eur{52,00}, l'\emph{Utile percentuale} mostrato nella tabella del patrimonio risulterebbe:
  \begin{equation*}
    \CalcoloRendimentoPercentuale{52,00}{50,00} = \SI{+3,6889}{\percent}
  \end{equation*}
  mentre l'\emph{Utile in valuta} risulterebbe:
  \begin{equation*}
    \num{100} \times{} (\num{52,00} - \num{50,00}) = \Eur{+185,00}
  \end{equation*}
  in realtà, vendendo tutte le quote al \emph{prezzo medio effettivo} di \Eur{52,00}, risulterebbe:
  \begin{itemize}
  \item \emph{prezzo medio netto}: \Eur{51,33};
  \item \emph{rendimento percentuale}: \SI{+2,3434}{\percent};
  \item \emph{rendimento in valuta}: \Eur{+117,52};
  \end{itemize}

\item  se l'ultimo  contratto  di  compravendita è  stato  chiuso  al prezzo  di  una  quota pari  a
  \Eur{48,00}, la \emph{Perdita percentuale} mostrata nella tabella del patrimonio risulterebbe:
  \begin{equation*}
    \CalcoloRendimentoPercentuale{48,00}{50,00} = \SI{-4,2871}{\percent}
  \end{equation*}
  mentre la \emph{Perdita in valuta} risulterebbe:
  \begin{equation*}
    \num{100} \times{} (\num{48,00} - \num{50,00}) = \Eur{-215,00}
  \end{equation*}
  in realtà, vendendo tutte le quote al \emph{prezzo medio effettivo} di \Eur{48,00}, risulterebbe:
  \begin{itemize}
  \item \emph{prezzo medio netto}: \Eur{47,85};
  \item \emph{perdita percentuale}: \SI{-4,5767}{\percent};
  \item \emph{perdita in valuta}: \Eur{-229,52}.
  \end{itemize}
\end{itemize}

%page
\section{Strategie per la scelta dei prezzi di vendita}


Si cerca  di vendere a  un prezzo maggiore  del prezzo di acquisto;  se occorresse liquidità  per le
proprie spese: si potrebbe essere costretti a disinvestire, vendendo a prezzi inferiori.

%page
\subsection{Metodo delle medie ponderate}


Si considerano tutte le quote di uno stesso  \Etf{}, acquistate a qualsiasi prezzo, come parte dello
stesso insieme.  Si cerca  di vendere le quote a un prezzo talmente  maggiore del prezzo di acquisto
che, recuperati i costi e pagate le tasse, si realizza un guadagno.
\begin{itemize}
\item Se  il \emph{prezzo medio netto  di vendita} è maggiore  del \emph{prezzo medio di  carico nel
     saldo} in quel momento: si recuperano i costi di acquisto e vendita; si recuperano le tasse; si
  chiude con un guadagno.

\item Se il \emph{prezzo medio netto di vendita} è uguale al \emph{prezzo medio di carico nel saldo}
  in quel  momento: si recuperano i  costi di acquisto e  vendita; si recuperano le  tasse; ma resta
  nessun guadagno, si chiude in pareggio.

\item Se  il \emph{prezzo  medio netto di  vendita} è  minore del \emph{prezzo  medio di  carico nel
     saldo}: tolti i costi di acquisto e  vendita ed eventualmente le tasse, l'operazione genera una
  perdita.
\end{itemize}
In ogni caso, restano  le minusvalenze accumulate che, se e quando  possibile, potranno essere usate
per ridurre le tasse da pagare in future operazioni che generano \emph{redditi diversi}.

Si supponga di avere un Conto Titoli con il seguente saldo:
\begin{itemize}
\item numero quote: \num{100};
\item prezzo medio effettivo: \Eur{50,00};
\item costo medio per quota: \Eur{0,15};
\item prezzo medio di carico: \Eur{50,15};
\item controvalore di carico: \Eur{5015,00};
\item minusvalenze accumulate: \Eur{0,00};
\end{itemize}
si vogliano vendere tutte le quote.

%% ------------------------------------------------------------------------

Come esempio di guadagno, si venda come segue:
\begin{itemize}
\item numero quote: \num{100};
\item prezzo medio eseguito: \Eur{52,00};
\item controvalore dell'operazione: \Eur{5200,00};
\item controvalore totale: \Eur{5132,52};
\item prezzo medio netto di vendita: \Eur{51,33};

\item reddito da capitale: \Eur{200,00};
\item tasse sul reddito: \Eur{52,00};
\item costo dell'operazione di vendita: \Eur{15,48};
\item costo convenzionale di acquisto delle quote vendute: \Eur{15,00};
\item reddito diverso: \Eur{-30,48};

\item rendimento percentuale: \SI{2,3434}{\percent};
\item rendimento in valuta: \Eur{117,52};
\end{itemize}
il \emph{prezzo  medio netto di  vendita} pari  a \Eur{51,33} è  maggiore del \emph{prezzo  medio di
   carico nel saldo} pari a \Eur{50,15}, quindi,  recuperati costi e tasse, si realizza un guadagno;
il reddito diverso è una minusvalenza da compensare con future operazioni.

%% ------------------------------------------------------------------------

Come esempio di perdita, si venda come segue:
\begin{itemize}
\item numero quote: \num{100};
\item prezzo medio eseguito: \Eur{50,30};
\item controvalore dell'operazione: \Eur{5030,00};
\item controvalore totale: \Eur{5007,13};
\item prezzo medio netto di vendita: \Eur{50,07};

\item reddito da capitale: \Eur{30,00};
\item tasse sul reddito: \Eur{7,80};
\item costo dell'operazione di vendita: \Eur{15,07};
\item costo convenzionale di acquisto delle quote vendute: \Eur{15,00};
\item reddito diverso: \Eur{-30,07};

\item rendimento percentuale: \SI{-0,1570}{\percent};
\item rendimento in valuta: \Eur{-7,87};
\end{itemize}
nonostante il \emph{prezzo medio eseguito} pari a \Eur{50,30} sia maggiore del \emph{prezzo medio di
   carico nel saldo}  pari a \Eur{50,15}, il  \emph{prezzo medio netto di vendita}  di \Eur{50,07} è
minore, quindi, tolti costi e tasse, si realizza  una perdita; il reddito diverso è una minusvalenza
da compensare con future operazioni.

%% ------------------------------------------------------------------------

Come ulteriore esempio di perdita, si venda come segue:
\begin{itemize}
\item numero quote: \num{100};
\item prezzo medio eseguito: \Eur{48,00};
\item controvalore dell'operazione: \Eur{4800,00};
\item controvalore totale: \Eur{4785,48};
\item prezzo medio netto di vendita: \Eur{47,85};

\item reddito da capitale: \Eur{0,00};
\item tasse sul reddito: \Eur{0,00};
\item costo dell'operazione di vendita: \Eur{14,52};
\item costo convenzionale di acquisto delle quote vendute: \Eur{15,00};
\item redditi diversi: \Eur{-229,52};

\item rendimento percentuale: \SI{-4,5767}{\percent};
\item rendimento in valuta: \Eur{-229,52};
\end{itemize}
il \emph{prezzo medio netto di vendita} pari a \Eur{47,85} è minore del \emph{prezzo medio di carico
   nel saldo} pari a \Eur{50,15}, quindi si realizza una perdita; si osserva che:
\begin{itemize}
\item il \emph{reddito da capitale} è zero e le tasse sono nulle;
\item la differenza tra il \emph{controvalore dell'operazione di vendita} e il \emph{controvalore di
     carico nel saldo}:
  \begin{equation*}
    \num{4800,00} - \num{5015,00} = \Eur{-215,00}
  \end{equation*}
  rappresenta la  perdita di reddito dovuta  alla vendita, è  negativa ed è registrata  come reddito
  diverso;
\item   la  differenza   tra  il   \emph{controvalore   totale  di   vendita}  e   il
  \emph{controvalore di carico nel saldo}:
  \begin{equation*}
    \num{4785,48} - \num{5015,00} = \Eur{-229,52}
  \end{equation*}
  è negativa  e rappresenta l'intero  reddito diverso, includendo sia  la perdita di  reddito dovuta
  alla vendita che il costo delle operazioni;
\end{itemize}
il reddito diverso è una minusvalenza da compensare con future operazioni.

%% ------------------------------------------------------------------------

La domanda  fondamentale è:  \textbf{possedendo un  numero di  quote a  un certo  \emph{prezzo medio
      effettivo} e un  certo \emph{prezzo medio di  carico} nel saldo, qual'è  il \emph{prezzo medio
      eseguito di vendita} che permette di realizzare un guadagno?}

Per esempio,  si posseggano  \num{100} quote  a un \emph{prezzo  medio effettivo  nel saldo}  pari a
\Eur{50,00} e a un \emph{prezzo medio di carico nel saldo} pari a \Eur{50,15}, si voglia determinare
il \emph{prezzo medio \underline{eseguito} di vendita} per  tutte le quote che realizzi un guadagno.
Detto \(X\) tale valore, risulta:
\begin{itemize}
\item \emph{controvalore dell'operazione}: \(\num{100} \times{} X\);
\item \emph{costo dell'operazione di vendita}:
  \begin{align*}
    \num{0,50} + \num{2,50} + \num{0,0024} \times{} (\num{100} \times{} X)
    &= \num{3} + \num{0,0024} \times{} \num{100} \times{} X = \\
    &= \num{3} + \num{0,24} \times{} X
  \end{align*}
\item \emph{tassa sul reddito da capitale}:
  \begin{align*}
    \num{0,26} \times{} \num{100} \times{} (X - \num{50,00})
    &= \num{26} \times{} X - \num{26} \times{} \num{50} = \\
    &= \num{26} \times{} X - \num{1300}
  \end{align*}
\item \emph{controvalore totale di vendita}:
  \begin{align*}
    (\num{100} \times{} X)
    &- (\num{3} + \num{0,24} \times{} X)
      - (\num{26} \times{} X - \num{1300})
      = \\
    &= \num{100} \times{} X
      - \num{3} - \num{0,24} \times{} X
      - \num{26} \times{} X + \num{1300}
      = \\
    &= (\num{100} - \num{0,24} - \num{26}) \times{} X + (\num{1300} - \num{3})
      = \\
    &= \num{73,76} \times{} X + \num{1297}
  \end{align*}
\item \emph{prezzo medio netto di vendita}:
  \begin{equation*}
    \frac{\num{73,76} \times{} X + \num{1297}}{100}
  \end{equation*}
\end{itemize}
il  pareggio  si  raggiungerebbe  vendendo  le  quote a  un  \emph{prezzo  medio  netto}  uguale  al
\emph{prezzo medio di carico nel saldo} pari a \Eur{50,15}; quindi:
\begin{equation*}
  \frac{\num{73,76} \times{} X + \num{1297}}{100} = \num{50,15}
\end{equation*}
risolvendo rispetto a \(X\):
\begin{equation*}
  X = \frac{\num{100} \times{} \num{50,15} - \num{1297}}{\num{73,76}}
  = \Eur{50,4067}
\end{equation*}
vendendo con un \emph{prezzo medio eseguito} maggiore di \Eur{50,4067} si realizza un guadagno.

%page
\subsection{Metodo delle linee di investimento}


Si è visto  come, dal punto di  vista contabile: tutte le quote  di uno stesso \Etf{}  in carico nel
Conto Titoli siano  parte dello stesso insieme;  in particolare, per ogni operazione  di vendita: le
\emph{tasse   sul   reddito  da   capitale}   si   calcolano   rispetto   al  valore   attuale   del
\PrezzoMedioEffettivoNelSaldo{}, indipendentemente da quanti acquisti si siano eseguiti in passato e
dal loro \PrezzoMedioEffettivoAcquisto{}.

Nonostante ciò, si immagini questo scenario:
\begin{itemize}
\item si disponga di un capitale di \Eur{15000,00};
\item si vogliano  definire tre \Virgolette{linee di investimento} formalmente  separate, dal nostro
  punto di vista; a ciascuna linea si allochino \Eur{5000,00};
\item come primo investimento, per ciascuna linea, si acquistino quote dello stesso \Etf{};
\item in futuro si  prevede di vendere queste quote e acquistare, per  ogni linea, prodotti diversi,
  mantenendo separato il destino di ognuno dei tre capitali iniziali da \Eur{5000,00};
\item si vuole che ogni linea di investimento risulti redditizia.
\end{itemize}

Le operazioni di acquisto siano come segue:
\begin{enumerate}
\item per la prima  linea di investimento si acquisti quando il prezzo  di mercato è \Eur{40,00} per
  quota; il numero di quote acquistabili risulta:
  \begin{equation*}
    \num{5000,00} / \num{40,00} = \num{125}
  \end{equation*}
  il riassunto dell'operazione di acquisto per la prima linea è:
  \begin{itemize}
  \item numero quote \num{125};
  \item prezzo medio eseguito di acquisto \Eur{40,00};
  \item controvalore dell'operazione \Eur{5000,00};
  \item costo dell'operazione di acquisto \Eur{15,00};
  \item controvalore totale \Eur{5015,00};
  \item prezzo medio di carico di acquisto \Eur{40,12};
  \end{itemize}
\item per la seconda linea di investimento si acquisti quando il prezzo di mercato è \Eur{50,00} per
  quota; il numero di quote acquistabili risulta:
  \begin{equation*}
    \num{5000,00} / \num{50,00} = \num{100}
  \end{equation*}
  il riassunto dell'operazione di acquisto per la seconda linea è:
  \begin{itemize}
  \item numero quote \num{100};
  \item prezzo medio eseguito di acquisto \Eur{50,00};
  \item controvalore dell'operazione \Eur{5000,00};
  \item costo dell'operazione di acquisto \Eur{15,00};
  \item controvalore totale \Eur{5015,00};
  \item prezzo medio di carico di acquisto \Eur{50,15};
  \end{itemize}
\item per la terza  linea di investimento si acquisti quando il prezzo  di mercato è \Eur{60,00} per
  quota; il numero di quote acquistabili risulta:
  \begin{equation*}
    \num{5000,00} / \num{60,00} \simeq{} \num{83,33}
  \end{equation*}
  arrotondato a \num{83}, da cui un capitale effettivamente investito pari a:
  \begin{equation*}
    \num{83} \times{} \num{60,00} = \Eur{4980,00}
  \end{equation*}
  il riassunto dell'operazione di acquisto per la terza linea è:
  \begin{itemize}
  \item numero quote \num{83};
  \item prezzo medio eseguito di acquisto \Eur{60,00};
  \item controvalore dell'operazione \Eur{4980,00};
  \item costo dell'operazione di acquisto \Eur{14,95};
  \item controvalore totale \Eur{4994,95};
  \item prezzo medio di carico di acquisto \Eur{60,18}.
  \end{itemize}
\end{enumerate}

Nel Conto Titoli compare il saldo per il numero totale di quote acquistate:
\begin{equation*}
  \num{125} + \num{100} + \num{83} = \num{308}
\end{equation*}
il  \emph{prezzo medio  effettivo nel  saldo} risulta  dalla media  ponderata dei  \emph{prezzi medi
   effettivi di acquisto}:
\begin{equation*}
  \frac{\num{125} \times{} \num{40,00} + \num{100} \times{} \num{50,00} + \num{83} \times{} \num{60}}
  {\num{125} + \num{100} + \num{83}} = \Eur{48,6364}
\end{equation*}
il \emph{prezzo medio  di carico nel saldo}  risulta dalla media ponderata dei  \emph{prezzi medi di
   carico di acquisto}:
\begin{equation*}
  \frac{\num{125} \times{} \num{40,12} + \num{100} \times{} \num{50,15} + \num{83} \times{} \num{60,18}}
  {\num{125} + \num{100} + \num{83}} = \Eur{48,7823}
\end{equation*}

Si è osservato  in precedenza che quando si  esegue una vendita parziale delle quote  in carico: nel
saldo del  Conto Titoli l'unico numero  che cambia è il  numero di quote in  carico, il \emph{prezzo
   medio effettivo nel saldo}  e il \emph{prezzo medio di carico nel  saldo} restano invariati!  Ciò
significa che  \textbf{l'ordine con  cui si  liquidano le quote  delle tre  linee di  investimento è
   ininfluente ai fini dei rendimenti ottenuti}.

%page
\paragraph{Vendita delle quote della prima linea di investimento.}
Si liquidino le \num{125} quote della  prima linea di investimento al \PrezzoMedioEffettivoVendita{}
di \Eur{46,00}, maggiore del \PrezzoMedioCaricoAcquisto{}  delle quote stesse \Eur{40,12}, ma minore
del \PrezzoMedioCaricoSaldo{} \Eur{48,7823}.  Il riassunto  della contabilità fiscale come calcolato
dalla Banca è il seguente:
\begin{itemize}
\item numero quote \num{125};
\item prezzo medio effettivo di vendita \Eur{46,00};
\item controvalore dell'operazione di vendita \Eur{5750,00};
\item reddito da capitale da plusvalenza sul capitale \Eur{0,00};
\item tassa sul reddito da capitale \Eur{0,00};
\item costo dell'operazione di vendita \Eur{16,80};
\item costo di acquisto delle quote vendute \Eur{18,24};
\item reddito diverso da minusvalenza sul capitale \Eur{-329,55};
\item reddito diverso da pagamento commissioni \Eur{-35,04};
\item reddito diverso totale \Eur{-364,59};
\item minusvalenza registrata nella posizione fiscale \Eur{364,59};
\item controvalore totale di vendita \Eur{5733,20}, sono i soldi che tornano sul Conto Corrente;
\item prezzo medio netto di vendita \Eur{45,8656};
\item rendimento percentuale rispetto al \PrezzoMedioCaricoSaldo{} \SI{-5,9790}{\percent};
\item rendimento in valuta rispetto al \PrezzoMedioCaricoSaldo{} \Eur{-364,59}.
\end{itemize}
Il risultato di  questa operazione dal punto di  vista della prima linea di  investimento può essere
calcolato in due modi:
\begin{enumerate}
\item  con la  quantità  di  soldi uscita  dal  Conto  Corrente per  acquistare  le  quote, cioè  il
  \emph{controvalore totale  di acquisto} \Eur{5015,00},  e la quantità  di soldi entrata  nel Conto
  Corrente dopo la vendita delle quote, cioè il \emph{controvalore totale di vendita} \Eur{5733,20};
\item con il \emph{prezzo medio di carico di  acquisto} \Eur{40,12} e il \emph{prezzo medio netto di
     vendita} \Eur{45,8656};
\end{enumerate}
il rendimento della compravendita risulta:
\begin{equation*}
  \CalcoloRendimentoPercentuale{5733,20}{5015,00} =
  \CalcoloRendimentoPercentuale{45,8656}{40,12} = \SI{14,32}{\percent}
\end{equation*}

%page
\paragraph{Vendita delle quote della seconda linea di investimento.}
Si    liquidino    le    \num{100}    quote    della    seconda    linea    di    investimento    al
\PrezzoMedioEffettivoVendita{} di  \Eur{49,50}, minore del \PrezzoMedioCaricoAcquisto{}  delle quote
stesse \Eur{50,00},  ma maggiore  del \PrezzoMedioCaricoSaldo{}  \Eur{48,7823}.  Il  riassunto della
contabilità fiscale come calcolato dalla Banca è il seguente:
\begin{itemize}
\item numero quote \num{100};
\item prezzo medio effettivo di vendita \Eur{49,50};
\item controvalore dell'operazione di vendita \Eur{4950,00};
\item reddito da capitale da plusvalenza sul capitale \Eur{86,36};
\item tassa sul reddito da capitale \Eur{22,45};
\item costo dell'operazione di vendita \Eur{14,88};
\item costo di acquisto delle quote vendute \Eur{14,59};
\item reddito diverso da minusvalenza sul capitale \Eur{0,00};
\item reddito diverso da pagamento commissioni \Eur{-29,47};
\item reddito diverso totale \Eur{-29,47};
\item minusvalenza registrata nella posizione fiscale \Eur{29,47};
\item controvalore totale di vendita \Eur{4912,67}, sono i soldi che tornano sul Conto Corrente;
\item prezzo medio netto di vendita \Eur{49,1267};
\item rendimento percentuale rispetto al \PrezzoMedioCaricoSaldo{} \SI{0,7059}{\percent};
\item rendimento in valuta rispetto al \PrezzoMedioCaricoSaldo{} \Eur{34,43}.
\end{itemize}
Il risultato di questa operazione dal punto di  vista della seconda linea di investimento può essere
calcolato in due modi:
\begin{enumerate}
\item  con la  quantità  di  soldi uscita  dal  Conto  Corrente per  acquistare  le  quote, cioè  il
  \emph{controvalore totale  di acquisto} \Eur{5000,00},  e la quantità  di soldi entrata  nel Conto
  Corrente dopo la vendita delle quote, cioè il \emph{controvalore totale di vendita} \Eur{4912,67};
\item con il \emph{prezzo medio di carico di  acquisto} \Eur{40,12} e il \emph{prezzo medio netto di
     vendita} \Eur{45,8656};
\end{enumerate}
il rendimento della compravendita risulta:
\begin{equation*}
  \CalcoloRendimentoPercentuale{4912,67}{5000,00} =
  \CalcoloRendimentoPercentuale{49,13}{50,15} = \SI{-2,04}{\percent}
\end{equation*}

%page
\paragraph{Vendita delle quote della terza linea di investimento.}
Si liquidino le  \num{83} quote della terza linea di  investimento al \PrezzoMedioEffettivoVendita{}
di \Eur{65,00}, maggiore del \PrezzoMedioCaricoAcquisto{}  delle quote stesse \Eur{60,00} e maggiore
del \PrezzoMedioCaricoSaldo{} \Eur{48,7823}.  Il riassunto  della contabilità fiscale come calcolato
dalla Banca è il seguente:
\begin{itemize}
\item numero quote \num{83};
\item prezzo medio effettivo di vendita \Eur{65,00};
\item controvalore dell'operazione di vendita \Eur{5395,00};
\item reddito da capitale da plusvalenza sul capitale \Eur{1358,18};
\item tassa sul reddito da capitale \Eur{353,13};
\item costo dell'operazione di vendita \Eur{15,95};
\item costo di acquisto delle quote vendute \Eur{12,11};
\item reddito diverso da minusvalenza sul capitale \Eur{0,00};
\item reddito diverso da pagamento commissioni \Eur{-28,06};
\item reddito diverso totale \Eur{-28,06};
\item minusvalenza registrata nella posizione fiscale \Eur{28,06};
\item controvalore totale di vendita \Eur{5025,92}, sono i soldi che tornano sul Conto Corrente;
\item prezzo medio netto di vendita \Eur{60,5533};
\item rendimento percentuale rispetto al \PrezzoMedioCaricoSaldo{} {\SI{24,1296}{\percent}};
\item rendimento in valuta rispetto al \PrezzoMedioCaricoSaldo{} \Eur{976,99}.
\end{itemize}
Il risultato di  questa operazione dal punto di  vista della terza linea di  investimento può essere
calcolato in due modi:
\begin{enumerate}
\item  con la  quantità  di  soldi uscita  dal  Conto  Corrente per  acquistare  le  quote, cioè  il
  \ControvaloreTotaleAcquisto{} \Eur{4994,95},  e la  quantità di soldi  entrata nel  Conto Corrente
  dopo la vendita delle quote, cioè il \ControvaloreTotaleVendita{} \Eur{5025,92};
\item   con   il   \PrezzoMedioCaricoAcquisto{}  \Eur{60,1801}   e   il   \PrezzoMedioNettoVendita{}
  \Eur{60,5533};
\end{enumerate}
il rendimento della compravendita risulta:
\begin{equation*}
  \CalcoloRendimentoPercentuale{5733,20}{4994,95} =
  \CalcoloRendimentoPercentuale{60,5533}{60,1801} = \SI{0,62}{\percent}
\end{equation*}

%page
%% ------------------------------------------------------------
%% Fine.
%% ------------------------------------------------------------

\include{fdl-1.3}

%page
%% ------------------------------------------------------------
%% Fine.
%% ------------------------------------------------------------

\end{document}

%%% end of file
% Local Variables:
% mode: latex
% TeX-master: t
% ispell-local-dictionary: "italiano"
% End:

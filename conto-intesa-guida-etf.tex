% conto-intesa-guida-etf.tex -*- coding: utf-8-unix -*-

\documentclass[12pt,a4paper]{article}
\renewcommand{\rmdefault}{ptm} % Times Roman font, for PDF output
\usepackage[italian]{babel}
\usepackage[utf8]{inputenc}
\usepackage[T1]{fontenc}
\usepackage[]{textcomp}
\usepackage[]{eurosym}
\usepackage[]{amsmath}
\usepackage[]{amsfonts}
\usepackage[]{amsthm}
\usepackage[]{amssymb}
\usepackage[]{syntonly}
\usepackage[copy-decimal-marker,retain-explicit-plus]{siunitx}
\usepackage[hidelinks]{hyperref} % To generate PDFs with hyperlinks

\pagestyle{headings}

%page
%% ------------------------------------------------------------
%% Comandi.
%% ------------------------------------------------------------

\newcommand{\Eur}[1]{\SI{#1}{\text{\euro{}}}}

\newcommand{\MediaPonderataDue}[4]{\frac{\num{#1} \times{} \num{#2} + \num{#3} \times{} \num{#4}}{\num{#1} + \num{#3}}}
\newcommand{\MediaPonderataTre}[6]{\frac{\num{#1} \times{} \num{#2} + \num{#3} \times{} \num{#4} + \num{#5} \times{} \num{#6}}{\num{#1} + \num{#3} + \num{#5}}}

\newcommand{\CostoOperazione}[1]{\num{0,50} + \num{2,50} + \num{0,0024} \times{} \num{#1}}

\newcommand{\RendimentoPercentuale}[2]{\frac{\num{#1} - \num{#2}}{#2} \times{} \num{100}}

% \everytexdraw{
%    \setgray 0 \linewd 0.01
%    \arrowheadsize l:0.12 w:0.04 \arrowheadtype t:F
% }

%page
%% ------------------------------------------------------------
%% Intestazione
%% ------------------------------------------------------------

\author{Marco Maggi}
\title{Pericolosa e incompleta guida agli investimenti in ETF con \emph{Banca Intesa Sanpaolo}}

\begin{document}

\maketitle

\begin{abstract}
  Questa guida è un  aiuto per chi volesse ricostruire i calcoli  di tenuta del Conto
  Titoli  di \emph{Banca  Intesa  Sanpaolo} per  gli  investimenti in  \emph{Exchange
     Traded Funds} (ETF).  Questa guida è incompleta:  non copre tutto ciò che c'è da
  sapere sugli ETF,  né l'operatività del sito di \emph{home  banking} di \emph{Banca
     Intesa Sanpaolo}.   Questa guida  è pericolosa:  ognuno la usa  a suo  rischio e
  pericolo.
\end{abstract}

\tableofcontents

\newpage{}

\noindent
Copyright \copyright{} 2017 Marco Maggi \texttt{<marco.maggi-ipsu@poste.it>}.

Permission is granted to copy, distribute and/or modify this document under the terms
of the GNU Free Documentation License, Version  1.3 or any later version published by
the Free Software  Foundation; with no Invariant Sections, no  Front-Cover Texts, and
no Back-Cover Texts.  A copy of the license is included in the section entitled ``GNU
Free Documentation License''.

Il codice sorgente di questa guida è disponibile in Rete all'indirizzo:
\begin{center}
  \url{https://github.com/marcomaggi/mmux-personal-finance}
\end{center}

\newpage{}

%page
\section{Introduzione}


Il calcolo dei redditi da capitale e dei redditi diversi per investimenti in ETF deve
essere eseguito come specificato nella ``Circolare dell'Agenzia dell'Entrate 21/E del
10 Luglio  2014''.  Il calcolo di  questi redditi permette sia  la determinazione dei
rendimenti che il calcolo delle tasse.

Per  ogni  operazione di  compravendita  eseguita  sul  Conto Titoli:  nella  sezione
documenti del sito  di \emph{home banking} è disponibile una  \emph{nota di eseguito}
con  tutti  i  dati  dell'operazione.   Nei calcoli  qui  illustrati:  alcuni  valori
riportati  nelle note  di eseguito  sono  i dati  di  partenza; altri  valori sono  i
risultati ricalcolati per verifica.

Questa guida  utilizza sia  la terminologia  della Banca  che quella  della circolare
dell'Agenzia dell'Entrate.  Da notare che i  prezzi di acquisto e vendita sulle Borse
di negoziazione sono chiamati \emph{effettivi} dalla circolare.

Tutti i  numeri in questa  guida sono presentati arrotondati,  il piú delle  volte al
centesimo  di  Euro;  l'esecuzione  dei  calcoli è  invece  eseguita  con  precisione
maggiore, per ridurre gli errori di arrotondamento.

%page
\section{Media aritmetica e media ponderata}


Nei  calcoli  di tenuta  Conto  Titoli  si  usa  spesso l'operazione  di  \emph{media
   aritmetica} e  in particolare  la sua riscrittura  come \emph{media  ponderata}; è
utile richiamare queste idee e alcune delle loro proprietà.

Si  ricorda che  la  \textbf{media  aritmetica} tra  i  numeri  \num{11}, \num{22}  e
\num{33} si scrive:
\begin{equation*}
  \frac{\num{11} + \num{22} + \num{33}}{3} = \num{22}
\end{equation*}
il denominatore è \num{3} perché al numeratore ci sono \num{3} numeri.

L'operazione si  costruisce nello stesso  modo se qualche  numero compare piú  di una
volta;  la media  aritmetica tra  i numeri:  \num{11}, \num{11},  \num{11}, \num{22},
\num{33} e \num{33} si scrive:
\begin{equation*}
  \frac{\num{11} + \num{11} + \num{11} + \num{22} + \num{33} + \num{33}}{6}
  = \frac{121}{6} \simeq \num{20,17}
\end{equation*}
il denominatore è \num{6} perché:
\begin{itemize}
\item \num{11} compare \num{3} volte;
\item \num{22} compare \num{1} volta;
\item \num{33} compare \num{2} volte;
\end{itemize}
quindi \(\num{3} + \num{1} + \num{2} = \num{6}\).

Il ``numero  di volte  in cui  un numero  compare'' si  chiama \textbf{molteplicità};
allora si può dire che:
\begin{itemize}
\item \num{11} compare con \emph{molteplicità} \num{3};
\item \num{22} compare con \emph{molteplicità} \num{1};
\item \num{33} compare con \emph{molteplicità} \num{2};
\end{itemize}
evidenziando le molteplicità e considerando le semplici identità:
\begin{align*}
  \num{11} + \num{11} + \num{11} &= \num{3} \times{} \num{11} &&&
  \num{22} &= \num{1} \times{} \num{22} &&&
  \num{33} + \num{33} &= \num{2} \times{} \num{33}
\end{align*}
l'espressione della \emph{media aritmetica} si può riscrivere:
\begin{align*}
  &\frac{\num{11} + \num{11} + \num{11} + \num{22} + \num{33} + \num{33}}{6}
  = \frac{\num{3} \times{} \num{11}
    + \num{1} \times{} \num{22}
    + \num{2} \times{} \num{33}}
    {6} = \\
  &=
    \frac{\num{3} \times{} \num{11}}{6} +
    \frac{\num{1} \times{} \num{22}}{6} +
    \frac{\num{2} \times{} \num{33}}{6}
  =
    \frac{\num{3}}{6} \times{} \num{11} +
    \frac{\num{1}}{6} \times{} \num{22} +
    \frac{\num{2}}{6} \times{} \num{33}
\end{align*}
in cui si evidenzia come:
\begin{itemize}
\item il numero \num{11} compaia con ``peso'' \num{3} rispetto al totale \num{6};
\item il numero \num{22} compaia con ``peso'' \num{1} rispetto al totale \num{6};
\item il numero \num{33} compaia con ``peso'' \num{2} rispetto al totale \num{6}.
\end{itemize}

In questo modo: la \emph{media aritmetica} tra i numeri \num{11}, \num{11}, \num{11},
\num{22}, \num{33} e  \num{33} è riscritta come media \textbf{media  ponderata} tra i
numeri \num{11}, \num{22} e \num{33} rispetto alle loro molteplicità \num{3}, \num{1}
e \num{2}.

Nei  calcoli di  tenuta  Conto Titoli:  si  eseguono medie  ponderate  tra prezzi  di
acquisto o vendita  di quote di fondi,  rispetto al numero di  quote acquistate; tale
numero di quote assume il ruolo di ``molteplicità di un prezzo''.

%page
\subsection{Calcolo incrementale della media ponderata}


L'attività di  investire in  un ETF  è costituita  da una  sequenza di  operazioni di
acquisto e vendita di quote del fondo; acquisti e vendite possono essere intercalati;
il  prezzo  del  totale  delle   quote  possedute  viene  calcolato  aggiornando  una
\emph{media ponderata} dopo  ogni operazione.  È utile richiamare come  il calcolo di
una \emph{media ponderata} possa essere eseguito passo dopo passo.

Si considerino  ancora i  numeri \num{11}, \num{11},  \num{11}, \num{22},  \num{33} e
\num{33}; si  è calcolato che  la loro  \emph{media aritmetica} può  essere riscritta
come \emph{media ponderata} e risulta:
\begin{equation*}
  \frac{\num{11} + \num{11} + \num{11} + \num{22} + \num{33} + \num{33}}{6}
  = \frac{\num{3} \times{} \num{11}
     + \num{1} \times{} \num{22}
     + \num{2} \times{} \num{33}}
  {6} \simeq \num{20,17}
\end{equation*}
si considerino  prima i  numeri \num{11},  \num{22} e \num{33};  poi si  aggiungano i
numeri \num{11} e \num{11}; infine si aggiunga il numero \num{33}:
\begin{enumerate}
\item la \emph{media ponderata} tra i numeri \num{11}, \num{22} e \num{33} risulta:
  \begin{equation*}
    \frac{\num{1} \times{} \num{11}
       + \num{1} \times{} \num{22}
       + \num{1} \times{} \num{33}}{\num{3}}
    = \num{22}
  \end{equation*}

\item  ora si  aggiungano  i numeri  \num{11} e  \num{11};  la \emph{media  ponderata
     aggiornata} risulta:
  \begin{equation*}
    \frac{\num{3} \times{} \num{11}
       + \num{1} \times{} \num{22}
       + \num{1} \times{} \num{33}}{\num{5}}
    = \frac{\num{88}}{\num{5}} \simeq \num{17,6}
  \end{equation*}
  e si può scrivere:
  \begin{align*}
    \frac{\num{3} \times{} \num{11}
    + \num{1} \times{} \num{22}
    + \num{1} \times{} \num{33}}{\num{5}}
    = \frac{\num{2} \times{} \num{11}}{\num{5}}
    + \frac{\num{1} \times{} \num{11}
    + \num{1} \times{} \num{22}
    + \num{1} \times{} \num{33}}{\num{5}}
  \end{align*}
  la seconda frazione al membro di destra si può riscrivere:
  \begin{align*}
    \frac{\num{3}}{\num{5}} \times \frac{\num{5}}{\num{3}} \times
    \frac{\num{1} \times{} \num{11}
    + \num{1} \times{} \num{22}
    + \num{1} \times{} \num{33}}{\num{5}}
    &= \frac{\num{3}}{\num{5}} \times
      \frac{\num{1} \times{} \num{11}
      + \num{1} \times{} \num{22}
      + \num{1} \times{} \num{33}}{\num{3}} \\
    &\simeq \frac{\num{3}}{\num{5}} \times{} \num{20,17}
      = \frac{\num{3} \times{} \num{20,17}}{\num{5}}
  \end{align*}
  e quindi la \emph{media ponderata aggiornata} risulta:
  \begin{align*}
    \frac{\num{2} \times{} \num{11}}{\num{5}}
    + \frac{\num{3} \times{} \num{20,17}}{\num{5}}
    = \frac{\num{2} \times{} \num{11} + \num{3} \times{} \num{20,17}}{\num{5}}
    \simeq \num{17,6}
  \end{align*}
  cioè la \emph{media ponderata aggiornata} è pari alla \emph{media ponderata} tra:
  \begin{itemize}
  \item la \emph{media ponderata precedente}, con la sua molteplicità;
  \item il nuovo numero aggiunto, con la sua molteplicità;
  \end{itemize}

\item  infine  si  aggiunga  il  numero  \num{33};  la  nuova  \emph{media  ponderata
     aggiornata} è la \emph{media ponderata totale} e procedendo come prima risulta:
  \begin{equation*}
    \frac{\num{1} \times{} \num{33}
       + \num{5} \times{} \num{17,6}}{\num{6}}
    \simeq \num{20,17}
  \end{equation*}
\end{enumerate}

Nei calcoli  di tenuta  Conto Titoli:  dopo ogni acquisto  di quote  di un  fondo, si
calcola  il  prezzo  medio  di   ogni  quota  posseduta  come  \emph{media  ponderata
   aggiornata} tra:
\begin{itemize}
\item  la  \emph{media ponderata  precedente},  con  il  precedente numero  di  quote
  possedute come molteplicità;
\item il prezzo delle nuove quote acquistate,  con il numero di quote acquistate come
  molteplicità.
\end{itemize}


%page
\section{Direzione dei trasferimenti di denaro}


Con  abusiva  semplificazione, le  somme  di  denaro  descritte  in questa  guida  si
ritengono scambiate tra il Conto Corrente e il Conto Titoli:
\begin{itemize}
\item le somme di denaro \emph{investite} in  quote di fondi sono in \emph{uscita dal
     Conto Corrente} e in \emph{entrata nel Conto Titoli};
\item le somme di denaro \emph{disinvestite}  da quote di fondi sono in \emph{entrata
     nel Conto Corrente} e in \emph{uscita dal Conto Titoli}.
\end{itemize}

Per esempio,  si supponga  di acquistare  quote di un  fondo per  \Eur{1000,00} tutto
incluso; questa quantità di denaro esce dal Conto Corrente ed entra nel Conto Titoli;
si considerino i due casi:
\begin{itemize}
\item si vendono le quote al prezzo netto di \Eur{1200,00}; questa quantità di denaro
  esce dal Conto Titoli ed entra nel  Conto Corrente; il bilancio delle operazioni si
  esprime con la differenza algebrica:
  \begin{equation*}
    \num{1200,00} - \num{1000,00} = \Eur{+200,00}
  \end{equation*}
  cioè, nel  saldo finale, \Eur{200,00} ``appaiono''  nel Conto Titoli, vi  escono ed
  entrano nel Conto Corrente: un guadagno;
\item si vendono le quote al prezzo  netto di \Eur{900,00}; questa quantità di denaro
  esce dal Conto Titoli ed entra nel  Conto Corrente; il bilancio delle operazioni si
  esprime con la differenza algebrica:
  \begin{equation*}
    \num{900,00} - \num{1000,00} = \Eur{-100,00}
  \end{equation*}
  cioè, nel saldo finale, \Eur{100,00} escono dal Conto Corrente ed entrano nel Conto
  Titoli in cui poi ``spariscono'': una perdita;
\end{itemize}
le differenze algebriche hanno due interpretazioni:
\begin{itemize}
\item \emph{uscite dal Conto Titoli} meno \emph{entrate nel Conto Titoli};
\item \emph{entrate nel Conto Corrente} meno \emph{uscite dal Conto Corrente}.
\end{itemize}

%page
\section{Descrizione di un'operazione di acquisto}


Scelta la strategia di  acquisto con prezzo limite: in una certa data,  e a una certa
ora, si inserisce  l'ordine di acquisto per un  numero di quote del fondo  a un certo
prezzo massimo (si può acquistare a qualsiasi prezzo al di sotto del limite fissato).
L'ordine  può essere  eseguito  in piú  fasi  in cui  solo una  parte  delle quote  è
acquistata, in  ogni fase a  un prezzo  diverso; è possibile  che non tutte  le quote
siano acquistate, nel qual caso ci interessa solo la parte ``eseguita'' dell'ordine.

Il \textbf{prezzo  medio eseguito di  un acquisto} (nella terminologia  della Banca),
anche  detto  \textbf{prezzo medio  effettivo  di  un acquisto}  (nella  terminologia
dell'Agenzia dell'Entrate), è  la media ponderata dei prezzi di  acquisto rispetto al
numero  di quote  acquistate.  Per  esempio,  se un  singolo ordine  di acquisto  per
\num{90} quote è eseguito nelle tre fasi:
\begin{enumerate}
\item acquisto di \num{20} quote al prezzo di \Eur{22,00};
\item acquisto di \num{30} quote al prezzo di \Eur{33,00};
\item acquisto di \num{40} quote al prezzo di \Eur{44,00};
\end{enumerate}
allora il \emph{prezzo medio eseguito} risulta:
\begin{equation*}
  \MediaPonderataTre{20}{22,00}{30}{33,00}{40}{44,00} = \Eur{35,44}
\end{equation*}

Il  \textbf{controvalore dell'operazione  di acquisto}  è il  prodotto tra  il numero
totale di  quote acquistate e  il \emph{prezzo medio  eseguito}.  Per esempio,  se si
acquistano  \num{90}  quote  al  \emph{prezzo   medio  eseguito}  di  \Eur{35,44}  il
controvalore dell'operazione risulta:
\begin{equation*}
  90 \times{} \num{35,44} = \Eur{3190,00}
\end{equation*}

Il  \textbf{costo  dell'operazione  di  acquisto}  è la  somma  tra  costi,  spese  e
commissioni associate all'operazione.  Per ogni  ordine di acquisto eseguito, occorre
pagare:
\begin{itemize}
\item le spese per la Banca pari a un fisso di \Eur{0,50};
\item i costi di intermediario pari a un fisso di \Eur{2,50};
\item   le   commissioni   per   la   Banca   pari   allo   \SI{0,24}{\percent}   del
  \emph{controvalore dell'operazione}.
\end{itemize}
Se un  ordine non è eseguito:  si paga nulla.  Se  un ordine è eseguito  in piú fasi:
nella prima fase si  pagano i costi fissi piú la  commissione percentuale; nelle fasi
successive  si   paga  solo   la  commissione  percentuale.    Per  esempio,   se  il
\emph{controvalore dell'operazione}  è \Eur{3189,60} il  \emph{costo dell'operazione}
risulta:
\begin{equation*}
  \CostoOperazione{3190,00} = \Eur{10,66}
\end{equation*}

Il  \textbf{controvalore totale  di un  acquisto} è  la somma  tra \emph{controvalore
   dell'operazione} e \emph{costo dell'operazione}; è la quantità di denaro in uscita
dal  Conto   Corrente  e  in   entrata  nel  Conto   Titoli.   Per  esempio,   se  il
\emph{controvalore dell'operazione} è \Eur{3189,60} e il \emph{costo dell'operazione}
è pari a \Eur{10,65}, il \emph{controvalore totale} risulta:
\begin{equation*}
  \num{3190,00} + \num{10,66} = \Eur{3200,66}
\end{equation*}

Il  \textbf{prezzo   medio  di  carico  di   un  acquisto}  é  il   rapporto  tra  il
\emph{controvalore  totale}  e  il  numero  di   quote  acquistate  ed  è  usato  per
l'aggiornamento del saldo:  è il prezzo medio di una  singola quota acquistata, tutto
incluso.  Per  esempio, se  si sono acquistate  \num{90} quote  al \emph{controvalore
   totale} di \Eur{3200,25}, il \emph{prezzo medio di carico} risulta:
\begin{equation*}
  \num{3200,66} / \num{90} = \Eur{35,56}
\end{equation*}

%page
\section{Aggiornamento del saldo dopo un'operazione di acquisto}


Dopo l'esecuzione  di ogni operazione di  acquisto occorre ricalcolare il  saldo.  Il
saldo di un investimento in quote di ETF è rappresentato dalle tre quantità:
\begin{itemize}
\item numero di quote in carico nel Conto Titoli;
\item \textbf{prezzo  medio di \underline{carico} nel  saldo}: è il prezzo  medio per
  quota acquistata, tutto incluso; si può calcolare come:
  \begin{itemize}
  \item media  ponderata tra  tutti i \emph{prezzi  medi di  \underline{carico} degli
       acquisti};
  \item media ponderata  tra il \emph{prezzo medio di \underline{carico}  di un nuovo
       acquisto} e il precedente \emph{prezzo medio di \underline{carico} nel saldo};
  \end{itemize}
\item \textbf{prezzo medio  \underline{effettivo} nel saldo}: è il  prezzo medio, per
  quota  acquistata,  degli ordini  eseguiti  sulla  Borsa  di negoziazione;  si  può
  calcolare come:
  \begin{itemize}
  \item media  ponderata tra  tutti i  \emph{prezzi medi  \underline{effettivi} degli
       acquisti};
  \item media ponderata  tra il \emph{prezzo medio \underline{effettivo}  di un nuovo
       acquisto} e il precedente \emph{prezzo medio \underline{effettivo} nel saldo};
  \end{itemize}
  nella documentazione della  Banca: questo valore è chiamato  \emph{Net Asset Value}
  (NAV)  del prezzo  medio di  carico  (da non  confondere con  il NAV  dell'attività
  sottostante il fondo).
\end{itemize}

Per esempio, si supponga di eseguire le operazioni:
\begin{enumerate}
\item acquisto di \num{100} quote al \emph{prezzo medio effettivo} di \Eur{110,00};
\item acquisto di \num{200} quote al \emph{prezzo medio effettivo} di \Eur{120,00};
\item acquisto di \num{300} quote al \emph{prezzo medio effettivo} di \Eur{130,00}.
\end{enumerate}
all'inizio si  consideri un Conto Titoli  vuoto, con saldo convenzionale  di: \num{0}
quote; \emph{prezzo  medio di  carico} di \Eur{0};  \emph{prezzo medio  effettivo} di
\Eur{0}.  L'aggiornamento del saldo avviene come segue.
\begin{enumerate}
\item Per  il primo  acquisto, \num{100}  quote al  \emph{prezzo medio  effettivo} di
  \Eur{110,00}, risulta:
  \begin{itemize}
  \item controvalore dell'operazione: \Eur{11000,00};
  \item costo dell'operazione: \Eur{29,40};
  \item controvalore totale: \Eur{11029,40};
  \item prezzo medio di carico: \Eur{110,29}.
  \end{itemize}

  Dopo il primo acquisto:
  \begin{itemize}
  \item  il numero  quote nel  saldo è  uguale al  numero quote  del primo  acquisto:
    \num{100};
  \item il  \emph{prezzo medio carico  nel saldo} è  uguale al \emph{prezzo  medio di
       carico} del primo acquisto: \Eur{110,29};
  \item il  \emph{prezzo medio effettivo  nel saldo}  è uguale al  \emph{prezzo medio
       effettivo} del primo acquisto: \Eur{110,00}.
  \end{itemize}

\item Per  il secondo acquisto, \num{200}  quote al \emph{prezzo medio  effettivo} di
  \Eur{120,00}, risulta:
  \begin{itemize}
    \item controvalore dell'operazione: \Eur{24000,00};
    \item costo dell'operazione: \Eur{60,60};
    \item controvalore totale: \Eur{24060,60};
    \item prezzo medio di carico: \Eur{120,30}.
  \end{itemize}

  Dopo il secondo acquisto:
  \begin{itemize}
    \item numero quote nel saldo: \(100 + 200 = 300\);
    \item il \emph{prezzo medio di carico carico  nel saldo} è la media ponderata dei
      \emph{prezzi medi di carico} del saldo precedente e del nuovo acquisto:
      \begin{equation*}
        \MediaPonderataDue{100}{110,29}{200}{120,30} = \Eur{116,97}
      \end{equation*}
    \item  il \emph{prezzo  medio  effettivo  nel saldo}  è  la  media ponderata  dei
      \emph{prezzi medi effettivi} del saldo precedente e del nuovo acquisto:
      \begin{equation*}
        \MediaPonderataDue{100}{110,00}{200}{120,00} = \Eur{116,67}
      \end{equation*}
  \end{itemize}

\item Per  il terzo  acquisto, \num{300}  quote al  \emph{prezzo medio  effettivo} di
  \Eur{130,00}, risulta:
  \begin{itemize}
  \item controvalore dell'operazione: \Eur{39000,00};
  \item costo dell'operazione: \Eur{96,60};
  \item controvalore totale: \Eur{39096,60};
  \item prezzo medio di carico: \Eur{130,32}.
  \end{itemize}

  Dopo il terzo acquisto:
  \begin{itemize}
    \item numero quote nel saldo: \(300 + 300 = 600\);
    \item il \emph{prezzo medio di carico carico  nel saldo} è la media ponderata dei
      \emph{prezzi medi di carico} del saldo precedente e del nuovo acquisto:
      \begin{equation*}
        \MediaPonderataDue{300}{116,97}{300}{130,32} = \Eur{123,64}
      \end{equation*}
    \item  il \emph{prezzo  medio  effettivo  nel saldo}  è  la  media ponderata  dei
      \emph{prezzi medi effettivi} del saldo precedente e del nuovo acquisto:
      \begin{equation*}
        \MediaPonderataDue{300}{116,67}{300}{130,00} = \Eur{123,33}
      \end{equation*}
  \end{itemize}
\end{enumerate}

Evidenziando il \emph{prezzo medio di carico  nel saldo}: ogni nuova quota acquistata
viene ``buttata nel secchio'' insieme  alle vecchie, indipendentemente dal suo prezzo
di acquisto; è come se ogni quota  fosse stata acquistata allo stesso prezzo, pari al
\emph{prezzo medio di carico nel saldo}.

Si osservi come il costo totale di tutte le operazioni di acquisto si possa calcolare
come somma di tutti i costi dei singoli acquisti:
\begin{align*}
  \num{29,40} + \num{60,60} + \num{96,60} = \Eur{186,60}
\end{align*}
ma anche a partire dalla differenza tra il \emph{prezzo medio di carico nel saldo} di
\Eur{123,64} e il \emph{prezzo medio effettivo nel saldo} di \Eur{123,33}:
\begin{equation*}
  \num{600} \times{} (\num{123,64} - \num{123,33}) = \Eur{186,60}
\end{equation*}
in    cui    si     è    evidenziato    il    \emph{costo     medio    per    quota}:
\(\num{123,64} - \num{123,33} = \Eur{0,31}\).

%page
\section{Descrizione di un'operazione di vendita}


Scelta la strategia  di vendita con prezzo limite:  in una certa data, e  a una certa
ora, si  inserisce l'ordine di vendita  per un numero di  quote del fondo a  un certo
prezzo minimo  (si può vendere  a qualsiasi prezzo al  di sopra del  limite fissato).
L'ordine può essere eseguito in piú fasi in cui solo una parte delle quote è venduta,
in ogni fase a  un prezzo diverso; è possibile che non tutte  le quote siano vendute,
nel qual caso ci interessa solo la parte ``eseguita'' dell'ordine.

Il \textbf{prezzo  medio eseguito di  una vendita} (nella terminologia  della Banca),
anche  detto   \textbf{prezzo  medio   effettivo  di  vendita}   (nella  terminologia
dell'Agenzia dell'Entrate),  è la media ponderata  dei prezzi di vendita  rispetto al
numero di quote vendute.   Per esempio, se un singolo ordine  di vendita per \num{90}
quote è eseguito nelle fasi:
\begin{enumerate}
\item vendita di \num{20} quote al prezzo di \Eur{102,00};
\item vendita di \num{30} quote al prezzo di \Eur{103,00};
\item vendita di \num{40} quote al prezzo di \Eur{104,00};
\end{enumerate}
allora il \emph{prezzo medio effettivo} risulta:
\begin{equation*}
  \MediaPonderataTre{20}{102,00}{30}{103,00}{40}{104,00} = \Eur{103,22}
\end{equation*}

Il  \textbf{controvalore dell'operazione  di vendita}  è  il prodotto  tra il  numero
totale  di quote  vendute e  il \emph{prezzo  medio effettivo}.   Per esempio,  se si
vendono  \num{90}  quote   al  \emph{prezzo  medio  effettivo}   di  \Eur{103,22}  il
controvalore dell'operazione risulta:
\begin{equation*}
  90 \times{} \num{103,22} = \Eur{9290,00}
\end{equation*}

Per ogni vendita occorre pagare le \textbf{tasse sul reddito da capitale}\footnote{Si
   deve pagare  anche il bollo sull'estratto  conto trimestrale per il  Conto Titoli,
   pari  allo \SI{0,2}{\percent}  all'anno; il  bollo è  addebitato direttamente  sul
   Conto Corrente,  perciò in  questa guida  lo si  considera conteggiato  a parte.}.
Secondo la circolare  dell'Agenzia dell'Entrate, il reddito da capitale  per quota si
calcola  come   differenza  tra  \emph{prezzo   medio  effettivo  della   vendita}  e
\emph{prezzo medio effettivo  nel saldo}.  Per esempio, se si  vendono \num{90} quote
al  \emph{prezzo  medio  effettivo}  di   \Eur{103,22}  e  nel  saldo  precedente  il
\emph{prezzo medio effettivo} è di \Eur{100,00}, il reddito da capitale risulta:
\begin{equation*}
  \num{90} \times{} (\num{103,22} - \num{100,00}) = \Eur{290,00}
\end{equation*}
per fondi  che non contengono Titoli  dello Stato Italiano, e  assimilati, l'aliquota
unica  è  del \SI{26}{\percent};  allora  la  \emph{tassa  sul reddito  da  capitale}
risulta:
\begin{equation*}
  \num{0,26} \times{} \num{290,00} = \Eur{75,40}
\end{equation*}

Peculiarmente, la circolare dell'Agenzia dell'Entrate specifica che:
\begin{itemize}
\item quando il reddito da capitale è \textbf{positivo}: esso \textbf{non genera} una
  plusvalenza utilizzabile per compensare minusvalenze da altri investimenti;
\item quando  il reddito  da capitale è  \textbf{negativo}: esso  \textbf{genera} una
  minusvalenza compensabile con plusvalenze da altri investimenti.
\end{itemize}

Il  \textbf{costo  dell'operazione  di  vendita}  è  la  somma  tra  costi,  spese  e
commissioni associate  all'operazione; per ogni  ordine di vendita  eseguito, occorre
pagare:
\begin{itemize}
\item le spese per la Banca pari a un fisso di \Eur{0,50};
\item i costi di intermediario pari a un fisso di \Eur{2,50};
\item   le   commissioni   per   la   Banca   pari   allo   \SI{0,24}{\percent}   del
  \emph{controvalore dell'operazione};
\end{itemize}
se un  ordine non è  eseguito: si paga  nulla; se un ordine  è eseguito in  piú fasi:
nella prima fase si  pagano i costi fissi piú la  commissione percentuale; nelle fasi
successive  si   paga  solo   la  commissione  percentuale.    Per  esempio,   se  il
\emph{controvalore dell'operazione}  è \Eur{9290,00} il  \emph{costo dell'operazione}
risulta:
\begin{equation*}
  \CostoOperazione{9290,00} = \Eur{25,30}
\end{equation*}

Da osservare  che: sia il  calcolo delle \emph{tasse sul  reddito da capitale}  che i
calcoli del \emph{costo dell'operazione} di acquisto  e vendita si eseguono usando il
\emph{prezzo medio effettivo};  in pratica: si pagano le tasse  anche sul \emph{costo
   dell'operazione} di  acquisto e vendita, che  però sono redditi di  qualcun altro,
non dell'investitore!

La circolare dell'Agenzia dell'Entrate specifica che il \emph{costo delle operazioni}
origina minusvalenze da registrare come redditi diversi.  Per esempio, si supponga di
possedere \num{90} quote e di venderle tutte con:
\begin{itemize}
\item \emph{prezzo medio effettivo nel saldo precedente}: \Eur{100,00};
\item \emph{prezzo medio di carico nel saldo precedente}: \Eur{100,27};
\item \emph{prezzo medio effettivo di vendita} \Eur{103,22};
\item reddito da capitale: \Eur{290,00};
\item \emph{costo dell'operazione di vendita}: \Eur{25,30};
\item il costo  medio delle operazioni di  acquisto per le quote  vendute, si calcola
  dalla differenza  tra il \emph{prezzo  medio di carico  nel saldo} precedente  e il
  \emph{prezzo medio effettivo nel saldo} precedente:
  \begin{equation*}
    \num{90} \times{} (\num{100,27} - \num{100,00}) = \Eur{24,30}
  \end{equation*}
\end{itemize}
allo scopo di far risultare un numero negativo, la circolare specifica che il reddito
diverso associato alla vendita deve essere calcolato come:
\begin{equation*}
  \left[\num{290,00} - \left(\num{25,30} + \num{24,30}\right)\right] - \num{290,00}
  = - (\num{25,30} + \num{24,30}) = \Eur{-49,60}
\end{equation*}
questa minusvalenza dovrebbe essere registrata  nel \emph{cassetto fiscale} del Conto
Titoli.

Il \textbf{controvalore totale di vendita}  è la differenza tra il \emph{controvalore
   dell'operazione} e  la somma  tra \emph{costo  dell'operazione} e  \emph{tasse sul
   reddito da capitale}; è  la quantità di denaro in entrata nel  Conto Corrente e in
uscita dal  Conto Titoli.  Per  esempio, se il \emph{controvalore  dell'operazione} è
\Eur{9290,00}, il  \emph{costo dell'operazione}  è \Eur{25,30}  e la  \emph{tassa sul
   reddito da capitale} è \Eur{75,40}, il \emph{controvalore totale} risulta:
\begin{equation*}
  \num{9290,00} - \num{25,30} - \num{75,40} = \Eur{9189,30}
\end{equation*}

Per un'operazione di  vendita: \textbf{non esiste} un \emph{prezzo  medio di carico};
la vendita modifica solo il numero di quote in carico nel saldo.  Invece si definisce
il \textbf{prezzo medio netto di vendita}  pari al rapporto tra il \emph{controvalore
   totale} e il  numero di quote vendute.   Per esempio, se si  sono vendute \num{90}
quote al \emph{controvalore totale} di  \Eur{9189,30}, il \emph{prezzo medio netto di
   vendita} risulta:
\begin{equation*}
  \num{9189,30} / \num{90} = \Eur{102,10}
\end{equation*}

%page
\section{Aggiornamento del saldo dopo un'operazione di vendita}


Dopo l'esecuzione  di ogni operazione  di vendita  occorre ricalcolare il  saldo.  Il
saldo di un investimento in quote di ETF è rappresentato dalle tre quantità:
\begin{itemize}
\item numero di quote in carico nel Conto Titoli;
\item \emph{prezzo medio di carico nel saldo};
\item \emph{prezzo medio effettivo nel saldo};
\end{itemize}
sia il  \emph{prezzo medio di carico  nel saldo} che il  \emph{prezzo medio effettivo
   nel  saldo} restano  \textbf{immutati} dopo  una  vendita: essa  modifica solo  il
numero di quote in carico di minuendolo del numero di quote vendute.

Per esempio, si supponga di avere un Conto Titoli con saldo iniziale:
\begin{itemize}
\item numero di quote \num{100};
\item \emph{prezzo medio effettivo nel saldo}: \Eur{100,00};
\item \emph{prezzo medio di carico nel saldo}: \Eur{100,27};
\end{itemize}
e di venderne \num{90}, non importa a quale prezzo; il saldo finale risulta:
\begin{itemize}
\item numero di quote \(\num{100} - \num{90} = \num{10}\);
\item \emph{prezzo medio effettivo nel saldo}: \Eur{100,00}, immutato;
\item \emph{prezzo medio di carico nel saldo}: \Eur{100,27}, immutato.
\end{itemize}

%page
\section{Rendimento indicato nel riepilogo del patrimonio}


Nel sito di \emph{home banking} è indicato, per ogni fondo nel Conto Titoli, il saldo
costituito dal numero di quote in carico  e dal \emph{prezzo medio di carico}; in piú
è visibile  la stima dell'\emph{Utile  o Perdita}  (U/P), nell'ipotesi di  vendita di
tutte le quote al prezzo dell'ultima  quotazione rilevata sulla Borsa di negoziazione
del titolo:
\begin{itemize}
\item  l'\emph{Utile} mostrato  è al  lordo sia  del \emph{costo  dell'operazione} di
  vendita che della tassa sul reddito da capitale;
\item la \emph{Perdita} mostrata è al lordo del costo \emph{costo dell'operazione} di
  vendita.
\end{itemize}

Si supponga di  possedere \num{100} quote al \emph{prezzo medio  di carico nel saldo}
di \Eur{100,27}:
\begin{itemize}
\item  se l'ultimo  rilevamento per  il prezzo  di una  quota fosse  di \Eur{105,00},
  l'\emph{Utile percentuale} mostrato nella tabella del patrimonio risulterebbe:
  \begin{equation*}
    \RendimentoPercentuale{105,00}{100,27} = \SI{4,7173}{\percent}
  \end{equation*}
  mentre l'\emph{Utile in valuta} risulterebbe:
  \begin{equation*}
    \num{100} \times{} (\num{105,00} - \num{100,27}) = \Eur{473,00}
  \end{equation*}
  in  realtà,   vendendo  tutte  le   quote  al  \emph{prezzo  medio   effettivo}  di
  \Eur{105,00}, risulta:
  \begin{itemize}
  \item prezzo medio netto: \Eur{103,42};
  \item rendimento percentuale: \SI{3,1395}{\percent};
  \item rendimento in valuta: \Eur{314,80};
  \end{itemize}

\item se  l'ultimo rilevamento per  il prezzo di una  quota fosse di  \Eur{98,00}, la
  \emph{Perdita percentuale} mostrata nella tabella del patrimonio risulterebbe:
  \begin{equation*}
    \RendimentoPercentuale{98,00}{100,27} = \SI{-2,2639}{\percent}
  \end{equation*}
  mentre la \emph{Perdita in valuta} risulterebbe:
  \begin{equation*}
    \num{100} \times{} (\num{98,00} - \num{100,27}) = \Eur{-227,00}
  \end{equation*}
  in  realtà,   vendendo  tutte  le   quote  al  \emph{prezzo  medio   effettivo}  di
  \Eur{98,00}, risulta:
  \begin{itemize}
  \item prezzo medio netto: \Eur{97,73};
  \item perdita percentuale: \SI{-2,5284}{\percent};
  \item perdita in valuta: \Eur{-253,52}.
  \end{itemize}
\end{itemize}

%page
\section{Strategia per la scelta dei prezzi di vendita}


In generale  si cerca di  vendere a  un prezzo maggiore  del prezzo di  acquisto.  Se
occorre  liquidità: si  potrebbe essere  costretti a  disinvestire vendendo  a prezzi
inferiori.

%page
\subsection{Accoppiamento diretto tra acquisti e vendite}

Si cerca  di vendere  ogni singola quota  a un \emph{prezzo  medio netto  di vendita}
maggiore del suo \emph{prezzo medio di carico di acquisto}.  Per esempio, si supponga
di eseguire le operazioni:
\begin{enumerate}
\item acquisto di \num{20} quote al \emph{prezzo  medio di carico di acquisto} pari a
  \Eur{22,00};
\item acquisto di \num{30} quote al \emph{prezzo  medio di carico di acquisto} pari a
  \Eur{33,00};
\item acquisto di \num{40} quote al \emph{prezzo  medio di carico di acquisto} pari a
  \Eur{44,00};
\item vendita  di \num{20}  quote a un  \emph{prezzo medio netto  di vendita}  pari a
  \Eur{28,00};
\item vendita  di \num{30}  quote a un  \emph{prezzo medio netto  di vendita}  pari a
  \Eur{38,00};
\item vendita  di \num{40}  quote a un  \emph{prezzo medio netto  di vendita}  pari a
  \Eur{48,00};
\end{enumerate}
per ogni gruppo di quote acquistate e poi vendute si realizzano i guadagni:
\begin{itemize}
\item       accoppiando       le       operazioni      \num{1}       e       \num{4}:
  \(\num{20} \times (\num{28,00} - \num{22,00}) = \Eur{120,00}\)
\item       accoppiando       le       operazioni      \num{2}       e       \num{5}:
  \(\num{30} \times (\num{38,00} - \num{33,00}) = \Eur{150,00}\);
\item       accoppiando       le       operazioni      \num{3}       e       \num{6}:
  \(\num{40} \times (\num{48,00} - \num{44,00}) = \Eur{160,00}\);
\end{itemize}
in  cui i  risultati  rappresentano uscite  dal  Conto Titoli  ed  entrate nel  Conto
Corrente.  Quindi il rendimento totale in valuta risulta:
\begin{align*}
  \num{120,00} + \num{150,00} + \num{160,00} = \Eur{430,00}
\end{align*}
e in percentuale:
\begin{align*}
  100 \times{} \frac{\num{430,00}}
  {20 \times{} \num{22,00} + 30 \times{} \num{33,00} + \num{40} \times{} \num{44,00}}
  = \SI{13,48}{\percent}
\end{align*}
Siccome  si sono  eseguiti  prima tutti  gli  acquisti  e poi  tutte  le vendite,  il
\emph{prezzo medio di carico nel saldo} risulta:
\begin{equation*}
  \frac{20 \times{} \num{22,00} + 30 \times{} \num{33,00} + 40 \times{} \num{44,00}}
  {20 + 30 + 40} = \Eur{35,44}
\end{equation*}
la media ponderata dei \emph{prezzi medi netti di vendita} risulta:
\begin{equation*}
  \frac{20 \times{} \num{28,00} + 30 \times{} \num{38,00} + 40 \times{} \num{48,00}}
  {20 + 30 + 40} = \Eur{40,22}
\end{equation*}
quindi il rendimento totale percentuale si può calcolare anche come:
\begin{align*}
  100 \times{} \frac{\num{40,22} - \num{35,44}}{35,44} = \SI{13,48}{\percent}
\end{align*}
e in valuta:
\begin{align*}
  (20 + 30 + 40) \times{} (\num{40,22} - \num{35,44}) = \Eur{430,20}
\end{align*}

%page
\subsection{Metodo delle medie ponderate}

Si cerca  di vendere le quote  a un prezzo  talmente maggiore del prezzo  di acquisto
che, recuperati i costi e pagate le tasse, si realizza un guadagno.
\begin{itemize}
\item Se il \emph{prezzo medio netto di vendita} è maggiore del \emph{prezzo medio di
     carico nel saldo} in  quel momento: si recuperano i costi  di acquisto e vendita
  e, dopo aver pagato le tasse, si realizza un guadagno.

\item Se il  \emph{prezzo medio netto di  vendita} è uguale al  \emph{prezzo medio di
     carico nel saldo}: l'operazione di vendita recupera i costi ma, dopo aver pagato
  le tasse, resta nessun guadagno.

\item Se il \emph{prezzo  medio netto di vendita} è minore  del \emph{prezzo medio di
     carico nel saldo}: tolti i costi  ed eventualmente le tasse, l'operazione genera
  una perdita.
\end{itemize}

Come esempio  di guadagno,  si posseggano  \num{100} quote  al \emph{prezzo  medio di
   carico nel saldo} di \Eur{88,00} con \emph{prezzo medio effettivo} di \Eur{87,00};
si vendano \num{50} quote al \emph{prezzo medio eseguito} di \Eur{95,00}; risulta:
\begin{itemize}
\item \emph{controvalore dell'operazione}: \(\num{50} \times{} \num{95,00} = \Eur{4750}\);
\item \emph{costo dell'operazione di vendita}: \(\num{0,50} + \num{2,50} + \num{0,0024} \times{} \num{4750} = \Eur{14,40}\);
\item \emph{tasse sul reddito da capitale}: \(\num{0,26} \times{} \num{50} \times{} (\num{95,00} - \num{87,00}) = \Eur{104,00}\);
\item \emph{controvalore totale di vendita}: \(\num{4750} - \num{104,00} - \num{14,40} = \Eur{4631,60}\);
\item \emph{prezzo medio netto di vendita}: \(\num{4631,60} / \num{50} = \Eur{92,632}\);
\end{itemize}
quindi un rendimento percentuale di:
\begin{equation*}
  100 \times{} \frac{\num{92,632} - \num{88,00}}{\num{88,00}} = \SI{+5,26}{\percent}
\end{equation*}

Come  esempio di  perdita, si  posseggano \num{100}  quote al  \emph{prezzo medio  di
   carico nel saldo} di \Eur{88,00} con \emph{prezzo medio effettivo} di \Eur{87,00};
si vendano \num{50} quote al \emph{prezzo medio eseguito} di \Eur{88,20}; risulta:
\begin{itemize}
\item \emph{controvalore dell'operazione}: \(\num{50} \times{} \num{88,20} = \Eur{4410,00}\);
\item \emph{costo dell'operazione di vendita}: \(\num{0,50} + \num{2,50} + \num{0,0024} \times{} \num{4410,00} = \Eur{13,58}\);
\item \emph{tasse sul reddito da capitale}: \(\num{0,26} \times{} \num{50} \times{} (\num{88,20} - \num{87,00}) = \Eur{15,60}\);
\item \emph{controvalore totale di vendita}: \(\num{4410,00} - \num{15,60} - \num{13,58} = \Eur{4380,82}\);
\item \emph{prezzo medio netto di vendita}: \(\num{4380,82} / \num{50} = \Eur{87,62}\);
\end{itemize}
quindi una perdita percentuale di:
\begin{equation*}
  100 \times{} \frac{\num{87,62} - \num{88,00}}{\num{88,00}} = \SI{-0.43}{\percent}
\end{equation*}
nonostante il  \emph{prezzo medio  eseguito} sia maggiore  del \emph{prezzo  medio di
   carico nel  saldo}: l'operazione può  andare in  perdita se il  \emph{prezzo medio
   netto di vendita} risulta minore.

Se si  considerano operazioni di vendita  che liquidano tutte le  quote possedute: si
può cercare di recuperare una vendita in perdita con altre in guadagno.  Per esempio,
si supponga di eseguire le operazioni:
\begin{enumerate}
\item acquisto di \num{20} quote al \emph{prezzo  medio di carico di acquisto} pari a
  \Eur{22,00};
\item acquisto di \num{30} quote al \emph{prezzo  medio di carico di acquisto} pari a
  \Eur{33,00};
\item acquisto di \num{40} quote al \emph{prezzo  medio di carico di acquisto} pari a
  \Eur{44,00};
\item vendita  di \num{40}  quote a un  \emph{prezzo medio netto  di vendita}  pari a
  \Eur{33,00};
\item vendita  di \num{50}  quote a un  \emph{prezzo medio netto  di vendita}  pari a
  \Eur{38,50};
\end{enumerate}
avendo eseguito  prima tutti  gli acquisti  e poi tutte  le vendite,  il \emph{prezzo
   medio di carico nel saldo} si calcola con la media ponderata:
\begin{equation*}
  \frac{\num{20} \times{} \num{22,00}
     + \num{30} \times{} \num{33,00}
     + \num{40} \times{} \num{44,00}}
  {\num{20} + \num{30} + \num{40}} = \Eur{35,44}
\end{equation*}
l'operazione  di vendita  al prezzo  di \Eur{33,00}  è in  perdita, mentre  quella al
prezzo di \Eur{38,50}  è in guadagno; la media ponderata  dei \emph{prezzi medi netti
   di vendita} risulta:
\begin{equation*}
  \frac{\num{40} \times{} \num{33,00} + \num{50} \times{} \num{38,50}}
  {40 + 50}
  = \Eur{36,06}
\end{equation*}
maggiore del \emph{prezzo medio di carico nel saldo} e quindi, alla fine, si realizza
un guadagno di:
\begin{equation*}
  \num{90}  \times{} (\num{36,06}  -  \num{35,44}) =  \Eur{55,80}
\end{equation*}

La  domanda  fondamentale  è:  \textbf{possedendo  un numero  di  quote  a  un  certo
   \emph{prezzo medio di carico nel saldo}  e un certo \emph{prezzo medio effettivo},
   qual'è  il \emph{prezzo  medio  eseguito di  vendita} che  permette  di andare  in
   guadagno?}

Per esempio, si posseggano \num{50} quote  al \emph{prezzo medio di carico nel saldo}
di  \Eur{88,00}  con  \emph{  prezzo   medio  effettivo}  di  \Eur{87,00};  si  vuole
determinare  il \emph{prezzo  medio  eseguito  di vendita}  per  tutte  le quote  che
permette il guadagno.  Detto \(X\) tale valore, risulta:
\begin{itemize}
\item \emph{controvalore dell'operazione}: \(\num{50} \times{} X\);
\item \emph{costo dell'operazione di vendita}:
  \begin{align*}
    \num{0,50} + \num{2,50} + \num{0,0024} \times{} (\num{50} \times{} X)
    &= \num{3,00} + \num{0,0024} \times{} \num{50} \times{} X = \\
    &= \num{3,00} + \num{0,12} \times{} X
  \end{align*}
\item \emph{tasse sul reddito da capitale}:
  \begin{align*}
    \num{0,26} \times{} \num{50} \times{} (X - \num{87,00})
    &= \num{13} \times{} X - \num{13} \times{} \num{87,00} = \\
    &= \num{13} \times{} X - \num{1131,00}
  \end{align*}
\item \emph{controvalore totale di vendita}:
  \begin{align*}
    (\num{50} \times{} X)
    &- (\num{3,00} + \num{0,12} \times{} X)
      - (\num{13} \times{} X - \num{1131,00})
      = \\
    &= \num{50} \times{} X
      - \num{3,00} - \num{0,12} \times{} X
      - \num{13} \times{} X + \num{1131,00}
      = \\
    &= (\num{50} - \num{0,12} - \num{13}) \times{} X + (\num{1131,00} - \num{3,00})
      = \\
    &= \num{36,88} \times{} X + \num{1131,00}
  \end{align*}
\item \emph{prezzo medio netto di vendita}:
  \begin{equation*}
    \frac{\num{36,88} \times{} X + \num{1131,00}}{50}
  \end{equation*}
\end{itemize}
il pareggio si raggiungerebbe vendendo le quote a un \emph{prezzo medio netto} uguale
al \emph{prezzo medio di carico nel saldo} di \Eur{88,00}; quindi:
\begin{equation*}
  \frac{\num{36,88} \times{} X + \num{1131,00}}{50} = \num{88,00}
\end{equation*}
risolvendo rispetto a \(X\):
\begin{equation*}
  X = \frac{\num{50} \times{} \num{88,00} - \num{1131,00}}{\num{36,88}}
  = \Eur{88,64}
\end{equation*}
vendendo con un  \emph{prezzo medio eseguito} maggiore di \Eur{88,64}  si realizza un
guadagno.

%page
%% ------------------------------------------------------------
%% Fine.
%% ------------------------------------------------------------

%%% fdl-1.3.tex --

\appendix

\section{\rlap{GNU Free Documentation License}}

\begin{center}
  Version 1.3, 3 November 2008


  Copyright \copyright{} 2000, 2001, 2002, 2007, 2008  Free Software Foundation, Inc.

  \bigskip

  \texttt{<https://fsf.org/>}

  \bigskip

  Everyone is permitted  to copy and distribute verbatim  copies of this
  license document, but changing it is not allowed.
\end{center}


\begin{center}
  {\bf\large Preamble}
\end{center}

The purpose of this License is to make a manual, textbook, or other
functional and useful document ``free'' in the sense of freedom: to
assure everyone the effective freedom to copy and redistribute it,
with or without modifying it, either commercially or noncommercially.
Secondarily, this License preserves for the author and publisher a way
to get credit for their work, while not being considered responsible
for modifications made by others.

This License is a kind of ``copyleft'', which means that derivative
works of the document must themselves be free in the same sense.  It
complements the GNU General Public License, which is a copyleft
license designed for free software.

We have designed this License in order to use it for manuals for free
software, because free software needs free documentation: a free
program should come with manuals providing the same freedoms that the
software does.  But this License is not limited to software manuals;
it can be used for any textual work, regardless of subject matter or
whether it is published as a printed book.  We recommend this License
principally for works whose purpose is instruction or reference.


\subsection{APPLICABILITY AND DEFINITIONS}

% \begin{center}
%   {\Large\bf 1. APPLICABILITY AND DEFINITIONS\par}
%   \phantomsection
%   \addcontentsline{toc}{section}{1. APPLICABILITY AND DEFINITIONS}
% \end{center}

This License applies to any manual or other work, in any medium, that
contains a notice placed by the copyright holder saying it can be
distributed under the terms of this License.  Such a notice grants a
world-wide, royalty-free license, unlimited in duration, to use that
work under the conditions stated herein.  The ``\textbf{Document}'', below,
refers to any such manual or work.  Any member of the public is a
licensee, and is addressed as ``\textbf{you}''.  You accept the license if you
copy, modify or distribute the work in a way requiring permission
under copyright law.

A ``\textbf{Modified Version}'' of the Document means any work containing the
Document or a portion of it, either copied verbatim, or with
modifications and/or translated into another language.

A ``\textbf{Secondary Section}'' is a named appendix or a front-matter section of
the Document that deals exclusively with the relationship of the
publishers or authors of the Document to the Document's overall subject
(or to related matters) and contains nothing that could fall directly
within that overall subject.  (Thus, if the Document is in part a
textbook of mathematics, a Secondary Section may not explain any
mathematics.)  The relationship could be a matter of historical
connection with the subject or with related matters, or of legal,
commercial, philosophical, ethical or political position regarding
them.

The ``\textbf{Invariant Sections}'' are certain Secondary Sections whose titles
are designated, as being those of Invariant Sections, in the notice
that says that the Document is released under this License.  If a
section does not fit the above definition of Secondary then it is not
allowed to be designated as Invariant.  The Document may contain zero
Invariant Sections.  If the Document does not identify any Invariant
Sections then there are none.

The ``\textbf{Cover Texts}'' are certain short passages of text that are listed,
as Front-Cover Texts or Back-Cover Texts, in the notice that says that
the Document is released under this License.  A Front-Cover Text may
be at most 5 words, and a Back-Cover Text may be at most 25 words.

A ``\textbf{Transparent}'' copy of the Document means a machine-readable copy,
represented in a format whose specification is available to the
general public, that is suitable for revising the document
straightforwardly with generic text editors or (for images composed of
pixels) generic paint programs or (for drawings) some widely available
drawing editor, and that is suitable for input to text formatters or
for automatic translation to a variety of formats suitable for input
to text formatters.  A copy made in an otherwise Transparent file
format whose markup, or absence of markup, has been arranged to thwart
or discourage subsequent modification by readers is not Transparent.
An image format is not Transparent if used for any substantial amount
of text.  A copy that is not ``Transparent'' is called ``\textbf{Opaque}''.

Examples of suitable formats for Transparent copies include plain
ASCII without markup, Texinfo input format, LaTeX input format, SGML
or XML using a publicly available DTD, and standard-conforming simple
HTML, PostScript or PDF designed for human modification.  Examples of
transparent image formats include PNG, XCF and JPG.  Opaque formats
include proprietary formats that can be read and edited only by
proprietary word processors, SGML or XML for which the DTD and/or
processing tools are not generally available, and the
machine-generated HTML, PostScript or PDF produced by some word
processors for output purposes only.

The ``\textbf{Title Page}'' means, for a printed book, the title page itself,
plus such following pages as are needed to hold, legibly, the material
this License requires to appear in the title page.  For works in
formats which do not have any title page as such, ``Title Page'' means
the text near the most prominent appearance of the work's title,
preceding the beginning of the body of the text.

The ``\textbf{publisher}'' means any person or entity that distributes
copies of the Document to the public.

A section ``\textbf{Entitled XYZ}'' means a named subunit of the Document whose
title either is precisely XYZ or contains XYZ in parentheses following
text that translates XYZ in another language.  (Here XYZ stands for a
specific section name mentioned below, such as ``\textbf{Acknowledgements}'',
``\textbf{Dedications}'', ``\textbf{Endorsements}'', or ``\textbf{History}''.)
To ``\textbf{Preserve the Title}''
of such a section when you modify the Document means that it remains a
section ``Entitled XYZ'' according to this definition.

The Document may include Warranty Disclaimers next to the notice which
states that this License applies to the Document.  These Warranty
Disclaimers are considered to be included by reference in this
License, but only as regards disclaiming warranties: any other
implication that these Warranty Disclaimers may have is void and has
no effect on the meaning of this License.


\subsection{VERBATIM COPYING}
% \begin{center}
%   {\Large\bf 2. VERBATIM COPYING\par}
%   \phantomsection
%   \addcontentsline{toc}{section}{2. VERBATIM COPYING}
% \end{center}

You may copy and distribute the Document in any medium, either
commercially or noncommercially, provided that this License, the
copyright notices, and the license notice saying this License applies
to the Document are reproduced in all copies, and that you add no other
conditions whatsoever to those of this License.  You may not use
technical measures to obstruct or control the reading or further
copying of the copies you make or distribute.  However, you may accept
compensation in exchange for copies.  If you distribute a large enough
number of copies you must also follow the conditions in section~3.

You may also lend copies, under the same conditions stated above, and
you may publicly display copies.


\subsection{COPYING IN QUANTITY}
% \begin{center}
%   {\Large\bf 3. COPYING IN QUANTITY\par}
%   \phantomsection
%   \addcontentsline{toc}{section}{3. COPYING IN QUANTITY}
% \end{center}


If you publish printed copies (or copies in media that commonly have
printed covers) of the Document, numbering more than 100, and the
Document's license notice requires Cover Texts, you must enclose the
copies in covers that carry, clearly and legibly, all these Cover
Texts: Front-Cover Texts on the front cover, and Back-Cover Texts on
the back cover.  Both covers must also clearly and legibly identify
you as the publisher of these copies.  The front cover must present
the full title with all words of the title equally prominent and
visible.  You may add other material on the covers in addition.
Copying with changes limited to the covers, as long as they preserve
the title of the Document and satisfy these conditions, can be treated
as verbatim copying in other respects.

If the required texts for either cover are too voluminous to fit
legibly, you should put the first ones listed (as many as fit
reasonably) on the actual cover, and continue the rest onto adjacent
pages.

If you publish or distribute Opaque copies of the Document numbering
more than 100, you must either include a machine-readable Transparent
copy along with each Opaque copy, or state in or with each Opaque copy
a computer-network location from which the general network-using
public has access to download using public-standard network protocols
a complete Transparent copy of the Document, free of added material.
If you use the latter option, you must take reasonably prudent steps,
when you begin distribution of Opaque copies in quantity, to ensure
that this Transparent copy will remain thus accessible at the stated
location until at least one year after the last time you distribute an
Opaque copy (directly or through your agents or retailers) of that
edition to the public.

It is requested, but not required, that you contact the authors of the
Document well before redistributing any large number of copies, to give
them a chance to provide you with an updated version of the Document.


\subsection{MODIFICATIONS}
% \begin{center}
%   {\Large\bf 4. MODIFICATIONS\par}
%   \phantomsection
%   \addcontentsline{toc}{section}{4. MODIFICATIONS}
% \end{center}

You may copy and distribute a Modified Version of the Document under
the conditions of sections 2 and 3 above, provided that you release
the Modified Version under precisely this License, with the Modified
Version filling the role of the Document, thus licensing distribution
and modification of the Modified Version to whoever possesses a copy
of it.  In addition, you must do these things in the Modified Version:

\begin{itemize}
\item[A.]
  Use in the Title Page (and on the covers, if any) a title distinct
  from that of the Document, and from those of previous versions
  (which should, if there were any, be listed in the History section
  of the Document).  You may use the same title as a previous version
  if the original publisher of that version gives permission.

\item[B.]
  List on the Title Page, as authors, one or more persons or entities
  responsible for authorship of the modifications in the Modified
  Version, together with at least five of the principal authors of the
  Document (all of its principal authors, if it has fewer than five),
  unless they release you from this requirement.

\item[C.]
  State on the Title page the name of the publisher of the
  Modified Version, as the publisher.

\item[D.]
  Preserve all the copyright notices of the Document.

\item[E.]
  Add an appropriate copyright notice for your modifications
  adjacent to the other copyright notices.

\item[F.]
  Include, immediately after the copyright notices, a license notice
  giving the public permission to use the Modified Version under the
  terms of this License, in the form shown in the Addendum below.

\item[G.]
  Preserve in that license notice the full lists of Invariant Sections
  and required Cover Texts given in the Document's license notice.

\item[H.]
  Include an unaltered copy of this License.

\item[I.]
  Preserve the section Entitled ``History'', Preserve its Title, and add
  to it an item stating at least the title, year, new authors, and
  publisher of the Modified Version as given on the Title Page.  If
  there is no section Entitled ``History'' in the Document, create one
  stating the title, year, authors, and publisher of the Document as
  given on its Title Page, then add an item describing the Modified
  Version as stated in the previous sentence.

\item[J.]
  Preserve the network location, if any, given in the Document for
  public access to a Transparent copy of the Document, and likewise
  the network locations given in the Document for previous versions
  it was based on.  These may be placed in the ``History'' section.
  You may omit a network location for a work that was published at
  least four years before the Document itself, or if the original
  publisher of the version it refers to gives permission.

\item[K.]
  For any section Entitled ``Acknowledgements'' or ``Dedications'',
  Preserve the Title of the section, and preserve in the section all
  the substance and tone of each of the contributor acknowledgements
  and/or dedications given therein.

\item[L.]
  Preserve all the Invariant Sections of the Document,
  unaltered in their text and in their titles.  Section numbers
  or the equivalent are not considered part of the section titles.

\item[M.]
  Delete any section Entitled ``Endorsements''.  Such a section
  may not be included in the Modified Version.

\item[N.]
  Do not retitle any existing section to be Entitled ``Endorsements''
  or to conflict in title with any Invariant Section.

\item[O.]
  Preserve any Warranty Disclaimers.
\end{itemize}

If the Modified Version includes new front-matter sections or
appendices that qualify as Secondary Sections and contain no material
copied from the Document, you may at your option designate some or all
of these sections as invariant.  To do this, add their titles to the
list of Invariant Sections in the Modified Version's license notice.
These titles must be distinct from any other section titles.

You may add a section Entitled ``Endorsements'', provided it contains
nothing but endorsements of your Modified Version by various
parties---for example, statements of peer review or that the text has
been approved by an organization as the authoritative definition of a
standard.

You may add a passage of up to five words as a Front-Cover Text, and a
passage of up to 25 words as a Back-Cover Text, to the end of the list
of Cover Texts in the Modified Version.  Only one passage of
Front-Cover Text and one of Back-Cover Text may be added by (or
through arrangements made by) any one entity.  If the Document already
includes a cover text for the same cover, previously added by you or
by arrangement made by the same entity you are acting on behalf of,
you may not add another; but you may replace the old one, on explicit
permission from the previous publisher that added the old one.

The author(s) and publisher(s) of the Document do not by this License
give permission to use their names for publicity for or to assert or
imply endorsement of any Modified Version.


\subsection{COMBINING DOCUMENTS}
% \begin{center}
%   {\Large\bf 5. COMBINING DOCUMENTS\par}
%   \phantomsection
%   \addcontentsline{toc}{section}{5. COMBINING DOCUMENTS}
% \end{center}


You may combine the Document with other documents released under this
License, under the terms defined in section~4 above for modified
versions, provided that you include in the combination all of the
Invariant Sections of all of the original documents, unmodified, and
list them all as Invariant Sections of your combined work in its
license notice, and that you preserve all their Warranty Disclaimers.

The combined work need only contain one copy of this License, and
multiple identical Invariant Sections may be replaced with a single
copy.  If there are multiple Invariant Sections with the same name but
different contents, make the title of each such section unique by
adding at the end of it, in parentheses, the name of the original
author or publisher of that section if known, or else a unique number.
Make the same adjustment to the section titles in the list of
Invariant Sections in the license notice of the combined work.

In the combination, you must combine any sections Entitled ``History''
in the various original documents, forming one section Entitled
``History''; likewise combine any sections Entitled ``Acknowledgements'',
and any sections Entitled ``Dedications''.  You must delete all sections
Entitled ``Endorsements''.

\subsection{COLLECTIONS OF DOCUMENTS}
% \begin{center}
%   {\Large\bf 6. COLLECTIONS OF DOCUMENTS\par}
%   \phantomsection
%   \addcontentsline{toc}{section}{6. COLLECTIONS OF DOCUMENTS}
% \end{center}

You may make a collection consisting of the Document and other documents
released under this License, and replace the individual copies of this
License in the various documents with a single copy that is included in
the collection, provided that you follow the rules of this License for
verbatim copying of each of the documents in all other respects.

You may extract a single document from such a collection, and distribute
it individually under this License, provided you insert a copy of this
License into the extracted document, and follow this License in all
other respects regarding verbatim copying of that document.


\subsection{AGGREGATION WITH INDEPENDENT WORKS}
% \begin{center}
%   {\Large\bf 7. AGGREGATION WITH INDEPENDENT WORKS\par}
%   \phantomsection
%   \addcontentsline{toc}{section}{7. AGGREGATION WITH INDEPENDENT WORKS}
% \end{center}


A compilation of the Document or its derivatives with other separate
and independent documents or works, in or on a volume of a storage or
distribution medium, is called an ``aggregate'' if the copyright
resulting from the compilation is not used to limit the legal rights
of the compilation's users beyond what the individual works permit.
When the Document is included in an aggregate, this License does not
apply to the other works in the aggregate which are not themselves
derivative works of the Document.

If the Cover Text requirement of section~3 is applicable to these
copies of the Document, then if the Document is less than one half of
the entire aggregate, the Document's Cover Texts may be placed on
covers that bracket the Document within the aggregate, or the
electronic equivalent of covers if the Document is in electronic form.
Otherwise they must appear on printed covers that bracket the whole
aggregate.


\subsection{TRANSLATION}
% \begin{center}
%   {\Large\bf 8. TRANSLATION\par}
%   \phantomsection
%   \addcontentsline{toc}{section}{8. TRANSLATION}
% \end{center}


Translation is considered a kind of modification, so you may
distribute translations of the Document under the terms of section~4.
Replacing Invariant Sections with translations requires special
permission from their copyright holders, but you may include
translations of some or all Invariant Sections in addition to the
original versions of these Invariant Sections.  You may include a
translation of this License, and all the license notices in the
Document, and any Warranty Disclaimers, provided that you also include
the original English version of this License and the original versions
of those notices and disclaimers.  In case of a disagreement between
the translation and the original version of this License or a notice
or disclaimer, the original version will prevail.

If a section in the Document is Entitled ``Acknowledgements'',
``Dedications'', or ``History'', the requirement (section~4) to Preserve
its Title (section~1) will typically require changing the actual
title.


\subsection{TERMINATION}
% \begin{center}
%   {\Large\bf 9. TERMINATION\par}
%   \phantomsection
%   \addcontentsline{toc}{section}{9. TERMINATION}
% \end{center}


You may not copy, modify, sublicense, or distribute the Document
except as expressly provided under this License.  Any attempt
otherwise to copy, modify, sublicense, or distribute it is void, and
will automatically terminate your rights under this License.

However, if you cease all violation of this License, then your license
from a particular copyright holder is reinstated (a) provisionally,
unless and until the copyright holder explicitly and finally
terminates your license, and (b) permanently, if the copyright holder
fails to notify you of the violation by some reasonable means prior to
60 days after the cessation.

Moreover, your license from a particular copyright holder is
reinstated permanently if the copyright holder notifies you of the
violation by some reasonable means, this is the first time you have
received notice of violation of this License (for any work) from that
copyright holder, and you cure the violation prior to 30 days after
your receipt of the notice.

Termination of your rights under this section does not terminate the
licenses of parties who have received copies or rights from you under
this License.  If your rights have been terminated and not permanently
reinstated, receipt of a copy of some or all of the same material does
not give you any rights to use it.


\subsection{FUTURE REVISIONS OF THIS LICENSE}
% \begin{center}
%   {\Large\bf 10. FUTURE REVISIONS OF THIS LICENSE\par}
%   \phantomsection
%   \addcontentsline{toc}{section}{10. FUTURE REVISIONS OF THIS LICENSE}
% \end{center}


The Free Software Foundation may publish new, revised versions
of the GNU Free Documentation License from time to time.  Such new
versions will be similar in spirit to the present version, but may
differ in detail to address new problems or concerns.  See
\texttt{https://www.gnu.org/licenses/}.

Each version of the License is given a distinguishing version number.
If the Document specifies that a particular numbered version of this
License ``or any later version'' applies to it, you have the option of
following the terms and conditions either of that specified version or
of any later version that has been published (not as a draft) by the
Free Software Foundation.  If the Document does not specify a version
number of this License, you may choose any version ever published (not
as a draft) by the Free Software Foundation.  If the Document
specifies that a proxy can decide which future versions of this
License can be used, that proxy's public statement of acceptance of a
version permanently authorizes you to choose that version for the
Document.


\subsection{RELICENSING}
% \begin{center}
%   {\Large\bf 11. RELICENSING\par}
%   \phantomsection
%   \addcontentsline{toc}{section}{11. RELICENSING}
% \end{center}


``Massive Multiauthor Collaboration Site'' (or ``MMC Site'') means any
World Wide Web server that publishes copyrightable works and also
provides prominent facilities for anybody to edit those works.  A
public wiki that anybody can edit is an example of such a server.  A
``Massive Multiauthor Collaboration'' (or ``MMC'') contained in the
site means any set of copyrightable works thus published on the MMC
site.

``CC-BY-SA'' means the Creative Commons Attribution-Share Alike 3.0
license published by Creative Commons Corporation, a not-for-profit
corporation with a principal place of business in San Francisco,
California, as well as future copyleft versions of that license
published by that same organization.

``Incorporate'' means to publish or republish a Document, in whole or
in part, as part of another Document.

An MMC is ``eligible for relicensing'' if it is licensed under this
License, and if all works that were first published under this License
somewhere other than this MMC, and subsequently incorporated in whole
or in part into the MMC, (1) had no cover texts or invariant sections,
and (2) were thus incorporated prior to November 1, 2008.

The operator of an MMC Site may republish an MMC contained in the site
under CC-BY-SA on the same site at any time before August 1, 2009,
provided the MMC is eligible for relicensing.


\subsection{ADDENDUM: How to use this License for your documents}
% \begin{center}
%   {\Large\bf ADDENDUM: How to use this License for your documents\par}
%   \phantomsection
%   \addcontentsline{toc}{section}{ADDENDUM: How to use this License for your documents}
% \end{center}

To use this License in a document you have written, include a copy of
the License in the document and put the following copyright and
license notices just after the title page:

\bigskip
\begin{quote}
  Copyright \copyright{}  YEAR  YOUR NAME.
  Permission is granted to copy, distribute and/or modify this document
  under the terms of the GNU Free Documentation License, Version 1.3
  or any later version published by the Free Software Foundation;
  with no Invariant Sections, no Front-Cover Texts, and no Back-Cover Texts.
  A copy of the license is included in the section entitled ``GNU
  Free Documentation License''.
\end{quote}
\bigskip

If you have Invariant Sections, Front-Cover Texts and Back-Cover Texts,
replace the ``with \dots\ Texts.''\ line with this:

\bigskip
\begin{quote}
  with the Invariant Sections being LIST THEIR TITLES, with the
  Front-Cover Texts being LIST, and with the Back-Cover Texts being LIST.
\end{quote}
\bigskip

If you have Invariant Sections without Cover Texts, or some other
combination of the three, merge those two alternatives to suit the
situation.

If your document contains nontrivial examples of program code, we
recommend releasing these examples in parallel under your choice of
free software license, such as the GNU General Public License,
to permit their use in free software.

%%% end of file


%page
%% ------------------------------------------------------------
%% Fine.
%% ------------------------------------------------------------

\end{document}

%%% end of file
% Local Variables:
% mode: latex
% page-delimiter: "^%page"
% TeX-master: t
% ispell-local-dictionary: "italiano"
% fill-column: 85
% End:

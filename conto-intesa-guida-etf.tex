% conto-intesa-guida-etf.tex -*- coding: utf-8-unix -*-

\documentclass[12pt,a4paper]{article}
\renewcommand{\rmdefault}{ptm} % Times Roman font, for PDF output
\usepackage[italian]{babel}
\usepackage[utf8]{inputenc}
\usepackage[T1]{fontenc}
\usepackage[]{textcomp}
\usepackage[]{eurosym}
\usepackage[]{amsmath}
\usepackage[]{amsfonts}
\usepackage[]{amsthm}
\usepackage[]{amssymb}
\usepackage[]{syntonly}
\usepackage[copy-decimal-marker,retain-explicit-plus]{siunitx}
\usepackage[hidelinks]{hyperref} % To generate PDFs with hyperlinks

\pagestyle{headings}

%page
%% ------------------------------------------------------------
%% Comandi.
%% ------------------------------------------------------------

\newcommand{\Eur}[1]{\SI{#1}{\text{\euro{}}}}

\newcommand{\MediaPonderataDue}[4]{\frac{\num{#1} \times{} \num{#2} + \num{#3} \times{} \num{#4}}{\num{#1} + \num{#3}}}
\newcommand{\MediaPonderataTre}[6]{\frac{\num{#1} \times{} \num{#2} + \num{#3} \times{} \num{#4} + \num{#5} \times{} \num{#6}}{\num{#1} + \num{#3} + \num{#5}}}

\newcommand{\CostoOperazione}[1]{\num{0,50} + \num{2,50} + \num{0,0024} \times{} \num{#1}}

\newcommand{\RendimentoPercentuale}[2]{\frac{\num{#1} - \num{#2}}{#2} \times{} \num{100}}

% \everytexdraw{
%    \setgray 0 \linewd 0.01
%    \arrowheadsize l:0.12 w:0.04 \arrowheadtype t:F
% }

%page
%% ------------------------------------------------------------
%% Intestazione
%% ------------------------------------------------------------

\author{Marco Maggi}
\title{Pericolosa e incompleta guida agli investimenti in ETF con \emph{Banca Intesa Sanpaolo}}

\begin{document}

\maketitle

\begin{abstract}
  Questa guida è un  aiuto per chi volesse ricostruire i calcoli  di tenuta del Conto
  Titoli  di \emph{Banca  Intesa  Sanpaolo} per  gli  investimenti in  \emph{Exchange
     Traded Funds} (ETF).  Questa guida è incompleta:  non copre tutto ciò che c'è da
  sapere sugli ETF,  né l'operatività del sito di \emph{home  banking} di \emph{Banca
     Intesa Sanpaolo}.   Questa guida  è pericolosa:  ognuno la usa  a suo  rischio e
  pericolo.
\end{abstract}

\tableofcontents

\newpage{}

\noindent
Copyright \copyright{} 2017 Marco Maggi \texttt{<marco.maggi-ipsu@poste.it>}.

Permission is granted to copy, distribute and/or modify this document under the terms
of the GNU Free Documentation License, Version  1.3 or any later version published by
the Free Software  Foundation; with no Invariant Sections, no  Front-Cover Texts, and
no Back-Cover Texts.  A copy of the license is included in the section entitled ``GNU
Free Documentation License''.

Il codice sorgente di questa guida è disponibile in Rete all'indirizzo:
\begin{center}
  \url{https://github.com/marcomaggi/mmux-personal-finance}
\end{center}

\newpage{}

%page
\section{Introduzione}


Il calcolo dei redditi da capitale e dei redditi diversi per investimenti in ETF deve
essere eseguito come specificato nella ``Circolare dell'Agenzia dell'Entrate 21/E del
10 Luglio  2014''.  Il calcolo di  questi redditi permette sia  la determinazione dei
rendimenti che il calcolo delle tasse.

Per  ogni  operazione di  compravendita  eseguita  sul  Conto Titoli:  nella  sezione
documenti del sito  di \emph{home banking} è disponibile una  \emph{nota di eseguito}
con  tutti  i  dati  dell'operazione.   Nei calcoli  qui  illustrati:  alcuni  valori
riportati  nelle note  di eseguito  sono  i dati  di  partenza; altri  valori sono  i
risultati ricalcolati per verifica.

Questa guida  utilizza sia  la terminologia  della Banca  che quella  della circolare
dell'Agenzia dell'Entrate.  Da notare che i  prezzi di acquisto e vendita sulle Borse
di negoziazione sono chiamati \emph{effettivi} dalla circolare.

Tutti i  numeri in questa  guida sono presentati arrotondati,  il piú delle  volte al
centesimo  di  Euro;  l'esecuzione  dei  calcoli è  invece  eseguita  con  precisione
maggiore, per ridurre gli errori di arrotondamento.

%page
\section{Media aritmetica e media ponderata}


Nei  calcoli  di tenuta  Conto  Titoli  si  usa  spesso l'operazione  di  \emph{media
   aritmetica} e  in particolare  la sua riscrittura  come \emph{media  ponderata}; è
utile richiamare queste idee e alcune delle loro proprietà.

Si  ricorda che  la  \textbf{media  aritmetica} tra  i  numeri  \num{11}, \num{22}  e
\num{33} si scrive:
\begin{equation*}
  \frac{\num{11} + \num{22} + \num{33}}{3} = \num{22}
\end{equation*}
il denominatore è \num{3} perché al numeratore ci sono \num{3} numeri.

L'operazione si  costruisce nello stesso  modo se qualche  numero compare piú  di una
volta;  la media  aritmetica tra  i numeri:  \num{11}, \num{11},  \num{11}, \num{22},
\num{33} e \num{33} si scrive:
\begin{equation*}
  \frac{\num{11} + \num{11} + \num{11} + \num{22} + \num{33} + \num{33}}{6}
  = \frac{121}{6} \simeq \num{20,17}
\end{equation*}
il denominatore è \num{6} perché:
\begin{itemize}
\item \num{11} compare \num{3} volte;
\item \num{22} compare \num{1} volta;
\item \num{33} compare \num{2} volte;
\end{itemize}
quindi \(\num{3} + \num{1} + \num{2} = \num{6}\).

Il ``numero  di volte  in cui  un numero  compare'' si  chiama \textbf{molteplicità};
allora si può dire che:
\begin{itemize}
\item \num{11} compare con \emph{molteplicità} \num{3};
\item \num{22} compare con \emph{molteplicità} \num{1};
\item \num{33} compare con \emph{molteplicità} \num{2};
\end{itemize}
evidenziando le molteplicità e considerando le semplici identità:
\begin{align*}
  \num{11} + \num{11} + \num{11} &= \num{3} \times{} \num{11} &&&
  \num{22} &= \num{1} \times{} \num{22} &&&
  \num{33} + \num{33} &= \num{2} \times{} \num{33}
\end{align*}
l'espressione della \emph{media aritmetica} si può riscrivere:
\begin{align*}
  &\frac{\num{11} + \num{11} + \num{11} + \num{22} + \num{33} + \num{33}}{6}
  = \frac{\num{3} \times{} \num{11}
    + \num{1} \times{} \num{22}
    + \num{2} \times{} \num{33}}
    {6} = \\
  &=
    \frac{\num{3} \times{} \num{11}}{6} +
    \frac{\num{1} \times{} \num{22}}{6} +
    \frac{\num{2} \times{} \num{33}}{6}
  =
    \frac{\num{3}}{6} \times{} \num{11} +
    \frac{\num{1}}{6} \times{} \num{22} +
    \frac{\num{2}}{6} \times{} \num{33}
\end{align*}
in cui si evidenzia come:
\begin{itemize}
\item il numero \num{11} compaia con ``peso'' \num{3} rispetto al totale \num{6};
\item il numero \num{22} compaia con ``peso'' \num{1} rispetto al totale \num{6};
\item il numero \num{33} compaia con ``peso'' \num{2} rispetto al totale \num{6}.
\end{itemize}

In questo modo: la \emph{media aritmetica} tra i numeri \num{11}, \num{11}, \num{11},
\num{22}, \num{33} e  \num{33} è riscritta come media \textbf{media  ponderata} tra i
numeri \num{11}, \num{22} e \num{33} rispetto alle loro molteplicità \num{3}, \num{1}
e \num{2}.

Nei  calcoli di  tenuta  Conto Titoli:  si  eseguono medie  ponderate  tra prezzi  di
acquisto o vendita  di quote di fondi,  rispetto al numero di  quote acquistate; tale
numero di quote assume il ruolo di ``molteplicità di un prezzo''.

%page
\subsection{Calcolo incrementale della media ponderata}


L'attività di  investire in  un ETF  è costituita  da una  sequenza di  operazioni di
acquisto e vendita di quote del fondo; acquisti e vendite possono essere intercalati;
il  prezzo  del  totale  delle   quote  possedute  viene  calcolato  aggiornando  una
\emph{media ponderata} dopo  ogni operazione.  È utile richiamare come  il calcolo di
una \emph{media ponderata} possa essere eseguito passo dopo passo.

Si considerino  ancora i  numeri \num{11}, \num{11},  \num{11}, \num{22},  \num{33} e
\num{33}; si  è calcolato che  la loro  \emph{media aritmetica} può  essere riscritta
come \emph{media ponderata} e risulta:
\begin{equation*}
  \frac{\num{11} + \num{11} + \num{11} + \num{22} + \num{33} + \num{33}}{6}
  = \frac{\num{3} \times{} \num{11}
     + \num{1} \times{} \num{22}
     + \num{2} \times{} \num{33}}
  {6} \simeq \num{20,17}
\end{equation*}
si considerino  prima i  numeri \num{11},  \num{22} e \num{33};  poi si  aggiungano i
numeri \num{11} e \num{11}; infine si aggiunga il numero \num{33}:
\begin{enumerate}
\item la \emph{media ponderata} tra i numeri \num{11}, \num{22} e \num{33} risulta:
  \begin{equation*}
    \frac{\num{1} \times{} \num{11}
       + \num{1} \times{} \num{22}
       + \num{1} \times{} \num{33}}{\num{3}}
    = \num{22}
  \end{equation*}

\item  ora si  aggiungano  i numeri  \num{11} e  \num{11};  la \emph{media  ponderata
     aggiornata} risulta:
  \begin{equation*}
    \frac{\num{3} \times{} \num{11}
       + \num{1} \times{} \num{22}
       + \num{1} \times{} \num{33}}{\num{5}}
    = \frac{\num{88}}{\num{5}} \simeq \num{17,6}
  \end{equation*}
  e si può scrivere:
  \begin{align*}
    \frac{\num{3} \times{} \num{11}
    + \num{1} \times{} \num{22}
    + \num{1} \times{} \num{33}}{\num{5}}
    = \frac{\num{2} \times{} \num{11}}{\num{5}}
    + \frac{\num{1} \times{} \num{11}
    + \num{1} \times{} \num{22}
    + \num{1} \times{} \num{33}}{\num{5}}
  \end{align*}
  la seconda frazione al membro di destra si può riscrivere:
  \begin{align*}
    \frac{\num{3}}{\num{5}} \times \frac{\num{5}}{\num{3}} \times
    \frac{\num{1} \times{} \num{11}
    + \num{1} \times{} \num{22}
    + \num{1} \times{} \num{33}}{\num{5}}
    &= \frac{\num{3}}{\num{5}} \times
      \frac{\num{1} \times{} \num{11}
      + \num{1} \times{} \num{22}
      + \num{1} \times{} \num{33}}{\num{3}} \\
    &\simeq \frac{\num{3}}{\num{5}} \times{} \num{20,17}
      = \frac{\num{3} \times{} \num{20,17}}{\num{5}}
  \end{align*}
  e quindi la \emph{media ponderata aggiornata} risulta:
  \begin{align*}
    \frac{\num{2} \times{} \num{11}}{\num{5}}
    + \frac{\num{3} \times{} \num{20,17}}{\num{5}}
    = \frac{\num{2} \times{} \num{11} + \num{3} \times{} \num{20,17}}{\num{5}}
    \simeq \num{17,6}
  \end{align*}
  cioè la \emph{media ponderata aggiornata} è pari alla \emph{media ponderata} tra:
  \begin{itemize}
  \item la \emph{media ponderata precedente}, con la sua molteplicità;
  \item il nuovo numero aggiunto, con la sua molteplicità;
  \end{itemize}

\item  infine  si  aggiunga  il  numero  \num{33};  la  nuova  \emph{media  ponderata
     aggiornata} è la \emph{media ponderata totale} e procedendo come prima risulta:
  \begin{equation*}
    \frac{\num{1} \times{} \num{33}
       + \num{5} \times{} \num{17,6}}{\num{6}}
    \simeq \num{20,17}
  \end{equation*}
\end{enumerate}

Nei calcoli  di tenuta  Conto Titoli:  dopo ogni acquisto  di quote  di un  fondo, si
calcola  il  prezzo  medio  di   ogni  quota  posseduta  come  \emph{media  ponderata
   aggiornata} tra:
\begin{itemize}
\item  la  \emph{media ponderata  precedente},  con  il  precedente numero  di  quote
  possedute come molteplicità;
\item il prezzo delle nuove quote acquistate,  con il numero di quote acquistate come
  molteplicità.
\end{itemize}


%page
\section{Direzione dei trasferimenti di denaro}


Con  abusiva  semplificazione, le  somme  di  denaro  descritte  in questa  guida  si
ritengono scambiate tra il Conto Corrente e il Conto Titoli:
\begin{itemize}
\item le somme di denaro \emph{investite} in  quote di fondi sono in \emph{uscita dal
     Conto Corrente} e in \emph{entrata nel Conto Titoli};
\item le somme di denaro \emph{disinvestite}  da quote di fondi sono in \emph{entrata
     nel Conto Corrente} e in \emph{uscita dal Conto Titoli}.
\end{itemize}

Per esempio,  si supponga  di acquistare  quote di un  fondo per  \Eur{1000,00} tutto
incluso; questa quantità di denaro esce dal Conto Corrente ed entra nel Conto Titoli;
si considerino i due casi:
\begin{itemize}
\item si vendono le quote al prezzo netto di \Eur{1200,00}; questa quantità di denaro
  esce dal Conto Titoli ed entra nel  Conto Corrente; il bilancio delle operazioni si
  esprime con la differenza algebrica:
  \begin{equation*}
    \num{1200,00} - \num{1000,00} = \Eur{+200,00}
  \end{equation*}
  cioè, nel  saldo finale, \Eur{200,00} ``appaiono''  nel Conto Titoli, vi  escono ed
  entrano nel Conto Corrente: un guadagno;
\item si vendono le quote al prezzo  netto di \Eur{900,00}; questa quantità di denaro
  esce dal Conto Titoli ed entra nel  Conto Corrente; il bilancio delle operazioni si
  esprime con la differenza algebrica:
  \begin{equation*}
    \num{900,00} - \num{1000,00} = \Eur{-100,00}
  \end{equation*}
  cioè, nel saldo finale, \Eur{100,00} escono dal Conto Corrente ed entrano nel Conto
  Titoli in cui poi ``spariscono'': una perdita;
\end{itemize}
le differenze algebriche hanno due interpretazioni:
\begin{itemize}
\item \emph{uscite dal Conto Titoli} meno \emph{entrate nel Conto Titoli};
\item \emph{entrate nel Conto Corrente} meno \emph{uscite dal Conto Corrente}.
\end{itemize}

%page
\section{Descrizione di un'operazione di acquisto}


\newcommand{\OneNumeroQuote}{100}
\newcommand{\OnePrezzoMedioEseguito}{53,80}
\newcommand{\OneControvaloreOperazione}{5380,00}
\newcommand{\OneCostoOperazione}{15,91}
\newcommand{\OneControvaloreTotale}{5395,91}
\newcommand{\OnePrezzoMedioCarico}{53,96}


Scelta la strategia di  acquisto con prezzo limite: in una certa data,  e a una certa
ora, si inserisce  l'ordine di acquisto per un  numero di quote del fondo  a un certo
prezzo massimo (si può acquistare a qualsiasi prezzo al di sotto del limite fissato).
L'ordine  può essere  eseguito  in piú  fasi  in cui  solo una  parte  delle quote  è
acquistata, in  ogni fase a  un prezzo  diverso; è possibile  che non tutte  le quote
siano acquistate, nel qual caso ci interessa solo la parte ``eseguita'' dell'ordine.

Il \textbf{prezzo  medio eseguito di  un acquisto} (nella terminologia  della Banca),
anche  detto  \textbf{prezzo medio  effettivo  di  un acquisto}  (nella  terminologia
dell'Agenzia dell'Entrate), è  la media ponderata dei prezzi di  acquisto rispetto al
numero  di quote  acquistate.  Per  esempio,  se un  singolo ordine  di acquisto  per
\num{\OneNumeroQuote} quote è eseguito nelle tre fasi:
\begin{enumerate}
\item acquisto di \num{20} quote al prezzo di \Eur{52,00};
\item acquisto di \num{30} quote al prezzo di \Eur{53,00};
\item acquisto di \num{50} quote al prezzo di \Eur{55,00};
\end{enumerate}
allora il \emph{prezzo medio eseguito} risulta:
\begin{equation*}
  \MediaPonderataTre{20}{52,00}{30}{53,00}{50}{55,00} = \Eur{\OnePrezzoMedioEseguito{}}
\end{equation*}

Il  \textbf{controvalore dell'operazione  di acquisto}  è il  prodotto tra  il numero
totale di  quote acquistate e  il \emph{prezzo medio  eseguito}.  Per esempio,  se si
acquistano \num{\OneNumeroQuote} quote al \emph{prezzo medio eseguito} di \Eur{\OnePrezzoMedioEseguito}
il controvalore dell'operazione risulta:
\begin{equation*}
  \num{\OneNumeroQuote} \times{} \num{\OnePrezzoMedioEseguito{}}
  = \Eur{\OneControvaloreOperazione}
\end{equation*}

Il  \textbf{costo  dell'operazione  di  acquisto}  è la  somma  tra  costi,  spese  e
commissioni associate all'operazione.  Per ogni  ordine di acquisto eseguito, occorre
pagare:
\begin{itemize}
\item le spese per la Banca pari a un fisso di \Eur{0,50};
\item i costi di intermediario pari a un fisso di \Eur{2,50};
\item   le   commissioni   per   la   Banca   pari   allo   \SI{0,24}{\percent}   del
  \emph{controvalore dell'operazione}.
\end{itemize}
Se un  ordine non è eseguito:  si paga nulla.  Se  un ordine è eseguito  in piú fasi:
nella prima fase si  pagano i costi fissi piú la  commissione percentuale; nelle fasi
successive  si   paga  solo   la  commissione  percentuale.    Per  esempio,   se  il
\emph{controvalore dell'operazione}  è \Eur{3189,60} il  \emph{costo dell'operazione}
risulta:
\begin{equation*}
  \CostoOperazione{\OneControvaloreOperazione} = \Eur{\OneCostoOperazione}
\end{equation*}

Il  \textbf{controvalore totale  di un  acquisto} è  la somma  tra \emph{controvalore
   dell'operazione} e \emph{costo dell'operazione}; è la quantità di denaro in uscita
dal  Conto   Corrente  e  in   entrata  nel  Conto   Titoli.   Per  esempio,   se  il
\emph{controvalore   dell'operazione}  è   \Eur{\OneControvaloreOperazione{}}  e   il
\emph{costo    dell'operazione}    è    pari    a    \Eur{\OneCostoOperazione},    il
\emph{controvalore totale} risulta:
\begin{equation*}
  \num{\OneControvaloreOperazione} + \num{\OneCostoOperazione}
  = \Eur{\OneControvaloreTotale}
\end{equation*}

Il  \textbf{prezzo   medio  di  carico  di   un  acquisto}  é  il   rapporto  tra  il
\emph{controvalore  totale}  e  il  numero  di   quote  acquistate  ed  è  usato  per
l'aggiornamento del saldo:  è il prezzo medio di una  singola quota acquistata, tutto
incluso.   Per  esempio,  se  si   sono  acquistate  \num{\OneNumeroQuote}  quote  al
\emph{controvalore totale} di \Eur{\OneControvaloreTotale},  il \emph{prezzo medio di
   carico} risulta:
\begin{equation*}
  \num{\OneControvaloreTotale} / \num{\OneNumeroQuote} = \Eur{\OnePrezzoMedioCarico}
\end{equation*}

%page
\section{Aggiornamento del saldo dopo un'operazione di acquisto}


Dopo l'esecuzione  di ogni operazione  di acquisto  occorre ricalcolare il  saldo del
Conto  Titoli;  per un  investimento  in  quote di  ETF,  la  parte da  aggiornare  è
rappresentata dalle tre quantità:
\begin{itemize}
\item numero di quote in carico nel Conto Titoli;
\item \textbf{prezzo medio  \underline{effettivo} nel saldo}: è il  prezzo medio, per
  quota acquistata,  degli ordini  eseguiti sulla Borsa  di negoziazione;  si calcola
  come media  ponderata tra il  \emph{prezzo medio \underline{effettivo} di  un nuovo
     acquisto} e  il precedente \emph{prezzo medio  \underline{effettivo} nel saldo};
  nella documentazione della  Banca: questo valore è chiamato  \emph{Net Asset Value}
  (NAV)  del prezzo  medio di  carico  (da non  confondere con  il NAV  dell'attività
  sottostante il fondo);
\item \textbf{prezzo  medio di \underline{carico} nel  saldo}: è il prezzo  medio per
  quota  acquistata,  tutto   incluso;  si  calcola  come  media   ponderata  tra  il
  \emph{prezzo  medio di  \underline{carico} di  un nuovo  acquisto} e  il precedente
  \emph{prezzo medio di \underline{carico} nel saldo}.
\end{itemize}

Dai valori nel saldo si può calcolare il \textbf{costo medio per quota nel saldo}: si
calcola come differenza tra il \emph{prezzo  medio di \underline{carico} nel saldo} e
il \emph{prezzo  medio di \underline{effettivo} nel  saldo}, divisa per il  numero di
quote nel saldo.

Per esempio, si supponga di eseguire le operazioni:
\begin{enumerate}
\item acquisto di \num{101} quote al \emph{prezzo medio effettivo} di \Eur{51,00};
\item acquisto di \num{102} quote al \emph{prezzo medio effettivo} di \Eur{52,00};
\item acquisto di \num{103} quote al \emph{prezzo medio effettivo} di \Eur{53,00}.
\end{enumerate}
all'inizio si  consideri un Conto Titoli  vuoto, con saldo convenzionale  di: \num{0}
quote; \emph{prezzo  medio di  carico} di \Eur{0};  \emph{prezzo medio  effettivo} di
\Eur{0}.  L'aggiornamento del saldo avviene come segue.
\begin{enumerate}
\item Per la prima operazione:
  \begin{itemize}
  \item numero di quote acquistate: \num{101};
  \item \emph{prezzo medio effettivo}: \Eur{51,00};
  \item \emph{controvalore dell'operazione}: \Eur{5151,00};
  \item \emph{costo dell'operazione}: \Eur{15,36};
  \item \emph{controvalore totale}: \Eur{5166,36};
  \item \emph{prezzo medio di carico}: \Eur{51,15}.
  \end{itemize}

  Dopo la prima operazione:
  \begin{itemize}
  \item numero quote nel saldo: \num{101}; uguale al numero quote del primo acquisto;
  \item \emph{prezzo medio effettivo nel  saldo}: \Eur{51,00}; uguale al \emph{prezzo
       medio effettivo} del primo acquisto;
  \item \emph{prezzo  medio carico  nel saldo}:  \Eur{51,15}; uguale  al \emph{prezzo
       medio di carico} del primo acquisto;
  \end{itemize}
  il \emph{costo medio per quota nel saldo} risulta:
  \begin{equation*}
    \frac{\num{51,15} - \num{51,00}}{\num{101}} = \Eur{0,1521}
  \end{equation*}

\item Per la seconda operazione:
  \begin{itemize}
  \item numero di quote acquistate: \num{102};
  \item \emph{prezzo medio effettivo}: \Eur{52,00};
  \item \emph{controvalore dell'operazione}: \Eur{5304,00};
  \item \emph{costo dell'operazione}: \Eur{15,73};
  \item \emph{controvalore totale}: \Eur{5319,73};
  \item \emph{prezzo medio di carico}: \Eur{52,15}.
  \end{itemize}

  Dopo la seconda operazione:
  \begin{itemize}
  \item numero quote nel saldo: \(101 + 102 = 203\);
  \item  il  \emph{prezzo  medio  effettivo  nel saldo}  è  la  media  ponderata  dei
    \emph{prezzi medi effettivi} del saldo precedente e del nuovo acquisto:
    \begin{equation*}
      \MediaPonderataDue{101}{51,00}{102}{52,00} = \Eur{51,50}
    \end{equation*}
  \item  il  \emph{prezzo  medio di  carico  nel  saldo}  è  la media  ponderata  dei
    \emph{prezzi medi di carico} del saldo precedente e del nuovo acquisto:
    \begin{equation*}
      \MediaPonderataDue{101}{51,15}{102}{52,15} = \Eur{51,66}
    \end{equation*}
  \end{itemize}
  il \emph{costo medio per quota nel saldo}:
  \begin{equation*}
    \frac{\num{51,66} - \num{51,50}}{\num{203}} = \Eur{0,1532}
  \end{equation*}

\item Per la terza operazione:
  \begin{itemize}
  \item numero di quote acquistate: \num{103};
  \item \emph{prezzo medio effettivo}: \Eur{53,00};
  \item \emph{controvalore dell'operazione}: \Eur{5459,00};
  \item \emph{costo dell'operazione}: \Eur{16,10};
  \item \emph{controvalore totale}: \Eur{5475,10};
  \item \emph{prezzo medio di carico}: \Eur{53,16}.
  \end{itemize}

  Dopo la terza operazione:
  \begin{itemize}
  \item numero quote nel saldo: \(203 + 103 = 306\);
  \item  il \emph{prezzo  medio  effettivo  nel saldo}  è  la  media ponderata  dei
    \emph{prezzi medi effettivi} del saldo precedente e del nuovo acquisto:
    \begin{equation*}
      \MediaPonderataDue{203}{51,50}{103}{53,00} = \Eur{52,01}
    \end{equation*}
  \item il \emph{prezzo medio di carico carico  nel saldo} è la media ponderata dei
    \emph{prezzi medi di carico} del saldo precedente e del nuovo acquisto:
    \begin{equation*}
      \MediaPonderataDue{203}{51,66}{103}{53,16} = \Eur{52,16}
    \end{equation*}
  \end{itemize}
  il \emph{costo medio per quota nel saldo} risulta:
  \begin{equation*}
    \frac{\num{52,16} - \num{52,01}}{\num{306}} = \Eur{0,1542}
  \end{equation*}
\end{enumerate}

Evidenziando i prezzi medi: ogni nuova quota acquistata viene ``buttata nel secchio''
insieme alle  vecchie, indipendentemente dal suo  prezzo di acquisto; è  come se ogni
quota fosse stata acquistata allo stesso prezzo, pari al prezzo medio.

Si  osservi come,  \textbf{solo quando  si sono  eseguiti degli  acquisti ma  nessuna
   vendita}:
\begin{itemize}
\item il \emph{prezzo medio effettivo nel  saldo} si possa calcolare anche come media
  ponderata tra i \emph{prezzi medi effettivi degli acquisti}:
  \begin{equation*}
    \MediaPonderataTre{101}{51,00}{102}{52,00}{103}{53,00} = \Eur{52,01}
  \end{equation*}
\item il \emph{prezzo medio di carico nel  saldo} si possa calcolare anche come media
  ponderata tra i \emph{prezzi medi di carico degli acquisti}:
  \begin{equation*}
    \MediaPonderataTre{101}{51,15}{102}{52,15}{103}{53,16} = \Eur{52,16}
  \end{equation*}
\item il  costo totale di  tutte le operazioni  si possa sia  come somma tra  tutti i
  costi:
  \begin{equation*}
    \num{15,36} + \num{15,73} + \num{16,10} = \Eur{47,19}
  \end{equation*}
  che come  prodotto tra  il numero  di quote nel  saldo e  il \emph{costo  medio per
     quota}:
  \begin{equation*}
    \num{306} \times{} \num{0,1542} = \Eur{47,19}
  \end{equation*}
\end{itemize}
dopo la prima operazione di vendita: queste proprietà non valgono piú!

%page
\section{Descrizione di un'operazione di vendita}


Scelta la strategia  di vendita con prezzo limite:  in una certa data, e  a una certa
ora, si  inserisce l'ordine di vendita  per un numero di  quote del fondo a  un certo
prezzo minimo  (si può vendere  a qualsiasi prezzo al  di sopra del  limite fissato).
L'ordine può essere eseguito in piú fasi in cui solo una parte delle quote è venduta,
in ogni fase a  un prezzo diverso; è possibile che non tutte  le quote siano vendute,
nel qual caso ci interessa solo la parte ``eseguita'' dell'ordine.

Il \textbf{prezzo  medio eseguito di  una vendita} (nella terminologia  della Banca),
anche  detto   \textbf{prezzo  medio   effettivo  di  vendita}   (nella  terminologia
dell'Agenzia dell'Entrate),  è la media ponderata  dei prezzi di vendita  rispetto al
numero di quote vendute.  Per esempio, se  un singolo ordine di vendita per \num{100}
quote è eseguito nelle fasi:
\begin{enumerate}
\item vendita di \num{20} quote al prezzo di \Eur{52,00};
\item vendita di \num{30} quote al prezzo di \Eur{53,00};
\item vendita di \num{50} quote al prezzo di \Eur{55,00};
\end{enumerate}
allora il \emph{prezzo medio effettivo} risulta:
\begin{equation*}
  \MediaPonderataTre{20}{52,00}{30}{53,00}{50}{55,00} = \Eur{53,80}
\end{equation*}

Il  \textbf{controvalore dell'operazione  di vendita}  è  il prodotto  tra il  numero
totale  di quote  vendute e  il \emph{prezzo  medio effettivo}.   Per esempio,  se si
vendono  \num{100}  quote   al  \emph{prezzo  medio  effettivo}   di  \Eur{53,80}  il
controvalore dell'operazione risulta:
\begin{equation*}
  \num{100} \times{} \num{53,80} = \Eur{5380,00}
\end{equation*}

La  circolare dell'Agenzia  dell'Entrate specifica  che il  \emph{reddito per  quota}
derivante  da  un'operazione  di  vendita  è la  differenza  tra  \emph{prezzo  medio
   effettivo della vendita} e \emph{prezzo medio effettivo nel saldo}; peculiarmente:
\begin{itemize}
\item quando la  differenza tra prezzi medi effettivi è  \textbf{positiva}: essa è da
  considerarsi \emph{reddito da capitale}, quindi \textbf{non genera} una plusvalenza
  utilizzabile per compensare precedenti minusvalenze da altri investimenti;
\item quando la  differenza tra prezzi medi effettivi è  \textbf{negativa}: essa è da
  considerarsi  un \emph{reddito  diverso}, quindi  \textbf{genera} una  minusvalenza
  compensabile con successive plusvalenze da altri investimenti;
\end{itemize}
da notare che \textbf{le minusvalenze sono  registrate nello zainetto fiscale solo al
   momento della vendita di quote}.

Quando il  reddito di  una vendita  è positivo: occorre  pagare la  \textbf{tassa sul
   reddito da  capitale}\footnote{Si deve pagare  anche il bollo  sull'estratto conto
   trimestrale per il Conto Titoli, pari allo \SI{0,2}{\percent} all'anno; il bollo è
   addebitato direttamente sul Conto Corrente, perciò in questa guida lo si considera
   conteggiato a parte.}.  Per esempio, se si vendono \num{100} quote al \emph{prezzo
   medio  effettivo} di  \Eur{53,80} e  nel  saldo precedente  il \emph{prezzo  medio
   effettivo} è di \Eur{50,00}, il \emph{reddito da capitale} risulta:
\begin{equation*}
  \num{100} \times{} (\num{53,80} - \num{50,00}) = \Eur{380,00}
\end{equation*}
per fondi  che non contengono Titoli  dello Stato Italiano, e  assimilati, l'aliquota
unica  è  del \SI{26}{\percent};  allora  la  \emph{tassa  sul reddito  da  capitale}
risulta:
\begin{equation*}
  \num{0,26} \times{} \num{380,00} = \Eur{98,80}
\end{equation*}

Il  \textbf{costo  dell'operazione  di  vendita}  è  la  somma  tra  costi,  spese  e
commissioni associate  all'operazione; per ogni  ordine di vendita  eseguito, occorre
pagare:
\begin{itemize}
\item le spese per la Banca pari a un fisso di \Eur{0,50};
\item i costi di intermediario pari a un fisso di \Eur{2,50};
\item   le   commissioni   per   la   Banca   pari   allo   \SI{0,24}{\percent}   del
  \emph{controvalore dell'operazione};
\end{itemize}
se un  ordine non è  eseguito: si paga  nulla; se un ordine  è eseguito in  piú fasi:
nella prima fase si  pagano i costi fissi piú la  commissione percentuale; nelle fasi
successive  si   paga  solo   la  commissione  percentuale.    Per  esempio,   se  il
\emph{controvalore dell'operazione}  è \Eur{5380,00} il  \emph{costo dell'operazione}
risulta:
\begin{equation*}
  \CostoOperazione{5380,00} = \Eur{15,91}
\end{equation*}

Da osservare  che: sia il  calcolo della \emph{tassa sul  reddito da capitale}  che i
calcoli del \emph{costo dell'operazione} di acquisto  e vendita si eseguono usando il
\emph{prezzo medio effettivo};  in pratica: si pagano le tasse  anche sul \emph{costo
   dell'operazione} di  acquisto e vendita, che  però sono redditi di  qualcun altro,
non dell'investitore!

La circolare dell'Agenzia dell'Entrate specifica che il \emph{costo delle operazioni}
genera un \emph{redditi  diverso}, da registrare come minusvalenza.   Per esempio, si
supponga di possedere un Conto Titoli:
\begin{itemize}
\item numero quote nel saldo precedente: \num{100};
\item \emph{prezzo medio effettivo nel saldo precedente}: \Eur{50,00};
\item \emph{prezzo medio di carico nel saldo precedente}: \Eur{50,15};
\item \emph{costo medio per quota nel saldo precedente}: \Eur{0,15};
\end{itemize}
si vendano tutte le quote; l'operazione di vendita sia descritta da:
\begin{itemize}
\item numero quote vendute: \num{100};
\item \emph{prezzo medio effettivo di vendita} \Eur{53,80};
\item \emph{reddito da capitale}: \Eur{380,00};
\item \emph{costo dell'operazione di vendita}: \Eur{15,91};
\end{itemize}
ogni  quota venduta  era stata  acquistata  con un  \emph{costo medio  per quota}  di
\Eur{0,15}; perciò il \emph{costo di acquisto delle quote vendute} risulta:
\begin{equation*}
  \num{100} \times{} \num{0,15} = \Eur{15,00}
\end{equation*}
allo  scopo di  far  risultare un  numero  negativo, la  circolare  specifica che  il
\emph{reddito diverso} associato alla vendita deve essere calcolato come:
\begin{equation*}
  \left[\num{380,00} - \left(\num{15,91} + \num{15,00}\right)\right] - \num{380,00}
  = - (\num{15,91} + \num{15,00}) = \Eur{-30,91}
\end{equation*}
questa  minusvalenza deve  essere  registrata nello  ``zainetto  fiscale'' del  Conto
Titoli.

Il \textbf{controvalore totale di vendita}  è la differenza tra il \emph{controvalore
   dell'operazione} e  la somma  tra \emph{costo  dell'operazione} e  \emph{tassa sul
   reddito da  capitale}; è la  quantità di  denaro in uscita  dal Conto Titoli  e in
entrata nel Conto Corrente.  Per esempio, se il \emph{controvalore dell'operazione} è
\Eur{5380,00}, il  \emph{costo dell'operazione}  è \Eur{15,91}  e la  \emph{tassa sul
   reddito da capitale} è \Eur{98,80}, il \emph{controvalore totale} risulta:
\begin{equation*}
  \num{5380,00} - \num{15,91} - \num{98,80} = \Eur{5265,29}
\end{equation*}

Per un'operazione di  vendita: \textbf{non esiste} un \emph{prezzo  medio di carico};
la vendita modifica solo il numero di quote in carico nel saldo.  Invece si definisce
il \textbf{prezzo medio netto di vendita}  pari al rapporto tra il \emph{controvalore
   totale} e il numero  di quote vendute.  Per esempio, se  si sono vendute \num{100}
quote al \emph{controvalore totale} di  \Eur{5265,29}, il \emph{prezzo medio netto di
   vendita} risulta:
\begin{equation*}
  \num{5265,29} / \num{100} = \Eur{52,65}
\end{equation*}

Il  \emph{rendimento \underline{percentuale}  dell'operazione  di vendita},  rispetto
alle quote  in carico nel saldo  precedente, si calcola considerando  il \emph{prezzo
   medio di carico  nel saldo precedente} e il \emph{prezzo  medio netto di vendita};
in caso di  guadagno è un numero positivo,  in caso di perdita è  un numero negativo.
Per esempio, se il \emph{prezzo medio di carico nel saldo precedente} è \Eur{50,15} e
il  \emph{prezzo  medio   netto  di  vendita}  è   \Eur{52,65},  il  \emph{rendimento
   percentuale} risulta:
\begin{equation*}
  \RendimentoPercentuale{52,65}{50,15} = \SI{+4,9908}{\percent}
\end{equation*}

Il \emph{rendimento \underline{in valuta}  dell'operazione di vendita}, rispetto alle
quote in carico  nel saldo precedente, si calcola considerando  il \emph{prezzo medio
   di carico nel saldo precedente} e il \emph{prezzo medio netto di vendita}; in caso
di guadagno  è un  numero positivo,  in caso di  perdita è  un numero  negativo.  Per
esempio, se  si vendono \num{100}  quote con \emph{prezzo  medio di carico  nel saldo
   precedente}  pari a  \Eur{50,15} e  \emph{prezzo medio  netto di  vendita} pari  a
\Eur{52,65}, il \emph{rendimento in valuta} risulta:
\begin{equation*}
  \num{100} \times{} (\num{52,65} - \num{50,15}) = \Eur{250,29}
\end{equation*}

%page
\section{Aggiornamento del saldo dopo un'operazione di vendita}


Dopo l'esecuzione  di ogni operazione  di vendita  occorre ricalcolare il  saldo.  Il
saldo di un investimento in quote di ETF è rappresentato dalle tre quantità:
\begin{itemize}
\item numero di quote in carico nel Conto Titoli;
\item \emph{prezzo medio di carico nel saldo};
\item \emph{prezzo medio effettivo nel saldo};
\end{itemize}
sia il  \emph{prezzo medio di carico  nel saldo} che il  \emph{prezzo medio effettivo
   nel  saldo} restano  \textbf{immutati} dopo  una  vendita: essa  modifica solo  il
numero di quote in carico diminuendolo del numero di quote vendute.

Per esempio, si supponga di avere un Conto Titoli con saldo iniziale:
\begin{itemize}
\item numero di quote: \num{100};
\item \emph{prezzo medio effettivo nel saldo}: \Eur{50,00};
\item \emph{prezzo medio di carico nel saldo}: \Eur{50,15};
\end{itemize}
se  si  vendessero \num{90}  quote,  non  importa a  quale  prezzo,  il saldo  finale
risulterebbe:
\begin{itemize}
\item numero di quote: \(\num{100} - \num{90} = \num{10}\);
\item \emph{prezzo medio effettivo nel saldo}: \Eur{50,00}, immutato;
\item \emph{prezzo medio di carico nel saldo}: \Eur{50,15}, immutato.
\end{itemize}

%page
\section{Rendimento indicato nel riepilogo del patrimonio}


Nel sito di \emph{home banking} è indicato, per ogni fondo nel Conto Titoli, il saldo
costituito dal numero di quote in carico  e dal \emph{prezzo medio di carico}; in piú
è visibile  la stima dell'\emph{Utile  o Perdita}  (U/P), nell'ipotesi di  vendita di
tutte le quote al prezzo dell'ultima  quotazione rilevata sulla Borsa di negoziazione
del titolo:
\begin{itemize}
\item  l'\emph{Utile} mostrato  è al  lordo  sia del  \emph{costo dell'operazione  di
     vendita} che della \emph{tassa sul reddito da capitale};
\item  la \emph{Perdita}  mostrata è  al  lordo del  \emph{costo dell'operazione}  di
  vendita.
\end{itemize}

Si supponga di  possedere \num{100} quote al \emph{prezzo medio  di carico nel saldo}
di \Eur{50,15}:
\begin{itemize}
\item  se l'ultimo  rilevamento per  il  prezzo di  una quota  fosse di  \Eur{52,00},
  l'\emph{Utile percentuale} mostrato nella tabella del patrimonio risulterebbe:
  \begin{equation*}
    \RendimentoPercentuale{52,00}{50,00} = \SI{+3,6889}{\percent}
  \end{equation*}
  mentre l'\emph{Utile in valuta} risulterebbe:
  \begin{equation*}
    \num{100} \times{} (\num{52,00} - \num{50,00}) = \Eur{+185,00}
  \end{equation*}
  in realtà, vendendo tutte le quote al \emph{prezzo medio effettivo} di \Eur{52,00},
  risulterebbe:
  \begin{itemize}
  \item \emph{prezzo medio netto}: \Eur{51,33};
  \item \emph{rendimento percentuale}: \SI{+2,3434}{\percent};
  \item \emph{rendimento in valuta}: \Eur{+117,52};
  \end{itemize}

\item se  l'ultimo rilevamento per  il prezzo di una  quota fosse di  \Eur{48,00}, la
  \emph{Perdita percentuale} mostrata nella tabella del patrimonio risulterebbe:
  \begin{equation*}
    \RendimentoPercentuale{48,00}{50,00} = \SI{-4,2871}{\percent}
  \end{equation*}
  mentre la \emph{Perdita in valuta} risulterebbe:
  \begin{equation*}
    \num{100} \times{} (\num{48,00} - \num{50,00}) = \Eur{-215,00}
  \end{equation*}
  in realtà, vendendo tutte le quote al \emph{prezzo medio effettivo} di \Eur{48,00},
  risulterebbe:
  \begin{itemize}
  \item \emph{prezzo medio netto}: \Eur{47,85};
  \item \emph{perdita percentuale}: \SI{-4,5767}{\percent};
  \item \emph{perdita in valuta}: \Eur{-229,52}.
  \end{itemize}
\end{itemize}

%page
\section{Strategie per la scelta dei prezzi di vendita}


Si  cerca di  vendere a  un prezzo  maggiore del  prezzo di  acquisto; se  occorresse
liquidità per le proprie spese: si potrebbe essere costretti a disinvestire, vendendo
a prezzi inferiori.

%page
\subsection{Accoppiamento diretto tra acquisti e vendite}

Si cerca  di vendere  ogni singola quota  a un \emph{prezzo  medio netto  di vendita}
maggiore del suo \emph{prezzo medio di carico di acquisto}.  Per esempio, si supponga
di eseguire le operazioni:
\begin{enumerate}
\item acquisto di \num{101} quote al \emph{prezzo medio di carico di acquisto} pari a
  \Eur{51,00};
\item acquisto di \num{102} quote al \emph{prezzo  medio di carico di acquisto} pari a
  \Eur{52,00};
\item acquisto di \num{103} quote al \emph{prezzo  medio di carico di acquisto} pari a
  \Eur{53,00};
\item vendita  di \num{101}  quote a un  \emph{prezzo medio netto  di vendita}  pari a
  \Eur{55,10};
\item vendita  di \num{102}  quote a un  \emph{prezzo medio netto  di vendita}  pari a
  \Eur{55,20};
\item vendita  di \num{103}  quote a un  \emph{prezzo medio netto  di vendita}  pari a
  \Eur{55,30};
\end{enumerate}
per ogni gruppo di quote acquistate e poi vendute:
\begin{itemize}
\item accoppiando le operazioni \num{1} e \num{4}, il rendimento percentuale risulta:
  \begin{equation*}
    \RendimentoPercentuale{55,10}{51,00} = \SI{8,04}{\percent}
  \end{equation*}
  il rendimento in valuta risulta:
  \begin{equation*}
    \num{101} \times (\num{55,10} - \num{51,00}) = \Eur{414,10}
  \end{equation*}

\item accoppiando le operazioni \num{2} e \num{5}, il rendimento percentuale risulta:
  \begin{equation*}
    \RendimentoPercentuale{55,20}{52,00} = \SI{6,15}{\percent}
  \end{equation*}
  il rendimento in valuta risulta:
  \begin{equation*}
    \num{102} \times (\num{55,20} - \num{52,00}) = \Eur{326,40}
  \end{equation*}

\item accoppiando le operazioni \num{3} e \num{6}, il rendimento percentuale risulta:
  \begin{equation*}
    \RendimentoPercentuale{55,30}{53,00} = \SI{4,34}{\percent}
  \end{equation*}
  il rendimento in valuta risulta:
  \begin{equation*}
    \num{103} \times (\num{55,30} - \num{53,00}) = \Eur{236,90}
  \end{equation*}
\end{itemize}

Con questo  metodo, i  dati del saldo  sono usati per  calcolare: le  \emph{tasse sui
   redditi da  capitale}, le minusvalenze accumulate  e i \emph{prezzi medi  netti di
   vendita};  quindi tutti  i calcoli  \textbf{devono} essere  eseguiti.  Però,  ogni
operazione di acquisto è considerata separatamente dalle altre per quanto riguarda la
scelta del prezzo di vendita e la determinazione del conseguente rendimento.

%page
\subsection{Metodo delle medie ponderate}

Si cerca  di vendere le quote  a un prezzo  talmente maggiore del prezzo  di acquisto
che, recuperati i costi e pagate le tasse, si realizza un guadagno.
\begin{itemize}
\item Se il \emph{prezzo medio netto di vendita} è maggiore del \emph{prezzo medio di
     carico nel saldo} in quel momento: si  recuperano i costi di acquisto e vendita;
  si recuperano le tasse; si chiude con un guadagno.

\item Se il  \emph{prezzo medio netto di  vendita} è uguale al  \emph{prezzo medio di
     carico nel saldo} in quel momento: si  recuperano i costi di acquisto e vendita;
  si recuperano le tasse; ma resta nessun guadagno, si chiude in pareggio.

\item Se il \emph{prezzo  medio netto di vendita} è minore  del \emph{prezzo medio di
     carico  nel saldo}:  tolti i  costi di  acquisto e  vendita ed  eventualmente le
  tasse, l'operazione genera una perdita.
\end{itemize}
In ogni caso, restano le minusvalenze accumulate che, se e quando possibile, potranno
essere  usate per  ridurre  le tasse  da  pagare in  future  operazioni che  generano
\emph{redditi diversi}.

Si supponga di avere un Conto Titoli con il seguente saldo:
\begin{itemize}
\item numero quote: \num{100};
\item prezzo medio effettivo: \Eur{50,00};
\item costo medio per quota: \Eur{0,15};
\item prezzo medio carico: \Eur{50,15};
\item controvalore di carico: \Eur{5015,00};
\item minusvalenze accumulate: \Eur{0,00};
\end{itemize}
si vogliano vendere tutte le quote.

%% ------------------------------------------------------------------------

Come esempio di guadagno, si venda come segue:
\begin{itemize}
\item numero quote: \num{100};
\item prezzo medio eseguito: \Eur{52,00};
\item controvalore dell'operazione: \Eur{5200,00};
\item controvalore totale: \Eur{5132,52};
\item prezzo medio netto di vendita: \Eur{51,33};

\item reddito da capitale: \Eur{200,00};
\item tasse sul reddito: \Eur{52,00};
\item costo dell'operazione di vendita: \Eur{15,48};
\item costo convenzionale di acquisto delle quote vendute: \Eur{15,00};
\item reddito diverso: \Eur{-30,48};

\item rendimento percentuale: \SI{2,3434}{\percent};
\item rendimento in valuta: \Eur{117,52};
\end{itemize}
il \emph{prezzo medio netto di vendita} pari a\Eur{51,33} è maggiore del \emph{prezzo
   medio di carico  nel saldo} pari a \Eur{50,15}, quindi,  recuperati costi e tasse,
si realizza  un guadagno;  il reddito  diverso è una  minusvalenza da  compensare con
future operazioni.

%% ------------------------------------------------------------------------

Come esempio di perdita, si venda come segue:
\begin{itemize}
\item numero quote: \num{100};
\item prezzo medio eseguito: \Eur{50,30};
\item controvalore dell'operazione: \Eur{5030,00};
\item controvalore totale: \Eur{5007,13};
\item prezzo medio netto di vendita: \Eur{50,07};

\item reddito da capitale: \Eur{30,00};
\item tasse sul reddito: \Eur{7,80};
\item costo dell'operazione di vendita: \Eur{15,07};
\item costo convenzionale di acquisto delle quote vendute: \Eur{15,00};
\item reddito diverso: \Eur{-30,07};

\item rendimento percentuale: \SI{-0,1570}{\percent};
\item rendimento in valuta: \Eur{-7,87};
\end{itemize}
nonostante  il \emph{prezzo  medio  eseguito}  pari a  \Eur{50,30}  sia maggiore  del
\emph{prezzo medio  di carico nel  saldo} pari  a \Eur{50,15}, il  \emph{prezzo medio
   netto  di vendita}  di  \Eur{50,07} è  minore,  quindi, tolti  costi  e tasse,  si
realizza una perdita; il reddito diverso  è una minusvalenza da compensare con future
operazioni.

%% ------------------------------------------------------------------------

Come ulteriore esempio di perdita, si venda come segue:
\begin{itemize}
\item numero quote: \num{100};
\item prezzo medio eseguito: \Eur{48,00};
\item controvalore dell'operazione: \Eur{4800,00};
\item controvalore totale: \Eur{4785,48};
\item prezzo medio netto di vendita: \Eur{47,85};

\item reddito da capitale: \Eur{0,00};
\item tasse sul reddito: \Eur{0,00};
\item costo dell'operazione di vendita: \Eur{14,52};
\item costo convenzionale di acquisto delle quote vendute: \Eur{15,00};
\item redditi diversi: \Eur{-229,52};

\item rendimento percentuale: \SI{-4,5767}{\percent};
\item rendimento in valuta: \Eur{-229,52};
\end{itemize}
il \emph{prezzo medio netto di vendita}  pari a \Eur{47,85} è minore del \emph{prezzo
   medio di carico nel saldo} pari a  \Eur{50,15}, quindi si realizza una perdita; si
osserva che:
\begin{itemize}
\item il \emph{reddito da capitale} è zero e le tasse sono nulle;
\item  la differenza  tra  il  \emph{controvalore dell'operazione  di  vendita} e  il
  \emph{controvalore di carico nel saldo}:
  \begin{equation*}
    \num{4800,00} - \num{5015,00} = \Eur{-215,00}
  \end{equation*}
  rappresenta la perdita  di reddito dovuta alla vendita, è  negativa ed è registrata
  come reddito diverso;
\item   la  differenza   tra  il   \emph{controvalore   totale  di   vendita}  e   il
  \emph{controvalore di carico nel saldo}:
  \begin{equation*}
    \num{4785,48} - \num{5015,00} = \Eur{-229,52}
  \end{equation*}
  è negativa  e rappresenta l'intero  reddito diverso,  includendo sia la  perdita di
  reddito dovuta alla vendita che il costo delle operazioni;
\end{itemize}
il reddito diverso è una minusvalenza da compensare con future operazioni.

%% ------------------------------------------------------------------------

La  domanda  fondamentale  è:  \textbf{possedendo  un numero  di  quote  a  un  certo
   \emph{prezzo medio effettivo} e un certo  \emph{prezzo medio di carico} nel saldo,
   qual'è il  \emph{prezzo medio eseguito di  vendita} che permette di  realizzare un
   guadagno?}

Per esempio,  si posseggano  \num{100} quote  a un  \emph{prezzo medio  effettivo nel
   saldo} pari a  \Eur{50,00} e a un  \emph{prezzo medio di carico nel  saldo} pari a
\Eur{50,15},  si  vuole determinare  il  \emph{prezzo  medio \underline{eseguito}  di
   vendita} per  tutte le quote che  realizzi un guadagno.  Detto  \(X\) tale valore,
risulta:
\begin{itemize}
\item \emph{controvalore dell'operazione}: \(\num{100} \times{} X\);
\item \emph{costo dell'operazione di vendita}:
  \begin{align*}
    \num{0,50} + \num{2,50} + \num{0,0024} \times{} (\num{100} \times{} X)
    &= \num{3} + \num{0,0024} \times{} \num{100} \times{} X = \\
    &= \num{3} + \num{0,24} \times{} X
  \end{align*}
\item \emph{tassa sul reddito da capitale}:
  \begin{align*}
    \num{0,26} \times{} \num{100} \times{} (X - \num{50,00})
    &= \num{26} \times{} X - \num{26} \times{} \num{50} = \\
    &= \num{26} \times{} X - \num{1300}
  \end{align*}
\item \emph{controvalore totale di vendita}:
  \begin{align*}
    (\num{100} \times{} X)
    &- (\num{3} + \num{0,24} \times{} X)
      - (\num{26} \times{} X - \num{1300})
      = \\
    &= \num{100} \times{} X
      - \num{3} - \num{0,24} \times{} X
      - \num{26} \times{} X + \num{1300}
      = \\
    &= (\num{100} - \num{0,24} - \num{26}) \times{} X + (\num{1300} - \num{3})
      = \\
    &= \num{73,76} \times{} X + \num{1297}
  \end{align*}
\item \emph{prezzo medio netto di vendita}:
  \begin{equation*}
    \frac{\num{73,76} \times{} X + \num{1297}}{100}
  \end{equation*}
\end{itemize}
il pareggio si raggiungerebbe vendendo le quote a un \emph{prezzo medio netto} uguale
al \emph{prezzo medio di carico nel saldo} pari a \Eur{50,15}; quindi:
\begin{equation*}
  \frac{\num{73,76} \times{} X + \num{1297}}{100} = \num{50,15}
\end{equation*}
risolvendo rispetto a \(X\):
\begin{equation*}
  X = \frac{\num{100} \times{} \num{50,15} - \num{1297}}{\num{73,76}}
  = \Eur{50,4067}
\end{equation*}
vendendo con un \emph{prezzo medio eseguito} maggiore di \Eur{50,4067} si realizza un
guadagno.

%page
%% ------------------------------------------------------------
%% Fine.
%% ------------------------------------------------------------

\include{fdl-1.3}

%page
%% ------------------------------------------------------------
%% Fine.
%% ------------------------------------------------------------

\end{document}

%%% end of file
% Local Variables:
% mode: latex
% page-delimiter: "^%page"
% TeX-master: t
% ispell-local-dictionary: "italiano"
% fill-column: 85
% End:
